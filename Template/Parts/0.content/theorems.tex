\chapter {Bibliothèque Theorems}


\section{Création d'une question}

\begin{question}
    Quelle heure est-il ?
\end{question}

\lgreen{LOC}{Body}
\begin{Latex}{Code pour la création d'une question}
\begin{question}
    Quelle heure est-il ?
\end{question}
\end{Latex}

\section{Création d'une reponse}

\begin{reponse}
    il est 18 h.
\end{reponse}

\lgreen{LOC}{Body}
\begin{Latex}{Code pour la création d'une reponse}
\begin{reponse}
    il est 18 h.
\end{reponse}
\end{Latex}

\section{Création d'une propriete}

\begin{propriete}
    Un produit scalaire est commutatif.
\end{propriete}

\lgreen{LOC}{Body}
\begin{Latex}{Code pour la création d'une propriete}
\begin{propriete}
    Un produit scalaire est commutatif.
\end{propriete}
\end{Latex}

\section{Création d'une proposition}

\begin{proposition}
    Les chats sont des mammifères.
\end{proposition}

\lgreen{LOC}{Body}
\begin{Latex}{Code pour la création d'une proposition}
\begin{proposition}
    Les chats sont des mammifères.
\end{proposition}
\end{Latex}

\section{Création d'une remarque}

\begin{remarque}
remarque sur Latex
\end{remarque}

\lgreen{LOC}{Body}
\begin{Latex}{Code pour la création d'une remarque}
\begin{remarque}
remarque sur Latex
\end{remarque}
\end{Latex}

\section{Création d'un exemple}

\begin{exemple}
    Ceci est un exemple d'exemple
\end{exemple}

\lgreen{LOC}{Body}
\begin{Latex}{Code pour la création d'une exemple}
\begin{exemple}
    Ceci est un exemple d'exemple
\end{exemple}
\end{Latex}

\section{Création d'une définition}

\begin{definition}
    Une phrase est un ensemble de mots.
\end{definition}

\lgreen{LOC}{Body}
\begin{Latex}{Code pour la création d'une definition}
\begin{definition}
    Une phrase est un ensemble de mots.
\end{definition}
\end{Latex}

