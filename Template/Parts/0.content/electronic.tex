\chapter{Bibliothèque Electronic}

La bibliothèque \bold{Electronic} permet de générer des chronogrammes et des schémas électriques

\section{Création de chronogrammes fixes}

\begin{numeric}{Exemple 1 chronogramme fixe}
    D1 &  20{C}   \\
    D2 &  [green] 1H1L1L1L1H1L1L1H1L1H1L1H1L1H1L1H  \\
    D7 &  [black] 1H1L1L1L1H1L1L1H1L1H1L1H1L1H1L1H  \\
    D8 & 8D5U7U5D \\
    D9 & LLL 2{0.1H 0.1L} 0.6H HH \\
    D10 & ZZ G ZZ G XX G X \\
    D11 & [d] 4{5D{Text}} 0.2D \\
    D12 & [L][timing/slope=1.0] HL HL HL HL HL \\
  \end{numeric}


  \begin{numeric}{Exemple 2 - Chronogramme du compteur 4 bits}
    INPUT &  CC [blue]16{CC} CCC   \\
    D0 &  HL 8{LHHL} LHL   \\
    D1 &  H  4{LLLLHHHH} LLLL \\
    D2 &  H 2{LLLLLLLLHHHHHHHH} LLLL   \\
    D3 &  H{LLLLLLLLLLLLLLLLHHHHHHHHHHHHHHHH} LLLL  \\
    END &  LL [green]14{LL} LHHLLL  \\
    VALUE & L 2D{0} 2D{1} 2D{2} 2D{3} 2D{4} 2D{5} 2D{6} 2D{7} 2D{8} 2D{9} 2D{10} 2D{11} 2D{12} 2D{13} 2D{14} 2D{15} 2D{0} 2D{1}  \\
  \end{numeric}%

\lgreen{LOC}{Body}
\begin{Latex}{Code pour la création de chronogrammes fixes [exemple 1]}
  \begin{numeric}{exemple 1 chronogramme fixe}
    D1 &  20{C}   \\
    D2 &  [green] 1H1L1L1L1H1L1L1H1L1H1L1H1L1H1L1H  \\
    D7 &  [black] 1H1L1L1L1H1L1L1H1L1H1L1H1L1H1L1H  \\
    D8 & 8D5U7U5D \\
    D9 & LLL 2{0.1H 0.1L} 0.6H HH \\
    D10 & ZZ G ZZ G XX G X \\
    D11 & [d] 4{5D{Text}} 0.2D \\
    D12 & [L][timing/slope=1.0] HL HL HL HL HL \\
  \end{numeric}
\end{Latex}

\lgreen{LOC}{Body}
\begin{Latex}{Code pour la création de chronogrammes fixes [exemple 2]}
  \begin{numeric}{Exemple 2 - Chronogramme du compteur 4 bits}
    INPUT &  CC [blue]16{CC} CCC   \\
    D0 &  HL 8{LHHL} LHL   \\
    D1 &  H  4{LLLLHHHH} LLLL \\
    D2 &  H 2{LLLLLLLLHHHHHHHH} LLLL   \\
    D3 &  H{LLLLLLLLLLLLLLLLHHHHHHHHHHHHHHHH} LLLL  \\
    END &  LL [green]14{LL} LHHLLL  \\
    VALUE & L 2D{0} 2D{1} 2D{2} 2D{3} 2D{4} 2D{5} 2D{6} 2D{7} 2D{8} 2D{9} 2D{10} 2D{11} 2D{12} 2D{13} 2D{14} 2D{15} 2D{0} 2D{1}  \\
  \end{numeric}%
\end{Latex}


\section{Création de chronogrammes flottants}

Notre signal d'horloge (\texttiming{[blue]CCCCCC}) provient d'un oscillateur à quartz.
Notre signal d'horloge (\texttiming[timing/draw grid]{LHLHLHLHLHLHLHL}) provient d'un oscillateur à quartz. 

\lgreen{LOC}{Body}
\begin{Latex}{Code pour la création de chronogrammes flottants}
  Notre signal d'horloge (\texttiming{[blue]CCCCCC}) provient d'un oscillateur à quartz.
  Notre signal d'horloge (\texttiming[timing/draw grid]{LHLHLHLHLHLHLHL}) provient d'un oscillateur à quartz. 
\end{Latex}



\section{Création de schémas électriques}

  
  \begin{schema} {Exemple de schéma électrique}
  
    \addPower{6,5}{power1}{$+5V$}
    \addGround{4,0}{gnd1}{}
  
    \setDeviceBackgroundColor{white}
    \setRotate{0}
    \addLogicGate{0,0}{mynor}{nor}{}{A}{B}{G1}
  
    \setDeviceBackgroundColor{green}
    \addLogicGate{0,2}{mynand}{nand}{}{C}{D}{G2}
    \addLogicGate{2,1}{myor}{or}{}{}{}{G3}
    \resetColors
            
    \addTransistor{6,1}{npnA}{nmos}{B}{C}{E}
    \addTransistor{6,3}{pnpA}{pmos}{b}{e}{c}
  
    \resetColors
    \addTransistor{10,2}{npnR}{nmos}{b}{e}{c}
  
    \addWire{mynor.out}{myor.in 2}{\orthogonalWireA}
    \addWire{mynand.out}{myor.in 1}{\orthogonalWireA}

    \addWire{mynand.out}{pnpA.B}{\orthogonalWireA}
    \addWire{pnpA.C}{npnA.C}{\orthogonalWireA}
  
    \addWire{pnpA.E}{power1}{\orthogonalWireA}
  
    \addWire{npnA.E}{gnd1}{\orthogonalWireA}
  
    \addNode{$(pnpA.C)+(1,0)$}{node1}{}
    \addWire{pnpA.C}{node1}{\orthogonalWireA}
  
    \setDeviceBackgroundColor{red}
    \addLed{myor.out}{\Right}{npnA.B}{\orthogonalWireA}{L1}
    \addResistor{node1}{\Right}{npnR.B}{\orthogonalWireA}
            
  \end{schema}
  
  \lgreen{LOC}{Body}
  \begin{Latex}{Code pour la création de schémas électriques}
    \begin{schema} {Exemple de schéma électrique}
  
      \addPower{6,5}{power1}{$+5V$}
      \addGround{4,0}{gnd1}{}
    
      \setDeviceBackgroundColor{white}
      \setRotate{0}
      \addLogicGate{0,0}{mynor}{nor}{}{A}{B}{G1}
    
      \setDeviceBackgroundColor{green}
      \addLogicGate{0,2}{mynand}{nand}{}{C}{D}{G2}
      \addLogicGate{2,1}{myor}{or}{}{}{}{G3}
      \resetColors
              
      \addTransistor{6,1}{npnA}{nmos}{B}{C}{E}
      \addTransistor{6,3}{pnpA}{pmos}{b}{e}{c}
    
      \resetColors
      \addTransistor{10,2}{npnR}{nmos}{b}{e}{c}
    
      \addWire{mynor.out}{myor.in 2}{\orthogonalWireA}
      \addWire{mynand.out}{myor.in 1}{\orthogonalWireA}
  
      \addWire{mynand.out}{pnpA.B}{\orthogonalWireA}
      \addWire{pnpA.C}{npnA.C}{\orthogonalWireA}
    
      \addWire{pnpA.E}{power1}{\orthogonalWireA}
    
      \addWire{npnA.E}{gnd1}{\orthogonalWireA}
    
      \addNode{$(pnpA.C)+(1,0)$}{node1}{}
      \addWire{pnpA.C}{node1}{\orthogonalWireA}
    
      \setDeviceBackgroundColor{red}
      \addLed{myor.out}{\Right}{npnA.B}{\orthogonalWireA}{L1}
      \addResistor{node1}{\Right}{npnR.B}{\orthogonalWireA}
              
    \end{schema}
  \end{Latex}
  