\documentclass[11pt,french]{memoir}
  \usepackage{babel}
  %% Activer les lignes appropriées selon le moteur utilisé
  \usepackage[utf8]{inputenc}   % LaTeX
  \usepackage[T1]{fontenc}      % LaTeX
  % \usepackage{fontspec}         % XeLaTeX
  \usepackage[colorlinks]{hyperref}

  %% Numéroter jusqu'aux sous-sections
  \maxsecnumdepth{subsection}

  \title{Initiation au système de mise en page \LaTeX}
  \author{Vincent Goulet}

\begin{document}

\maketitle

\frontmatter

\tableofcontents

\mainmatter

\chapter{Présentation du langage R}
\label{chap:rpresentation}

\section{Bref historique}
\label{sec:rpresentation:historique}

À l'origine fut le S, un langage pour «programmer avec des données»
développé chez Bell Laboratories à partir du milieu des années 1970
par une équipe de chercheurs menée par John~M.\ Chambers. Au fil du
temps, le S a connu quatre principales versions communément
identifiées par la couleur du livre dans lequel elles étaient
présentées: %
version «originale», %
version 2, %
version 3 et %
version 4; %
voir aussi <citation> et <citation> pour plus de détails.

\subsection{Les années S-PLUS}

Dès la fin des années 1980 et pendant près de vingt ans, le S a
principalement été popularisé par une mise en {\oe}uvre commerciale
nommée S-PLUS. En 2008, Lucent Technologies a vendu le langage S à
Insightful Corporation, ce qui a effectivement stoppé le développement
du langage par ses auteurs originaux. Aujourd'hui, le S est
commercialisé de manière relativement confidentielle sous le nom
Spotfire S$+$ par TIBCO Software.

\subsection{L'arrivée de R}

Ce qui a fortement contribué à la perte d'influence de S-PLUS, c'est
une nouvelle mise en {\oe}uvre du langage développée au milieu des
années 1990. Inspirés à la fois par le S et par Scheme (un dérivé du
Lisp), Ross Ihaka et Robert Gentleman proposent un langage pour
l'analyse de données et les graphiques qu'ils nomment R. À la
suggestion de Martin Maechler de l'ETH de Zurich, les auteurs décident
d'intégrer leur nouveau langage au projet GNU\footnote{%
  \url{http://www.gnu.org}}, %
faisant de R un logiciel libre.

Ainsi disponible gratuitement et ouvert aux contributions de tous,
R gagne rapidement en popularité là même où S-PLUS avait acquis ses
lettres de noblesse, soit dans les milieux académiques. De simple
dérivé «\emph{not unlike S}», R devient un concurrent sérieux à
S-PLUS, puis le surpasse lorsque les efforts de développement se
rangent massivement derrière le projet libre. D'ailleurs John Chambers
place aujourd'hui ses efforts de réflexion et de développement dans le
projet R.


\section{Description sommaire de R}
\label{sec:rpresentation:description}

R est un environnement intégré de manipulation de données, de calcul
et de préparation de graphiques. Toutefois, ce n'est pas seulement un
«autre» environnement statistique (comme SPSS ou SAS, par exemple),
mais aussi un langage de programmation complet et autonome.

Tel que mentionné précédemment, le R est un langage principalement
inspiré du S et de Scheme. Le S était à son tour inspiré de plusieurs
langages, dont l'APL (autrefois un langage très prisé par les
actuaires) et le Lisp. Comme tous ces langages, le R est
\emph{interprété}, c'est-à-dire qu'il requiert un autre programme ---
l'\emph{interprète} --- pour que ses commandes soient exécutées. Par
opposition, les programmes de langages \emph{compilés}, comme le C ou
le C++, sont d'abord convertis en code machine par le compilateur puis
directement exécutés par l'ordinateur.

Cela signifie donc que lorsque l'on programme en R, il n'est pas
possible de plaider l'attente de la fin de la phase de compilation
pour perdre son temps au travail. Désolé!

Le programme que l'on lance lorsque l'on exécute R est en fait
l'interprète. Celui-ci attend que l'on lui soumette des commandes dans
le langage R, commandes qu'il exécutera immédiatement, une à une et
en séquence.

Par analogie, Excel est certes un logiciel de manipulation de données,
de mise en forme et de préparation de graphiques, mais c'est aussi au
sens large un langage de programmation interprété. On utilise le
langage de programmation lorsque l'on entre des commandes dans une
cellule d'une feuille de calcul. L'interprète exécute les commandes et
affiche les résultats dans la cellule.

Le R est un langage particulièrement puissant pour les applications
mathématiques et statistiques (et donc actuarielles) puisque
précisément développé dans ce but. Parmi ses caractéristiques
particulièrement intéressantes, on note:
\begin{itemize}
\item langage basé sur la notion de vecteur, ce qui simplifie les
  calculs mathématiques et réduit considérablement le recours aux
  structures itératives (boucles \texttt{for}, \texttt{while}, etc.);
\item pas de typage ni de déclaration obligatoire des variables;
\item programmes courts, en général quelques lignes de code seulement;
\item temps de développement très court.
\end{itemize}



\section{Interfaces}
\label{sec:rpresentation:interfaces}

R est d'abord et avant tout une application n'offrant qu'une invite de
commande du type de celle présentée à la
\autoref{fig:rpresentation:console}. En soi, cela n'est pas si
différent d'un tableur tel que Excel: la zone d'entrée de texte dans
une cellule n'est rien d'autre qu'une invite de commande\footnote{%
  Merci à Markus Gesmann pour cette observation.}, par ailleurs aux
capacités d'édition plutôt réduites.

\begin{figure}
  \centering
  \fbox{image}
  \caption{Fenêtre de la console sous macOS au démarrage de R}
  \label{fig:rpresentation:console}
\end{figure}

\begin{itemize}
\item Sous Unix et Linux, R n'est accessible que depuis la ligne de
  commande du système d'exploitation (terminal). Aucune interface
  graphique n'est offerte avec la distribution de base de R.
\item Sous Windows, une interface graphique plutôt rudimentaire est
  disponible. Elle facilite certaines opérations tel que
  l'installation de packages externes, mais elle offre autrement peu
  de fonctionnalités additionnelles pour l'édition de code R.
\item L'interface graphique de R sous macOS est la plus élaborée.
  Outre la console présentée à la
  \autoref{fig:rpresentation:console}, l'application \texttt{R.app}
  comporte de nombreuses fonctionnalités, dont un éditeur de code
  assez complet.
\end{itemize}

\appendix

\chapter{Installation de packages dans R}

Cette annexe explique comment installer et charger des extensions
(\emph{package}) dans R.

\end{document}
