%This file contains infomations about rules in document such as link to files, directories...


\section{Conventions}


\subsection*{Commandes}
Les commandes à saisir sont dans des encadrés similaires : \\

\begin{Bash}{Exemple de commande}
sudo apt-get update
\end{Bash}

Parfois, ces encadrés contiendront des instructions qu'il faudra placer dans certains fichiers.

%image commande
Toutes les commandes peuvent être saisies une fois que le prompt (l’invité de commande) est affiché : 
Chaque prompt est de la forme suivante : \\
\begin{Bash}{Invité de commande}
user@name-pc:~\$ 
\end{Bash}

où 	

\begin{itemize}
    \item \bold{user} représente l’identifiant de la personne connectée
    \item \bold{name-pc} représente le nom de la machine lorsqu’elle est visible sur le réseau
    \item \bold{$\sim$} représente le répertoire de travail de l’utilisateur ($\sim$=espace personnel)
    \item \bold{\$} veut dire que l'utilisateur est en invité standard. Si ce dernier est en  
super-utilisateur, le \$ devient un \# (mode administrateur) et il faut, dans ce cas, faire attention aux commandes que l’on tape, sous peine de détruire la machine.
\end{itemize}

\subsection*{Références}

\begin{itemize}
    \item Les fichiers sont indiqués par le repère \file{fichier}
    \item Les dossiers sont indiqués par le repère \dir{dossier}
    \item Les logiciels sont indiqués par le repère \lib{logiciel}\footnote{Sont également concernés les paquest Linux et les bibliothèques des langages}
    %\item Les liens sont indiqués par le repère \url{lien}
\end{itemize}

