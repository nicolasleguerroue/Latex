

     \chapter{Les transistors MOSFET}

     \section{Présentation}

     Nous avons vu l'utilisation des transistors bipolaires. \\
     Ces derniers sont assez contraignants à mettre en oeuvre car ils sont commandés en courant.

     Nous allons utiliser cette fois-ci la technologie des MOSFET \footnote{MOSFET : Metal Oxide Semiconductor Field Effect Transistor = Transistor à effet de champ à structure métal-oxyde-semiconducteur} car ces derniers ont l'avantage d'être contrôlés en \bold{tension}.

     Ce composant possède trois broches : 
     
     \begin{itemize}
     
       \item Le drain (D)
       \item la porte (G)\footnote{'G' pour Gate}
       \item la source (S)
     
     \end{itemize}
     
     \img{\rootImages/mosfet.png}{La représentation du transistor MOSFET}{0.1}
     
     \section{Conventions}
     
     Afin de simplifier les calculs par la suite, posons également les normes suivantes : 
     
     \begin{itemize}
     
       \item Le courant entrant dans le Drain est appelé $I_{D}$
       \item Le courant entrant dans la Porte est appelé $I_{G}$
       \item Le courant sortant de la Source est appelé $I_{S}$

       \item La tension entre la Porte et la Source est appelée $V_{GS}$
       \item La tension entre le Drain et la Source est appelée $V_{DS}$
     \end{itemize}
     
     \subsection{Les familles de transistors MOSFET}

     Les transistors MOSFET sont classés en deux catégories : 
     
     \begin{itemize}
     
       \item Les transistors MOSFET à canal N\footnote{Le nom de ces familles provient du type de jonction utilisé en interne. Pour plus de rensiegnement, consulter les diodes et semi-conducteurs}
       \item Les transitors MOSFET à canal P
     
     \end{itemize}
     
     Le principe de fonctionnement est similaire entre ces deux familles, seul le branchement et le niveau de commande diffère.\\
     Dans ce document, nous utiliserons essentiellement des transistors MOSFET à canal N car ces derniers utilisent des grandeurs positives.
     
     \img{\rootImages/mos.png}{Transistors à canal N et P}{0.6}

     \section{Les paramètres de sélection du transistor}
     
     Les paramètres de sélection de nos transistors MOSFET sont identiques aux transistors bipolaires, c'est à dire :
     
     \begin{itemize}
     
       \item La tension admissible sur $V_{DS}$ du transistor
       \item Le courant admissible entre le Drain et la Source.
     \end{itemize}
     
     Pour la suite de la présentation, on supposera que notre transistor a été dimensionné pour répondre à ces deux contraintes.


     \section{Le principe}

     Ce type de transistor fonctionne comme les transistors bipolaires mais est commandé en tension et non en courant.

     Par analogie, le drain joue le rôle du collecteur, la source celui de l'émetteur et la porte celui de la base.
     Le courant de l'élément à contrôler (moteur, résistance de puissance....) transite entre le drain et la source et la tension de commande est aux bornes de la porte.\\
     
     Les transistors MOSFET deviennent passant\footnote{Le transistor laisse passer tout le courant autorisé.} lorsque la tension sur la porte dépasse une tension de déclenchement appelée $V_{GS_{th}}$.
     Cette valeur est généralement comprise entre $2$ et $4$ Volts.\footnote{Lorsque la tension $V_{GS}$ est inférieure à $V_{GS_{th}}$, $I_D$ vaut $K \cdot ( (V_{GS}-V_{th})\cdot V_{DS}-\frac{1}{2}V_{DS}^2)$ \\Cette relation montre que l'étude en amplification est plus complexe car non linéaire.}\\

     \bold{Lorsque cette tension $V_{GS_{th}}$ est atteinte, notre transistor peut être remplacé d'un point de vue électrique entre le drain et la source par une résistance de très faible valeur, appelée $R_{DS_{on}}$}

    \section{Comparaison avec les transistors bipolaires}

    Par nature, la porte du MOSFET est vue comme un condensateur. Le transistor ne consomme pas de courant, excepté pendant les commutations.\\
    Ainsi, le courant est nul dans la porte pour maintenir le moteur en marche alors que pour un bipolaire, il faut maintenir un courant dans la base.\\

    Les MOSFET sont donc plus économes en énergie que les bipolaires.\\
    De plus, ils peuvent généralement supporter des courants plus importants que les bipolaires.\\

    En revanche, en hautes fréquences, les MOSFET sont moins réactifs du fait de leur capacité en entrée.
     \section{Mise en pratique}

     Nous souhaitons faire tourner le même moteur que celui utilisé avec notre transistor bipolaire.\\
     Nous allons le commander avec une carte Arduino.

     \subsection{Branchements}

     Tout d'abord, il convient de placer le moteur entre notre alimentation et le drain.\\
     
 
    \messageBox{Remarque}{cyan}{white}{{\color{black}Toutes les charges à contrôler avec ce type de transistor se placent entre l'alimentation et le drain.}}{black}
 
     Enfin, il ne nous reste plus qu'à relier une sortie numérique de l'Arduino vers notre porte \bold{sans} résistance.
 
     Nous obtenons donc le schéma suivant.
 
     \img{\rootImages/schema_mosfet.png}{Branchement du transistor MOSFET}{0.45}
  
     \subsection{Exemple de programme Arduino}
 
     Voici un code permettant de faire tourner le moteur périodiquement pendant 5 secondes puis de l'arrêter pendant 5 secondes. Il s'agit du même code que pour le transistor bipolaire.
 
 \begin{Cpp}{Code Arduino avec MOSFET}
 #define D8 8     //GATE du transistor
 
 void setup() {
 
   pinMode(D8, OUTPUT); //Mise en sortie de la broche
 
 }//End setup
 
 void loop() {
 
   digitalWrite(D8, HIGH);    //déclenchement du moteur
   delay(5000);               //Délai de 5s
   digitalWrite(D8, LOW);     //arrêt du moteur
   delay(5000);               //Délai de 5s
 
 }//End loop
 
 \end{Cpp}
