
\chapter{Montage suiveur }


\section{Présentation}


Ce montage permet de reproduire à l’identique une tension d'entrée. \\
L'intérêt de ce montage réside dans le fait que l’impédance d’entrée de l’AOP est considérée comme infinie et que son impédance de sortie est considérée comme nulle.\\

Ainsi, le comportement de la charge en entrée ne sera pas affecté par l’AOP, le signal d'entrée ne sera donc pas modifié. \\

Ce montage sert donc à faire une \bold{adaptation d’impédance}.

\section{Montage}

\img{\rootImages/suiveur.png}{Le montage suiveur}{0.6}


\section{Démonstration} 
La réaction négative implique que $\varepsilon=0$ (fonctionnement linéaire)
$$ E_+=Ve $$
$$ E_-=Vs $$

$$ \Rightarrow Ve=Vs$$ car $E_+=E_-$


%\section{Application}

\begin{exemple}
On souhaite mesurer une tension au borne d’un capteur avec un appareil de mesure.

\img{\rootImages/sensor.png}{Le capteur}{0.6}

On place ensuite une charge $R_c$ au bornes de $A$ et $b$. Cette résistance $R_c$ représente l'appareil d'acquisition.

\img{\rootImages/rc.png}{Le modèle d'acquisition}{0.3}
\end{exemple}

\begin{question}
Quelle est l’influence de $Rc$ sur $U_{AB}$ dans le montage suivant ?
\end{question}

\begin{reponse}

Sans la charge $Rc$ : 

$$ U_{AB}= \frac{U \cdot R_2}{R_1+R_2}$$

Avec  la charge $Rc$ :

$$ U_{AB}= \frac{U \cdot R_{equ}}{R_1+R_{equ}}$$


Avec $R_{equ}$ la résistance équivalente entre $R_2$ et $R_c$ \\



Si $R_c \rightarrow + \infty$ alors $R_{equ} \rightarrow \frac{U_{AB} \cdot R_2}{R_1+R_2}$ \\

Si $R_c \rightarrow + 0 $ alors $R_{equ} \rightarrow 0 \Rightarrow U_{AB} \rightarrow 0$

\end{reponse}


D'où le montage suivant, avec $R_c \rightarrow + \infty$, le signal n'est pas déformé.


\img{\rootImages/measure.png}{L'adaptation d'impédance}{0.8}

