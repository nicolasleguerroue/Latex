

\chapter{Montages comparateurs}
\section{Présentation}

Un montage comparateur se reconnaît par son branchement : 

\begin{itemize}
  \item {\color{red}Aucune contre réaction} n’est présente
  \item Une {\color{red} contre réaction a lieu sur l’entrée non inverseuse} via un dipôle passif
\end{itemize}

Le montage comparateur permet de comparer deux tensions entre elles. \\
Cependant, cette comparaison peut s’effectuer de plusieurs manières, avec un ou deux seuils, de manière inversée ou non...

\subsection{Comparateur non inverseur simple seuil}

\subsubsection{Présentation}

Ce montage permet de comparer simplement deux tensions entre elle.

\subsubsection{Montage}

\img{Images/aop/cmp.png}{Comparateur simple seuil}{0.7}

Ce mode est le plus simple et est régi de la manière suivante : \\

On sait que $\varepsilon = E_+ - E_-$ et que $V_s=\varepsilon \cdot A_d$ avec $Ad=+\infty$ \\
(Circuit en boucle ouverte) \\


Si $E_+>E_-$ :
$$Vs=Vsat_+$$ 
 si $E_+<E_-$ :
$$Vs=Vsat_-$$

Si $V_2=0V$, on obtient la caractéristique de transfert suivante  

\img{Images/aop/12.jpg}{Caractéristique de transfert, $V_1$ est la tension d’entrée de l’AOP}{0.3}


\subsection{Comparateur inverseur simple seuil}

Le raisonnement est le même sauf que les entrées sont inversées. \\
De ce fait, le seuil de basculement se fait dans l’autre sens. \\

Si $E_+>E_-$ :
$$Vs=Vsat-$$
Si $E_+<E_-$ :
$$V_s=Vsat_+$$

\subsection{Comparateur non inverseur double seuil}


\subsubsection{Présentation}

Ce type de montage permet d’éliminer les tensions “parasites”, c’est à dire les tensions bruités et non indésirables. \\

\subsubsection{Montage}

\img{Images/aop/double_cmp.png}{Montage comparateur non inverseur double seuil }{0.8}

\subsubsection{Application}

Par exemple, un capteur de lumière résistif (photo-résistance) sera sensible aux variations de lumière (nuages…). \\

\img{Images/aop/14.png}{Un signal bruité}{0.5}

Or, si on compare ce signal par rapport à une référence, on ne veut pas que le capteur déclenche plusieurs fois l’action.

\img{Images/aop/trig.png}{Le cycle de déclenchement}{1}

Pour éviter ce problème, on utilise un comparateur à double seuil : \\
toute les tensions parasites ayant une amplitude inférieure à la tension de différence entre les deux seuils seront ignorées.\\

\img{Images/aop/dual.png}{Le principe}{0.8}

On va chercher les deux valeurs de basculement : 

$$E_+ = \frac{V_eR_2 + V_s R_1}{R_1 + R_2}$$
$$E_- = U_0=0$$

Étudions le cas où $\varepsilon>0$

\begin{align}
\varepsilon>0 & \Leftrightarrow E_+ > E_-\\
& \Leftrightarrow \frac{V_eR_2 + V_s R_1}{R_1 + R_2} > U_0 \\
& \Leftrightarrow \frac{V_eR_2}{R_1 + R_2} > U_0 -  \frac{V_s R_1 }{R_1 + R_2} \\
& \Leftrightarrow  V_eR_2 > R_1 + R_2 \cdot  U_0 -  V_s R_1 \\
& \Leftrightarrow  V_e > \frac{R_1 + R_2 \cdot  U_0 -  V_s R_1}{R_2}
\end{align}

Ici, $U_0=0$ et $V_s=V_{sat_+}$ car $\varepsilon>0$

D'où $V_e > \frac{-V_{sat} + R_1}{R_2}$


\messageBox{Remarque}{orange}{white}{$U_0$ peut être différent de $0V$ en mettant une source de tension sur $E_-$}{black}


Étudions le cas où $\varepsilon>0$ : 

\begin{align}
\varepsilon>0 & \Leftrightarrow E_+ < E_-\\
& \Leftrightarrow \frac{V_eR_2 + V_s R_1}{R_1 + R_2} < U_0 \\
& \Leftrightarrow  \frac{V_eR_2}{R_1 + R_2} < U_0 -  \frac{V_s R_1 }{R_1 + R_2} \\
& \Leftrightarrow  V_eR_2 < R_1 + R_2 \cdot  U_0 -  V_s R_1 \\
& \Leftrightarrow  V_e < \frac{R_1 + R_2 \cdot  U_0 -  V_s R_1}{R_2}
\end{align}

Ici, $U_0=0$ et $V_s=V_{sat_-}$ car $\varepsilon<0$ \\

D'où $V_e < \frac{-V_{sat} + R_1}{R_2}$



On obtient deux seuils $S1$ et $S2$ de valeurs respectives : \\


$\frac{-V_{sat}+R1}{R2}$ et $\frac{Vsat - R_1}{R_2}$ \\



Afin de basculer, la tension d’entrée doit dépasser $S1 V$ et afin de basculer dans l’autre sens, la tension d’entrée doit être inférieure à $S2 V$ \\


On obtient le cycle d’hystérésis suivant : \\

$(VB=S2 et VH=S1)$

\img{Images/aop/hys1.png}{Cycle d'hystérésis non inverseur}{0.8}



$U_{milieu\_de\_cycle}=(V_{seuil1}+V_{seuil2}) \cdot 0.5$ \\
$Largeur_{cycle}=V_{seuil1}-V_{seuil2}$



\subsection{Comparateur inverseur double seuil}

\subsubsection{Montage}

\img{Images/aop/double_inv.png}{Montage comparateur inverseur double seuil}{0.5} %dede

\subsubsection{Démonstration}

La démarche est rigoureusement identique avec $\varepsilon>0$, on a $V_s=Vsat_+$
et pour  $\varepsilon<0$ on a $V_s=Vsat_-$

On obtient le cycle d’hystérésis suivant : \\


\img{Images/aop/hys2.png}{Cycle d'hystérésis inverseur}{0.8}