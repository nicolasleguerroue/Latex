


\chapter{Montage inverseur}
\section{Présentation}

Ce montage amplifie la tension $V_e$ par un \bold{gain $A_0$ négatif}.

\section{Montage}

\img{Images/aop/non_inv.png}{Le montage amplificateur non inverseur}{0.8}

\section{Démonstration}

Mode linéaire : $\varepsilon = 0$
\begin{align}
E_+&=0 \\
E_-&= \frac{ \frac{V_e}{R_1}+\frac{V_s}{R_2} } { \frac{1}{R_1} + \frac{1}{R_2}} \\
E_-&=\frac{V_e \cdot R_2 + V_s \cdot R_1}{R_1 + R_2} \\
\Rightarrow \frac{V_e}{V_s} &= -\frac{R_1}{R_2}\\
\Rightarrow V_s &= -V_e \cdot \frac{R_2}{R_1} 
\end{align}

Avec $A0= -\frac{R_2}{R1}$

\section{Application}

\begin{exemple}
On souhaite amplifier un signal sinusoïdal par un coefficient $k=-5$.\\
On peut donc utiliser le montage précédent. \\
On prendra $R_1=1 k\Omega$ et $R_2=5k\Omega$ pour avoir $A_0=-5$
\img{Images/aop/inv_signal.png}{Amplification du signal noir par -5}{0.4}

\end{exemple}
