%\usepackage[Lenny]{fncychap}   %Sonny, Lenny, Glenn, Conny, Rejne, Bjarne
%#############################################################
%#### Settings
%#############################################################
%#############################################################
%If you want to set on fullpage, change this bloc
%#############################################################
\geometry{hmargin=2cm,vmargin=2.5cm}
%#############################################################
%#############################################################
%If you want rename the chapter name, change the value of argument
%#############################################################

%#############################################################
%#############################################################
%If you want to add presentation, modify the next bloc
%The firt line is about the header : 
% {left content}{center content}{right content}
%The second line is about the footer : 
% {left content}{center content}{right content}
%####
%to get the current chapter name, use \currentChapter command as content
%to get the current number page, use \currentPagecommand as content
\addPresentation
{Alimentations linéaires} {} {\currentChapter}
{Club de Robotique et d'Electronique\\ Programmable de Ploemeur} {} {\currentPage}
%#############################################################
%Change the width of footer line and header line
%To delete it, set value to 0
\setHeaderLine{0.2}
\setFooterLine{0.2}

\setcounter{tocdepth}{2} %depth of table of content
\setcounter{secnumdepth}{2}

%#############################################################
%Change the name of the section "Nomenclature"
%To delete it, set value to 0
%\renewcommand{\nomname}{Conventions}
%Setting up the parameter of PDF file as name, author...
\begin{comment}
@input Titre du PDF
@input Auteur(s)
@input Sujet du fichier PDF (courte phrase)
@input Créateur du fichier PDF
@input Producteur du fichier PDF
@input Mots-clés (liste)
@input Couleurs des liens
@input Couleurs des citations dans la bibliographie
@input Couleurs des liens de fichier
\end{comment}
\setParameters {Alimentations linéaires} {Nicolas Le Guerroué} {Alimentations linaires} {Nicolas Le Guerroué}{Latex}{green}{blue}{blue}
  

