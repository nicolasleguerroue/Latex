
\chapter{Les familles des circuits logiques}

\section{Présentation}

Les circuits logiques sont des éléments permettant de réaliser des opérations avec l'algèbre de Boole (uniquement des '0' et des '1').
Pour réaliser ces opérations (ET, NON, OU, NON-OU, NON-ET...), des circuits ont vu le jours dans les années 60 avec deux grandes familles de circuits : \\
\begin{itemize}
    \item La famille TTL
    \item La famille CMOS
\end{itemize} 


De nouvelles technologies arrivent à maturité mais nous ne les évoquerons pas ici.

\subsection{Principe TTL}

Les circuits TTL sont composées de transistors bipolaires NPN ou PNP. \\
Les transistors bipolaires sont commandés en courant.


\subsection{Principe CMOS}

La famille CMOS, quant à elle, repose sur l'utilisation en interne de transistors MOS complémentaires (C). \\
Les transistors MOS, du fait de leur structure, sont commandés en tension.


\section{Comment les distinguer ?}

La famille des CMOS est rapidement identifiable car le nom du composant contient un numéro commençant par 40 et se termine avec un nombre à 2 ou 3 chiffres (40XX ou 40XXX). \\

Par exemple, CD4001, CD4017 sont des composants CMOS. \\



La famille des TTL contient en général le chiffre 74\footnote{La série militaire des TTL possède le numéro 54 et possède de meilleurs caractéristiques : plage de température de fonctionnement plus élevée, fréquence plus élevée...} encadré par des lettres et des chiffres. \\





\section{Avantages et inconvénients}

\begin{itemize}
    \item Les TTL consomment plus de courant que les CMOS \footnote{Ces derniers consomment uniquement lors des phases de commutation}
    \item Les CMOS ont des fréquences de fonctionnement plus faibles que les TTL.
\end{itemize}


