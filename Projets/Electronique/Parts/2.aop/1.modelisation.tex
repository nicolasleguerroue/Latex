
\chapter{Modélisation de l'AOP}
\section{Modèle théorique}

Dans un souci de simplification des calculs, un AOP peut être vu physiquement comme un composant ayant des caractéristiques parfaites. Ces caractéristiques sont les suivantes :

\begin{itemize}
  \item impédance d’entrée :  $Z\rightarrow+\infty \Omega$
  \item Impédance de sortie : $Z=0 \Omega$ 
  \item $Ad\rightarrow+\infty$
  \item Vsmax $ = Vcc_+$
  \item Vsmin $ = Vcc_-$
  \item Bande passante : $Fmax\rightarrow+\infty$
\end{itemize}

Cela se traduit par un modèle dont les deux entrées sont ouvertes (courant nul) et avec une tension de sortie qui ne serait pas affectée par le courant de sortie

\img{\rootImages/theorique.png}{Le modèle théorique de l'AOP}{0.6}

\section{Modèle réel} 


Cependant, il convient de noter que ce modèle n’est que théorique.\\

Du fait de la nature des composants constituant les AOP (transistors, condensateurs), la tension de sortie ne peut pas être égale à la tension d’alimentation. 

Cette tension de sortie max est appelée $Vsat_+$et $Vsat_-$ \\

D'où le modèle suivant : 

\begin{itemize}
  \item impédance d’entrée :  $Z>10^5 \Omega$
  \item Impédance de sortie : $Z>0 \Omega$ (courant de sortie max $20mA$)
  \item $Ad \gg 1000$
  \item $Vs_{max}$ =$Vcc_{sat_+}$
  \item $Vs_{min}$ =$Vcc_{sat_-}-$
  \item Bande passante : imposée par le constructeur (Ex : le LM324 tolère 1MHz)
\end{itemize}


\img{\rootImages/real.png}{Le modèle réel de l'AOP}{0.6}


Cependant, pour les calculs, {\color{red}l’hypothèse du courant d’entrée} nul sera retenue, tout comme celle du gain $Ad$


\section{ Modes de fonctionnement}


\subsection{Montages linéaires }
Le signal $V_s$ est une fonction mathématique du signal d’entrée $V_e$. \\

Dans certains cas, le signal de sortie conserve la forme du signal d’entrée sous couvert que que l’AOP ne rentre pas en saturation. \\


Un {\color{red}montage linéaire impose un $\varepsilon$ nul, sauf si l’AOP est saturé}, c’est à dire si $V_s=V_{sat}$




\subsection{Montages comparateurs}

Le signal de sortie ne peut prendre que deux valeurs, $Vsat_+$ou $Vsat_-$
En l’absence de contre-réaction, $Vs=\varepsilon Ad$ \\

\bold{Une réaction est un retour du signal sur une des deux entrées. Celle ci peut être positive ou négative en fonction de l’entrée choisie (entrée -,réaction négative et entrée +, réaction positive).}

\img{\rootImages/52.png}{Les modes de fonctionnement de l'AOP}{0.6}

\section{Résistance de charge}

Il est possible de mettre une résistance de charge entre $V_s$ et la masse. \\

Cette résistance symbolise un circuit relié directement à l'AOP. \\

Cependant, le courant de l’AOP étant limité à quelques dizaines de mA, il convient de prendre une résistance de charge $R_c$ suffisamment grande ($Rc>1000 \omega$). \\

Si $R_{c}$ est suffisamment élevée, cette dernière n’influence pas la tension de sortie $V_s$ \\


\img{\rootImages/load.png}{La résistance de charge}{0.8}

