

\chapter{Montage soustracteur}
\section{Présentation}
Ce montage permet de soustraire deux tensions d’entrée afin d’obtenir la différence en sortie.

\section{Montage}

\img{Images/aop/sous.png}{Montage soustracteur}{0.6}

\section{Démonstration}


Contre-réaction négative donc mode linéaire.\\


$$E_+=U_2\frac{R_4}{R_2+R_4}$$

\begin{align}
E_-&=\frac{\frac{U_1}{R_1}+\frac{U_5}{R3}}{\frac{1}{R_1}+\frac{1}{R_3}}\\
&=\frac{U_1R_3+U_5R_1}{R_1+R_3}
\end{align}

Or $E_+=E_-$

\begin{align}
&\Leftrightarrow U_2 \cdot \frac{R_4}{R_2+R_4} = \frac{U_1R_3+U_5R_1}{R_1+R_3} \\
&\Leftrightarrow  \frac{U_5R_1}{R_1+R_3} = \frac{U_2R_4}{R_2+R_4} -\frac{U_1R_3}{R_1+R_3} \\
&\Leftrightarrow  U_5R_1 = \frac{R_4(R_1+R_3)}{R_2+R_4} - \frac{U_1R_3}{R_1} 
\end{align}


Si $R_1=R_2=R_3=R_4$, on obtient : \\

$U_s=U_2-U_1$\\

\section{Application}

\begin{exemple}
On souhaite mesurer une tension entre deux points A et B d’un circuit (tension différentielle)
\img{Images/aop/variable.png}{Tension différentielle $AB$}{0.4}
\end{exemple}


Pour étudier la différence de potentiel entre les deux points du circuit, on peut utiliser un montage soustracteur afin qu’en sortie du montage avec l’AOP, on ait : $$V_s=V_a-V_b$$

On peut réaliser le montage suivant.

\img{Images/aop/diff.png}{Un montage pour lire une tension entre deux points}{0.55}

