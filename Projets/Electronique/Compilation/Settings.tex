%\usepackage[Lenny]{fncychap}   %Sonny, Lenny, Glenn, Conny, Rejne, Bjarne
%#############################################################
%#### Settings
%#############################################################
%#############################################################
%If you want to set on fullpage, change this bloc
%#############################################################
\geometry{hmargin=2cm,vmargin=2.5cm}
%#############################################################
%#############################################################
%If you want rename the chapter name, change the value of argument
%#############################################################
\setAliasChapter{Section}
%#############################################################
%#############################################################
%If you want to add presentation, modify the next bloc
%The firt line is about the header : 
% {left content}{center content}{right content}
%The second line is about the footer : 
% {left content}{center content}{right content}
%####
%to get the current chapter name, use \currentChapter command as content
%to get the current number page, use \currentPagecommand as content
\addPresentation
{Synthèse Latex} {} {\currentChapter}
{Bibliothèque Utils} {} {\currentPage}
%#############################################################
%Change the width of footer line and header line
%To delete it, set value to 0
\setHeaderLine{0.2}
\setFooterLine{0.2}

\setcounter{tocdepth}{2} %depth of table of content
\setcounter{secnumdepth}{2}

%#############################################################
%Change the name of the section "Nomenclature"
%To delete it, set value to 0
\renewcommand{\nomname}{Conventions}
%Setting up the parameter of PDF file as name, author...
\begin{comment}
@input Titre du PDF
@input Auteur(s)
@input Sujet du fichier PDF (courte phrase)
@input Créateur du fichier PDF
@input Producteur du fichier PDF
@input Mots-clés (liste)
@input Couleurs des liens
@input Couleurs des citations dans la bibliographie
@input Couleurs des liens de fichier
\end{comment}
\setParameters {Tutoriel Latex} {Nicolas Le Guerroué} {Bibliothèque Utils} {Nicolas Le Guerroué}{Latex}{green}{blue}{blue}
  


\titleformat{\chapter}
  {\gdef\chapterlabel{}
   \normalfont\sffamily\Huge\bfseries\scshape}
  {\gdef\chapterlabel{\thechapter)\ }}{0pt}
  {\begin{tikzpicture}[remember picture,overlay]
    \node[yshift=-2cm] at (current page.north west)
      {\begin{tikzpicture}[remember picture, overlay]
        \draw[fill=MediumBlue] (0,0) rectangle
          (\paperwidth,2cm);
        \node[anchor=east,xshift=.9\paperwidth,rectangle,
              rounded corners=20pt,inner sep=11pt,
              fill=white]
              {\color{black}#1};%\chapterlabel
          \node[anchor=west,yshift=1cm,xshift=2cm,inner sep=11pt,
              fill=MediumBlue]
              {\color{white}{\large Synthèse Latex}};%\chapterlabel
       \end{tikzpicture}
      };
   \end{tikzpicture}
  }
  \titlespacing*{\chapter}{0pt}{50pt}{-60pt}


