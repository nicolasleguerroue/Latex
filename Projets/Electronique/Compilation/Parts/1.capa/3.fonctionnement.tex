
\chapter{Principe de fonctionnement}

Un condensateur est un dipôle électrique composé de deux armatures conductrices appelées électrodes et séparées par un matériel isolant ou diélectrique.

\img{\rootImages/condensateur.jpg}{schéma condensateur}{1}

Lorsque l'on exerce une tension sur un condensateur, une force électrique déplace des électrons vers la première armature pour s'y déposer. Cette augmentation du nombre d'électrons vient chargée négativement l'armature. Une force se créée entre les deux plaques et vient arracher des électrons à la seconde armature et donc charger positivement l'armature. \\ 

Malgré la présence d'un isolant entre les deux plaques, le courant dans le circuit n'est pas nul. En effet, tant que le condensateur n'est pas chargé ,un nombre d'électrons arrive vers le condensateur. Le nombre d'électrons arrivant sur la première plaque est égale au nombre d'électrons quittant la seconde plaque. Des charges sont arrachées à la seconde armature et continuent de se déplacer dans le circuit.  \\ 

Le condensateur continue de se charger tant que la tension entre les deux armatures n'est pas égale à la tension exercée sur ses bornes. Si on exerce une tension sur le condensateur supérieur à sa tension admissible, le composant va casser ou exploser. Lorsque le condensateur est complètement chargé, les nouveaux électrons arrivant sont repoussés par ceux déjà présents sur la plaque, il n'y a plus déplacement de charges , le courant devient nul. \\

Un condensateur est caractérisé par un coefficient de proportionnalité entre la charge et la tension à ses bornes. On note ce coefficient capacité électrique et il s'exprime en Farad.

$$ Q = C(V_1 - V_2) \;\; ou \;\; i = C \frac{du}{dt}$$ \\

La capacité d'un condensateur est déterminée par la géométrie du composant et la nature de l'isolant. \\

Le tableau suivant résume la valeur de capacité pour différente géométrie de condensateur. $ \varepsilon_r $ représente la permittivité relative de l'isolant.\\


\img{\rootImages/capacitor.png}{calcul capacité}{0.8}

