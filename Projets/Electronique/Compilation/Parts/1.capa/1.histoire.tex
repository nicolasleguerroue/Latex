
\chapter{Histoire} 

C’est en 1745, que le physicien allemand Ewald Georg von Kleist invente le premier condensateur. Il a enroulé une feuille d'argent autour d'une bouteille en verre,  et  a  chargé  la  feuille  à  l'aide  d'un  générateur  à  friction. Il était convaincu  qu'une  charge  pourrait  être  accumulée  lorsqu'il  a  reçu  un  choc  électrique  significatif (par un générateur par exemple). \\

Un an plus tard, Pieter van Musschenrboek poursuivra les recherches sur cette invention et lui donnera le nom de : “bouteille de Leyde”. Pour la petite histoire, ce nom vient de l’université où travaillait ce dernier : l’université de Leyde. Le condensateur est une véritable révolution car il permet de contenir une importante charge électrique dans un très petit volume.  La bouteille de Leyde est un condensateur formé de deux conducteurs séparés par le verre de la bouteille. Le premier conducteur est généralement constitué d'une électrode supérieure, reliée à des feuilles en étain mises dans  la  bouteille.  Le  second  conducteur  est  formé  par  une feuille  métallique  autour la  bouteille. Ces deux conducteurs permettent de créer deux charges égales mais de signes opposées. \\


\img{\rootImages/layde.jpg}{Bouteille de Leyde}{1.1} 

Puis avec le temps, d'autres condensateurs ont vu le jour : 

\img{\rootImages/condo_type.jpg}{Différents types de condensateurs}{0.8
}
