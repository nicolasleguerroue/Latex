



\chapter{Montage intégrateur}
\section{Présentation}
Ce montage intègre une tension d’entrée. \\
En sortie, on obtient une tension $V_s$ valant $V_s=k \cdot \int V_e$

\section{Montage}

\img{Images/aop/integrateur.png}{Montage intégrateur}{0.6}

\section{Démonstration}

Montage en mode linéaire car contre-réaction négative. \\


\begin{align}
I_{R1} + I_{C1}  = 0 & 
& \Leftrightarrow I_{R1} = -I_{C1} \\
& \Leftrightarrow \frac{E}{R} = - C \cdot \frac{dV_s}{dt} \\
& \Leftrightarrow -\frac{1}{RC}\cdot E = \frac{dV_s}{dt} \\
& \Leftrightarrow V_s = -\frac{1}{RC} \int V_e \\
k=-\frac{1}{RC} &
\end{align}
   
Ce montage est notamment présent dans certains Convertisseurs Analogiques Numériques dit “simple” ou “double” rampe. 


\section{Application}

\begin{exemple}
On souhaite générer un signal triangulaire.
\end{exemple}


En intégrant un signal rectangulaire, on obtient un signal triangulaire.


\img{Images/aop/triangle.png}{Un signal triangulaire généré depuis un signal carré}{0.4}

Soit $V_e=2V$ ou $V_e=-2V$

Si $V_e=2V$ \\

\begin{align}
k \int V_e &= k V_e\cdot t + x\\
&= - \frac{2}{RC}t+c 
\end{align}

avec $k=-\frac{1}{RC} $ \\

$\Rightarrow$ droite d'équation $y=-\frac{2}{RC}+c$ \\


Si $V_e=-2V$ \\

\begin{align}
k \int V_e &= k V_e\cdot t + x \\
&= \frac{2}{RC}t+c 
\end{align}

avec $k=-\frac{1}{RC} $ \\

$\Rightarrow$ droite d'équation $y=\frac{2}{RC}+c$

