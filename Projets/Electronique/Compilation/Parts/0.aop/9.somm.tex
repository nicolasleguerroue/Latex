\chapter{Montage sommateur}

\section{Présentation}
Ce montage permet d'additionner en sortie plusieurs tensions d’entrée. Avec ce montage, la tension de sortie est multipliée par un coefficient -1

\section{Montage}
\img{Images/aop/somme.png}{Montage sommateur inverseur}{0.6}

\section{Démonstration}

Contre-réaction négative donc montage linéaire. \\
On applique le théorème de Millman \footnote{on fera abstraction de U3}

$$E_+=0$$
$$ E_- = \frac{ \frac{U_1}{R_1} + \frac{U_2}{R_1} + \frac{U_s}{R_1} }{\frac{1}{R_1} + \frac{1}{R_1} + \frac{1}{R_1}}=0$$

\begin{align}
&\Leftrightarrow \frac{U_1+U_2+U_s}{R_1}=0 \\
&\Leftrightarrow \frac{U_s}{R_1} = \frac{-(U_1+U_2)}{R_1} \\
&\Leftrightarrow U_s = -(U_1+U_2)
\end{align}

