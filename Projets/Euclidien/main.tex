 \documentclass[12pt]{report}  
%book, report, beamer, article
\usepackage[frenchb]{babel}     %french language
\usepackage{Utils/Utils}        %Utils package

%\usepackage[Lenny]{fncychap}   %Sonny, Lenny, Glenn, Conny, Rejne, Bjarne
%#############################################################
%#### Settings
%#############################################################
%#############################################################
%If you want to set on fullpage, change this bloc
%#############################################################
\geometry{hmargin=2.5cm,vmargin=2.5cm}
%#############################################################
%#############################################################
%If you want rename the chapter name, change the value of argument
%#############################################################
\setAliasChapter{Section}
%#############################################################
%#############################################################
%If you want to add presentation, modify the next bloc
%The firt line is about the header : 
% {left content}{center content}{right content}
%The second line is about the footer : 
% {left content}{center content}{right content}
%####
%to get the current chapter name, use \currentChapter command as content
%to get the current number page, use \currentPagecommand as content
\addPresentation
{Circuits de puissance} {} {\currentChapter}
{Club de Robotique et d'Electronique\\ Programmable de Ploemeur} {} {\currentPage}
%#############################################################
%Change the width of footer line and header line
%To delete it, set value to 0
\setHeaderLine{0.2}
\setFooterLine{0.2}

%#############################################################
%Change the name of the section "Nomenclature"
%To delete it, set value to 0
\renewcommand{\nomname}{Conventions}
%Setting up the parameter of PDF file as name, author...
\begin{comment}
@input Titre du PDF
@input Auteur(s)
@input Sujet du fichier PDF (courte phrase)
@input Créateur du fichier PDF
@input Producteur du fichier PDF
@input Mots-clés (liste)
@input Couleurs des liens
@input Couleurs des citations dans la bibliographie
@input Couleurs des liens de fichier
\end{comment}
\setParameters {Les Circuits de puissance} {Nicolas LE GUERROUE} {Transistors, MOSFETS, Bipolaires, relais} {Nicolas LE GUERROUÉ}{Circuits de puissance}{green}{blue}{blue}
  

\addtocounter{tocdepth}{1}



\titleformat{\chapter}
  {\gdef\chapterlabel{}
   \normalfont\sffamily\Huge\bfseries\scshape}
  {\gdef\chapterlabel{\thechapter)\ }}{0pt}
  {\begin{tikzpicture}[remember picture,overlay]
    \node[yshift=-2cm] at (current page.north west)
      {\begin{tikzpicture}[remember picture, overlay]
        \draw[fill=MediumBlue] (0,0) rectangle
          (\paperwidth,2cm);
        \node[anchor=east,xshift=.9\paperwidth,rectangle,
              rounded corners=20pt,inner sep=11pt,
              fill=white]
              {\color{black}#1};%\chapterlabel
          \node[anchor=west,yshift=1cm,xshift=2cm,inner sep=11pt,
              fill=MediumBlue]
              {\color{white}{\large Synthèse Latex}};%\chapterlabel
       \end{tikzpicture}
      };
   \end{tikzpicture}
  }
  \titlespacing*{\chapter}{0pt}{50pt}{-60pt}







\begin{document}

\setHeader{Générateur de documentation}{CHARLES Mathieu \and Mathieu FAGET \and Nicolas LE GUERROUÉ \and Titouan LESEC-ROLLAND}{juin 2020}

\maketitle
\tableofcontents

\newpage
\chapter{Mise en équation}

\section{Objectif}

L'objectif est de prédire l'évolution de la température sur une barre de fer. 

\section{Contexte}

Nous possédons les données de température sur 99 intervalles de temps. Ces données concernent un pas de temps évidemment petit car nous souhaitons prédire la température.

\section{Un premier modèle de température}

A partir du fichier mesures précises ,
en mettant en relation l'évolution de température, le temps et les positions sur la barre, on a crée un modèle de température.

\img{Images/temperature.png}{Modèle de température}{0.5}

On obtient l'équation suivante
$$
T(x, t+\Delta t) - T(x,t) = C \times ( (T(x+\Delta x, t)-T(x,t)) + (T(x-\Delta x,t)-T(x,t)))
$$

avec \( \Delta t \) l'intervalle de temps entre les deux mesures,
et \( \Delta x \) un déplacement infinitésimal de la position du capteur de température . \newline
\newpage
On a une relation de proportionnalité entre  \( \Delta t \),\( \Delta x \) et C, d'où :

$$
\frac{\partial T}{\partial t} = \alpha \frac{\partial^2 T}{\partial x^2}
$$


Par lecture graphique sur le fichier des températures, on a :
$$
\alpha =  0.049
$$

\section{L'équation résolvable}
Cependant, nous ne savons pas résoudre cette équation. L'objectif suivant sera de chercher une équation plus simple que nous pourrons calculer. \newline
On cherche donc une équation de la forme suivante
$$
T(x,t) = f(t)\times g(x)
$$
\newline
Nous allons poser \( T(x,t) = f(t)\times g(x) \). \newline
De ce fait, l'équation de la chaleur devient :
$$
f'(t)\times g(x) = \alpha f(t)\times g''(x)
$$

avec \(\frac{f'(t)}{\alpha f(t)}\) constante,
que nous appelerons \(\beta\).


La première équation devient alors :
$$g''(x) = \beta g(x)$$
on en déduit la deuxième :
$$f'(t) = \alpha \beta f(t)$$

Solution de la deuxième équation :
$$f(t) = C \times e^{\alpha \beta t}$$

nous cherchons à avoir une combinaison linéaire qui sera pour nous une solution pour se rapprocher au mieux de la réalité .
on prendra donc un 
\(\beta\) strictement négatif puisque la température sera ammenée à décroître.
Ici, nous sommes dans le troisième cas :
$$
g(x) = C_1\mbox{cos}(\sqrt{-\beta}x) + C_2\mbox{sin}(\sqrt{-\beta}x)$$

De plus, sous savons que g(0)=0, ce qui impose que \(C_1=0\). Il reste donc
$$ g(x) = C_2\mbox{sin}(\sqrt{-\beta}x)$$

Or g(2)=0,on a donc
$$ C_2\sin(2\sqrt{-\beta}) = 0 $$
Comme les solutions de \(\sin \phi =0\) sont les \(\phi=k\pi\) avec \(k\) entier, on en déduit
que \(2\sqrt{-\beta} = k\pi\) , donc que les valeurs possibles pour \(\beta\) sont
$$
\beta = -\frac{k^2\pi^2}{4}
$$ pour tout k entier.

 \(T(x,t)=f(t)g(x)\), nous permet de trouver une famille de solutions
possibles, tous les :
$$
e^{-\alpha\frac{k^2\pi^2}{4} t} \sin(\frac{k\pi}{2} x)
$$
pour k entier.
Or nous savons également que la combinaison linéaire de ces éléments est aussi solution , alors nous avons 
l'expression possible suivante :
$$
e^{-\alpha\frac{\pi^2}{4} t} \sin(\frac{\pi}{2} x), 
e^{-\alpha\frac{4\pi^2}{4} t} \sin(\frac{2\pi}{2} x), 
e^{-\alpha\frac{9\pi^2}{4} t} \sin(\frac{3\pi}{2} x), 
e^{-\alpha\frac{16\pi^2}{4} t} \sin(\frac{4\pi}{2} x), 
...
$$
avec les conditions initiales (t=0) on a :
$$
\alpha_1 e^{-\alpha\frac{\pi^2}{4} t} \sin(\frac{\pi}{2} x) +
\alpha_2 e^{-\alpha\frac{4\pi^2}{4} t} \sin(\frac{2\pi}{2} x) +
\alpha_3 e^{-\alpha\frac{9\pi^2}{4} t} \sin(\frac{3\pi}{2} x) +
...
=
\alpha_1 \sin(\frac{\pi}{2} x) +
\alpha_2 \sin(\frac{2\pi}{2} x) +
\alpha_3 \sin(\frac{3\pi}{2} x) +
...
$$



On cherche à calculer les coefficient alpha

On part de la relation suivante
$$
\alpha_i =  \int_{0}^{2} M(x)\cdot sin(\frac{i\cdot \pi}{2}) dx
$$

avec i represantant le numéro du coefficient $ \alpha $.

\newpage 


\chapter{Solution}
\section{Programme Python}

Nous avons codé un programme Python pour calculer automatiquement les coefficients alphas. 
\newline
Ce programme possède une fonction main qui permet de :

\begin{enumerate}
    \item calculer autant de coefficient alpha que souhaité
    \item Afficher la température de la barre à un t donnée (tableau des temps)
    \item Calculer et afficher l'écart de température entre t(0) et t(final) du tableur
    \item Calculer et afficher l'écart de température entre t(0) et t(variable) avec les coefficients calculés
\end{enumerate}


Le code source (classe Data) est disponible en annexe du dossier.
Voici la fonction main

\begin{Python}
def main():
    #Création d'un objet Data
	data = Data() 													
	data.accuracy(15)   											#Nombre de alphas
	data.alpha(0.0495) 												#Alpha
	
	#Lis les valeurs à t=0
	data.readCSV("mesuresPrecises.csv", step=0)
	
	#calcul des coefficients alphas
	data.compute() 	
	
	#Affiche le graphique des températures réelles et calculées avec les alphas
	data.displayGraph()  
	
	#Affiche le graphique de la température à t=...
	#data.displayGraphTime(0.1)
	
	
	#Affiche le graphique des températures passée en argument
	data.displayGraphTimes([0.0098, 0.2, 1, 2,  5, 20])  	
	
    #Affiche la différence de température entre t(0) et t(final) du tableau
    #et la différence de température entre t(0) et t(0.0098)
	data.displayDelta(0.0098)

if __name__ == '__main__':
	main()
\end{Python}

\section{Prédictions et résultats}

\img{Images/alphas.PNG}{Calcul des coefficients alphas}{0.8}
\img{Images/g1.PNG}{Vérification du modèle à t=0}{0.9}
\img{Images/g2.PNG}{Prédictions des températures}{0.9}
\img{Images/g3.PNG}{Vérification des résultats}{0.3}
La courbe bleue représente la différence de température de la barre enter t=0 et t=0.0098s



\end{document}
