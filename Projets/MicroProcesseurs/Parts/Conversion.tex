\chapter{Conversion des nombres}     

{\faCheckCircle} Lessive \\
{\faCheckCircle} Lessive \\



\section{Coder un nombre dans la norme IEEE754}

A travers un exemple, nous allons détailler la conversion de deux nombre dans la norme IEEE754. \\
Ces deux nombres seront $75.375$ et $-75.375$. Nous allons voir que la technique pour coder les nombres positifs ou négatifs est identique.

\subsection{Signe du nombre}

Le premier bit de la trame IEEE754 correspond au signe du nombre à coder.

\begin{itemize}
	\item 0 si le nombre est \bold{positif}
	\item 1 si le nombre est \bold{négatif}

\end{itemize}

Dans notre cas ($75.375$), le premier bit vaut \bin{0}

\subsection{La mantisse}

La mantisse correspond à la valeur significative du nombre. \\ 
Nous allons normaliser la partie entière de la mantisse afin d'avoir un nombre de la forme $n = 1.M \cdot 2^E$ \\

avec 
\begin{itemize}

\item $E$ l'exposant
\item $M$ la partie décimale de la mantisse.
\end{itemize}



\subsubsection{La partie entière}

On convertit la \bold{valeur absolue} de la partie entière en binaire. Pour cela, on peut utiliser la technique des puissances de 2.

Dans 75, il y a $ 1\cdot 2^6, 1\cdot 2^3, 1\cdot 2^1 et 1\cdot 2^0$ \\

On a donc $(75)_{10} = \bin{1001011}$



\subsubsection{La partie décimale}

On effectue la technique des multiplications par deux. \\

\begin{itemize}

	\item On prend notre partie décimale que l'on multiplie par deux. \\

$0.375 \cdot 2 = \aze{0}.75$ \\

	\item Ensuite, on prend la partie entière du résultat que l'on met de coté. \\

La partie décimale est mise à la ligne et est de nouveau multipliée par deux. \\

$0.75 \cdot 2 = \aze{1}.5$ \\

\italic{On exécute cet algorithme tant que le résultat est différent de 1.}

\end{itemize}

$0.5 \cdot 2 = \aze{1}$ \\

Ici, le résultat vaut $1$, la conversion est terminée. 


Les parties entières obtenues sont classées par pondérations décroisantes, c'est à dire que la première partie entière représente le bit de poids fort de la partie décimale. \\

Ainsi, $(0.375)_{10} = \bin{011}$

\subsection{Normalisation de la mantisse}

Le nombre $75.375$ codé de façon \bold{non normalisée} vaut \bin{1001011,011}

Nous allons normaliser ce nombre. \\

Pour cela, on décale la virgule de $n$ emplacement(s) vers la gauche afin d'obtenir un seul \bold{'1'} à gauche de la virgule. \\

Pour un décalage de $n$ emplacement(s), on multiplie le nombre binaire par $2^n$ \\

Retenir l'exposant \aze{n} pour la suite. \\

\bin{1001011,011} normalisée vaut donc $\aze{1},001011011 \cdot 2^6 $ 

\subsection{Gestion de l'exposant}

\subsubsection{Biais de l'exposant}

Pour un format simple précision 32 bits, il convient d'ajouter à l'exposant la valeur $127 (2^7-1)$

Notre exposant normalisé vaut donc $6+127=(133)_{10} = \bin{10000101}$

\subsection{Mise en place des composantes de la trame}

Nous avons nos trois éléments : \bold{le signe, la partie décimale de la mantisse et l'exposant normalisés}

On place d'abord le bit de signe puis l'exposant sur 8 bit et enfin la partie décimale de la mantisse sur 23 bits. \\

Notre partie décimale de la mantisse vaut donc \bin{001011011}. Pour mettre sur 23 bit (protocole), il suffit de compléter avec autant de 0 nécéssaires en bit de poids faibles. \\

La partie décimale de la mantisse sur 23 bits vaut donc $\bin{00101101100000000000000}$ \\


Le nombre binaire vaut donc : \\

$N_{IEEE} = \bitgreen{0} 
\bitblue{1} \bitblue{0} \bitblue{0} \bitblue{0} \bitblue{0} \bitblue{1} \bitblue{0} \bitblue{1}     
\bitorange{0} \bitorange{0} \bitorange{1} \bitorange{0} \bitorange{1} \bitorange{1}\bitorange{0} \bitorange{1} 
\bitorange{1} \bitorange{0}  \bitorange{0}  \bitorange{0}  \bitorange{0} \bitorange{0} \bitorange{0} \bitorange{0} \bitorange{0} \bitorange{0} \bitorange{0} \bitorange{0} \bitorange{0} \bitorange{0} \bitorange{0} $

Avec \\
\color{green}{Le signe} \\
\color{blue}{L'exposant} \\
\color{orange}{La partie décimale de la mantisse} \\
\color{black}{}

\subsection{Conversion en hexadécimal}

On regroupe les bits par paquets de 4 et on convertit les paquets en hexadécimal.

$N_{IEEE} = 
\bitred{0} \bitred{1} \bitred{0} \bitred{0} \, %4
\bitred{0} \bitred{0} \bitred{1} \bitred{0} \, %2
\bitred{1} \bitred{0} \bitred{0} \bitred{1} \, %9
\bitred{0} \bitred{1} \bitred{1} \bitred{0} \, %6
\bitred{1} \bitred{1} \bitred{0} \bitred{0} \, %C
\bitred{0} \bitred{0} \bitred{0} \bitred{0} \, %0
\bitred{0} \bitred{0} \bitred{0} \bitred{0} \, %0
\bitred{0} \bitred{0} \bitred{0} \bitred{0} $  %0

On obtient au final : \\

$(+75.375)_{10} = \hexa{0x4296C000}$


\subsection{Cas des valeurs négatives}

Pour les nombres négatifs, seuls le bit de signe change dans la trame sur 32 bits.

$N_{IEEE} = 
\bitred{1} \bitred{1} \bitred{0} \bitred{0} \, %C
\bitred{0} \bitred{0} \bitred{1} \bitred{0} \, %2
\bitred{1} \bitred{0} \bitred{0} \bitred{1} \, %9
\bitred{0} \bitred{1} \bitred{1} \bitred{0} \, %6
\bitred{1} \bitred{1} \bitred{0} \bitred{0} \, %C
\bitred{0} \bitred{0} \bitred{0} \bitred{0} \, %0
\bitred{0} \bitred{0} \bitred{0} \bitred{0} \, %0
\bitred{0} \bitred{0} \bitred{0} \bitred{0} $  %0

D'où: \\

$(-75.375)_{10} = \hexa{0xC296C000}$
