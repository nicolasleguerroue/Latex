
\chapter{Introduction}


\section{Connexion SFTP}

\begin{Bash}{Connexion SFTP}
sftp://nico@192.168.1.22
a2ensite crep.conf 00.tab -> mettre à jour les fichier
\end{Bash}

\section{Présentation}
Ce tutoriel vous permettra d'envoyer des mails via un serveur SMTP. \\
Le serveur SMTP sera un serveur GMail. \\

Il sera possible d'envoyer des mails de manière automatique via un script Bash ou PHP.

La version utilisée de PHP pour ce document est la 7.4

\section{Installation}

Le serveur de mail est Postfix. Il convient donc de mettre les paquets à jour et d'installer le paquet \bold{postfix} 

\begin{Bash}{Mise à jours et installation du paquet}
sudo apt-get update
sudo apt-get install -y postfix
\end{Bash}

Lors de l'installation du paquet \bold{postfix}, on vous demande le type de configuration du serveur. Veuillez choisir le type \bold{Site internet} 

\img{Images/select.png}{Choix du serveur Postfix}{0.6}



\chapter{Édition du fichier \\de configuration}



Nous allons définir un ensemble de propriétés dans le fichier de configuration de Postfix. \\
Ce fichier, appelé \bold{main.cf}, se trouve à l'emplacement \bold{/etc/postfix} \\

Tout d'abord, on commence par modifier le fichier de configuration. \\
La ligne concernée est celle contenant le mot clé \bold{relayhost}.


\begin{Bash}{Fichier de configuration \bold{/etc/postfix/main.cf}}
relayhost = [smtp.gmail.com]:587
\end{Bash}


Ensuite, on ajoute, à la fin du fichier les directives suivantes : \\
\begin{Bash}{Edition du fichier main.cf}
#... code précédent

# Activer l'authentication SASL
smtp_sasl_auth_enable = yes
# Disallow methods that allow anonymous authentication
smtp_sasl_security_options = noanonymous
# Location of sasl_passwd
smtp_sasl_password_maps = hash:/etc/postfix/sasl/sasl_passwd
# Enable STARTTLS encryption
smtp_tls_security_level = encrypt
# Location of CA certificates
smtp_tls_CAfile = /etc/ssl/certs/ca-certificates.crt
\end{Bash}

\chapter{Beep}

\begin{Bash}{Commandes Beep}
sudo apt-get install -y beep
sudo modprobe -a pcspkr
beep -f 440 -d 1000
\end{Bash}


