\chapter{PhpMyAdmin}

\section{Présentation}
Administrer une base de données peut s'avérer compliqué. Afin de le faire plus facilement, l'utilitaire PhpMyAdmin est là pour nous aider. \\
Cela évite de rédiger les requêtes fastidieuses de MySql, il nous suffit juste de cocher des cases et de remplir les champs appropriés. \\
\section{Installation}
Veuillez sasir la commande suivante : 
\begin{Bash}{Installation de PhpMyAdmin}
sudo apt-get install -y phpmyadmin
\end{Bash}

Lors de l’installation, un mot de passe vous sera demandé : ce mot de passe sera utilisé pour se connecter à PhpMyAdmin. \\

Ensuite, le gestionnaire d’installation va vous demander votre serveur HTTP.
Sélectionnez \bold{Apache2}

\messageBox{Avertissement}{orange}{white}{test}{black}




\section{Emplacement de PhpMyAdmin}

\begin{Bash}{Emplacement de PhpMyAdmin}
cd /usr/share/phpmyadmin
\end{Bash}


\section{Configuration de PhpMyAdmin}

\begin{Bash}{Configu PhpMyAdmin}
#sudo mysql --user=root

#Dans MySql :
#DROP USER 'root'@'localhost';
#CREATE USER 'root'@'localhost' IDENTIFIED BY 'password';
#GRANT ALL PRIVILEGES ON *.* TO 'root'@'localhost';


\end{Bash}


