\chapter{PHP, Un langage serveur}

\section{Présentation}
Pour un site Web, toutes les personnes qui se connectent au serveur via un navigateur Web (Firefox, Chrome…) sont appelées clients. \\
Il faut distinguer un langage qui est exécuté du côté client et un langage qui est exécuté du côté serveur. \\

Les langages clients sont par exemple le langage HTML,Javascript, Mysql \\
Les langages serveurs sont Apache, Php ou Mysql. \\

Lorsqu’un client se connecte sur un site, il envoie une requête à ce dernier. Cette requête passe par Apache qui demande à Php de générer la page HTML. \\
La page est envoyée au client et est affichée du côté client. \\


\subsection{Installation}
Pour installer PHP sur le serveur, il faut saisir :

\begin{Bash}{Installation de Php}
sudo apt-get install -y php7.0
\end{Bash}

Par la même occasion, nous allons installer le programme permettant de communiquer entre la base de données et le serveur

\begin{Bash}{Installation du gestionnaire de base de données PHP}
sudo apt-get install -y php-mysql
\end{Bash}


\section{Modifier le fichier de configuration de php}
\begin{Bash}{Fichier php.ini}
sudo nano /etc/php/7.X/apache2/php.ini
\end{Bash}



\section{Augmentation de la taille des fichiers de chargements}

Il faut lire la fonction phpinfo() pour lire la variable :
puis modifier le fichier :

\begin{Bash}{Fichier php.ini}
sudo nano /etc/php/7.X/apache2/php.ini
\end{Bash}


\begin{Bash}{Fichier php.ini}
; Maximum size of POST data that PHP will accept.
; Its value may be 0 to disable the limit. It is ignored if POST data reading
; is disabled through enable_post_data_reading.
; http://php.net/post-max-size
post_max_size = 8M
\end{Bash}

puis
\begin{Bash}{Fichier php.ini}
; Maximum amount of memory a script may consume (128MB)
; http://php.net/memory-limit
memory_limit = 128M
\end{Bash}

puis 
\begin{Bash}{Fichier php.ini}
; Maximum allowed size for uploaded files.
; http://php.net/upload-max-filesize
upload_max_filesize = 100M
\end{Bash}
