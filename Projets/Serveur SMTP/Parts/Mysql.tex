
\chapter{MySql}

\section{Présentation}
Un site qui commence à s’étoffer et à prendre de l’ampleur se verra un jour ou l’autre obligé d’utiliser une base de données pour traiter les informations et les stocker de manière efficace. 
Certaines  bases de données sont qualifiées de relationnelles car les éléments stockés entretiennent entre eux des points communs comme des noms, des catégories, des nombres… \\

La plupart des logiciels de gestion de base de données sont basés sur la relation client-serveur.\\
C’est à dire que pour le serveur fasse correctement son travail (stocker et récupérer des données), il faut un logiciel client qui dialogue avec le serveur. \\
Pour des raisons de facilités, nous installerons le logiciel client et serveur sur le même ordinateur, c'est à dire sur le serveur.

Aujourd’hui, de nombreux logiciels de gestion de base de données cohabitent : 

\begin{itemize}
    \item MariaDB
    \item MySql
    \item SQLite
    \item ...
\end{itemize}

\section{Installation}
Dans le cas de ce tutoriel, nous installerons MySql.

Il faut saisir les deux commandes suivantes : 

\begin{Bash}{Installation de MySql}
sudo apt-get install -y mysql-client
sudo apt-get install -y mysql-server
\end{Bash}

Selon la version de la distribution, un mot de passe peut être demandé ou non.
Par exemple, pour une distribution Ubuntu antérieur à la 18.04, un mot de passe est demandé, le cas échéant, le mot de passe par défaut est celui du compte root du serveur. \\
Dans tous les cas, retenez bien le mot de passe pour la suite.

\section{Connexion}

Deux méthodes de connexion sont possibles, en fonction de la distribution: 

\begin{Bash}{Lancement de MySql}
mysql -u root -p
\end{Bash}

-u 	représente l’utilisateur : root
-p 	indique que l’utilisateur va saisir un mot de passe

\begin{Bash}{Lancement de Mysql}
sudo mysql
\end{Bash}

Une fois le mot de passe saisi, un invité de commande apparaît

Le prompt mysql> indique que MySql a été correctement installé.

Pour ceux qui se sont connectés à MySql via > sudo mysql , il n’est pas recommandé de changer de mot de passe.

Pour les autres cas, vous pouvez changer de mot de passe en saisissant 
\begin{Bash}{Modifier un compte}
ALTER USER 'root'@'localhost' IDENTIFIED WITH mysql_native_password BY 'nouveau_mot_de_passe';
\end{Bash}

Il vous faudra donc vous connectez à votre compte Mysql avant de saisir la commande. 


Nous allons ensuite créer un compte qui possède des droits étendue :
Dans MySql \\


\begin{Bash}{Création d'un compte}
CREATE USER 'user'@'localhost' IDENTIFIED BY 'password';
GRANT ALL PRIVILEGES ON *.* TO 'user'@'localhost';
\end{Bash}


Vous pouvez remplacer user par un identifiant de votre choix.
password est à remplacer par le mot de passe de votre choix. \\

Enfin, si vous souhaitez supprimer un compte, il suffit de saisir
\begin{Bash}{Suppression d'un utilisateur Mysql}
DROP USER 'user'@'localhost';
\end{Bash}


