\documentclass[12pt]{report}  
%book, report, beamer, article
\usepackage[frenchb]{babel}     %french language
\usepackage{Utils/Utils}        %Utils package
%#############################################################
%#### Settings
%#############################################################
%#############################################################
%If you want to set on fullpage, change this bloc
%#############################################################
\geometry{hmargin=3cm,vmargin=3cm}
%#############################################################
%#############################################################
%If you want rename the chapter name, change the value of argument
%#############################################################
\setAliasChapter{Section}
%#############################################################
%#############################################################
%If you want to add presentation, modify the next bloc
%The firt line is about the header : 
% {left content}{center content}{right content}
%The second line is about the footer : 
% {left content}{center content}{right content}
%####
%to get the current chapter name, use \currentChapter command as content
%to get the current number page, use \currentPagecommand as content
\addPresentation
{Générateur de documentation} {} {\currentChapter}
{Nicolas LE GUERROUE} {} {\currentPage}
%#############################################################
%Change the width of footer line and header line
%To delete it, set value to 0
\setHeaderLine{0.2}
\setFooterLine{0.2}
%#############################################################
%Change the name of the section "Nomenclature"
%To delete it, set value to 0
\renewcommand{\nomname}{Conventions}
%Setting up the parameter of PDF file as name, author...
\begin{comment}
@input Titre du PDF
@input Auteur(s)
@input Sujet du fichier PDF (courte phrase)
@input Créateur du fichier PDF
@input Producteur du fichier PDF
@input Mots-clés (liste)
@input Couleurs des liens
@input Couleurs des citations dans la bibliographie
@input Couleurs des liens de fichier
\end{comment}
%\setParameters {Tutoriel Latex} {Nicolas Le Guerroué} {Tutoriel Latex pour la mise en place des outils} {Nicolas Le Guerroué}{Latex, Linux}{blue}{blue}{green}{blue}
\setParameters {Tutoriel Latex} {Nicolas Le Guerroué} {Tutoriel Latex pour la mise en place des outils} {Nicolas Le Guerroué}{Latex}{green}{blue}{blue}
  








