\documentclass[12pt]{report}    %Type de document
\chapter{Bibliothèque Header}

\section{Mise en forme de la page de garde avec une image}

\lgreen{LOC}{Header}
\begin{Latex}{Code pour la mise en forme de la page de garde avec une image}
\setHeaderImage{Emplacement_image}{0.8}{Titre}{sous-titre}{Auteurs}{\today \\ \pageref{LastPage} pages}
\end{Latex}


\section{Mise en forme de la page de garde sans image}

\lgreen{LOC}{Header}
\begin{Latex}{Code pour la mise en forme de la page de garde sans image}
  \setHeader{Titre}{Auteur 1 \\ Auteur 2}{Date}
\end{Latex}

\section{Mise en forme de la page des parties}

\lgreen{LOC}{Body}
\begin{Latex}{Code pour la mise en forme de la page des parties}
  \partImg{Partie}{Images/file.png}{0.2}
\end{Latex}

\section{Ajout d'un trait entre l'en-tête et le corps de la page}

\lgreen{LOC}{Header}
\begin{Latex}{Code pour l'ajout d'un trait entre l'en-tête et le corps de la page}
  \setHeaderLine{0.2}
\end{Latex}

\section{Ajout d'un trait entre le corps de la page et le bas de page}

\lgreen{LOC}{Header}
\begin{Latex}{Code pour l'ajout d'un trait entre le corps de la page et le bas de page}
  \setFooterLine{0.2}
\end{Latex}





\section{Définition de la présentation globale des pages}

\lgreen{LOC}{Header}
\begin{Latex}{Code pour la définition de la présentation globale des page}
  \addPresentation
  {Titre} {Centre} {\currentChapter}
  {Gauche} {} {\currentPage}
\end{Latex}

\section{Redéfinition des titres des chapitres}

%\setAliasChapter{Section}
\lgreen{LOC}{Header}
\begin{Latex}{Code pour la redéfinition des titres des chapitres}
  \setAliasChapter{Section}
\end{Latex}


\section{Mettre le document en pleine page}
\lgreen{LOC}{Header}
\begin{Latex}{Code pour mettre le document en pleine page}
\setFullPage
\end{Latex}

\section{Récuperer le chapitre courant}
\lgreen{LOC}{Body}
\begin{Latex}{Code pour récuperer le chapitre courant}
\currentChapter
\end{Latex}


%Début du document
\begin{document}

%Page de garde
\title{Automatisme S4P ENIB}
\author{CHARLES Mathieu \and GAUTIER Nicolas \and LE GUERROUÉ Nicolas}  
%Génère la table des matières
\maketitle 
\tableofcontents

%###########################################################################################################################
%###################### NEW PAGE ###########################################################################################
%###########################################################################################################################


\chapter{Introduction}
\section{Objectifs}


Ce projet vise à mettre en place une interface de supervision en langage \emph{Python}.\newline
Ainsi, une interface en Python sera réalisée pour un contrôle de l’API (Automate Programmable Industriel) à distance. \newline
Nous avons opté pour la réalisation de notre propre logiciel de supervision pour plusieurs raisons : \newline

\begin{enumerate}
\item Contrôle fin de la communication API-Ordinateur
\item Indépendance aux logiciels propriétaires (PC-Vue….) \newline
\end{enumerate} 

Cependant, nous devons être conscients que la place des éléments graphiques est plus complexe.

\section{Supports pédagogiques}

Ce projet vise à donner suite à l'expérimentation de la création d'une interface Python pour la supervision. \newline
Le logiciel précédent, permettant de contrôler un vérin à distance, était basé sur la bibliothèque Python TKinter. \newline
Le protocole de communication restera inchangé, seul le logiciel pour l'interface sera différent.



%###########################################################################################################################
%###################### NEW PAGE ###########################################################################################
%###########################################################################################################################

\chapter{Pré-requis}
\section{Logiciels}

Afin de bien débuter, il convient que quelques logiciels soient préalablement installés.
Il s'agit de : \newline


\begin{enumerate}
\item \textbf{Python} \newline
La version minimal requise est la 3.7 sous peine de mauvaises installations et compatibilités entre les logiciels. \newline
Le logiciel est disponible à l'adresse \url{https://www.python.org/downloads} \newline
\item \textbf{Bibliothèque Python Pip} \newline 
Pip est une bibliothèque permettant d'installer d'autres bibliothèques Python. Toutes les bibliothèques nécessaires au projet de supervision sont disponibles via Pip. \newline
La bibliothèque est disponible à l'adresse \url{https://pip.pypa.io/en/stable} \newline

Il est également possible d'installer pip via Anaconda (ouvrir un terminal ou console) :
\begin{lstlisting}
conda install pip
\end{lstlisting}

\item \textbf{Logiciel Unity Pro} \newline 
Unity Pro sera utilisé pour générer le grafcet de supervision. \newline 
\end{enumerate}

%###########################################################################################################################
%###################### NEW PAGE ###########################################################################################
%###########################################################################################################################

\chapter{Logiciels et outils}


L'interface graphique sera réalisée avec le logiciel QtDesigner, en complément de la bibliothèque Python PyQt5.
Le protocole de communication se basera sur la bibliothèque Python PyModBus.
L'API sera programmé avec le logiciel Unity-Pro

\section{PyQt5}

\subsection{Présentation}

Qt est une bibliothèque originellement développée pour le C++. Elle a pour vocation d'aider à développer des interfaces graphiques utilisateurs (GUI en anglais) et propose de nombreux outils pour manipuler des fichiers, des bases de données etc...\newline

Elle est distribuée sous deux versions : \newline
- l'une est commerciale et nécessite de payer un abonnement régulier\newline
- l'autre est distribuée sous licence LGPL (c'est cette version qui a été utilisée dans notre projet)
\smallSkip
Les deux versions proposent les mêmes fonctionnalitées à version identique.

\smallSkip
PyQt5 n'est autre qu'un portage de la bibliothèque Qt C++ dans sa version 5 sur Python.\newline
A noter que PyQt5 dispose lui aussi d'une version commerciale mais que nous avons utilisé la version distribuée sous licence GPL.

\subsection{Pourquoi ?}

Nous avons choisi Qt pour de nombreuses raisons :
\begin{enumerate}
    \item \textcolor{red}{\bold{Multi-plateforme}}\newline PyQt5 a l'énorme avantage d'être portable sur de très nombreuses plateformes. Un seul programme peut être utilisé sur mobile comme sur Windows ou Linux.\newline
    Par exemple, dans notre cas, nous avons programmé de manière indifférente sur Windows ou Linux pendant toute la durée du projet.
    
    \item \textcolor{red}{\bold{Polyvalence}}\newline Comme dit un peu plus haut, PyQt5 propose de nombreux outils permettant la gestion de base de données, de fichiers xml, d'interfaces graphiques, de programmes multi-thread et bien d'autres choses encore. Cette polyvalence permet d'éviter d'utiliser plusieurs bibliothèque différente. Dans ce cas, une seule bibliothèque permet de faire énormément de choses.
    
    \item \textcolor{red}{\bold{Licence}}\newline Nous l'avons aussi expliqué un peu plus haut, PyQt5 est disponible sous licence GPL.V3 ce qui permet de l'utiliser gratuitement (sous les conditions imposées par le licence GPLv3)
    
    \item \textcolor{red}{\bold{Expérience}}\newline Plusieurs membres du groupe ont déjà utilisés Qt à plusieurs reprises. Ainsi, même si nous utilisions la version Python, le temps d'accoutumance à cette version particulière était plus court que si nous avions dû apprendre à utiliser TKinter.
\end{enumerate}

\section{QtDesigner}

\subsection{Présentation}
QtDesigner est un logiciel pour créer des interfaces graphiques.\newline
L'interaction entre les différents éléments graphiques appelés "widgets" sera faite en langage Python. 
Un widget est un élément grapĥique comportant des propriétés telles qu'une couleur, une taille, etc. \newline \newline


QtDesigner permet donc de gagner en efficacité en terme de conception graphique. Ceci est valable pour les élements graphiques simples (bouton, champ de saisi...) mais à partir du moment ou nous voulons créer des élements graphiques plus complexes (tables d'animations...), nous passons par de la programmation python. \newline



\section{PyModBus}
\subsection{Présentation}
PyModBus est une bibliothèque Python pour communiquer entre des périphériques avec le protocole ModBus analogue à celui de l'API.

\subsection{Améliorations et documentation}

Cependant, pour des questions de robustesse et de facilité, nous utiliserons un module (ENIBSupervision) que nous avons développé pour simplifier les communications entre l'API et l'ordinateur. \newline
Ce module utilise la bibliothèque PyModBus en interne.
 Ainsi, toutes les fonctions d'écriture/lecture des données sont protégées des mauvaises manipulations de programmation. \newline
Un tutoriel de prise en main du module \textit{ENIBSupervision} est disponible en annexe.

\subsection{Installation}

Il est possible d'installer la bibliothèque Pymodbus de deux manières : \newline

\subsubsection{Installation par pip}
La commande pour installer Pymodbus est la suivante :
\begin{lstlisting}
pip install pymodbus
\end{lstlisting}

Pour une installation sur les machines de l'ENIB, il est nécessaire de préciser le proxy pour l'installation. \newline 

La commande suivante est à saisir dans un terminal :
\begin{lstlisting}
pip install --proxy "http://proxy.enib" pymodbus
\end{lstlisting}

\subsubsection{Installation par Anaconda}
 Il est également possible d'installer Pymodbus avec Anaconda : 
\begin{lstlisting}
conda install -c auto pymodbus
\end{lstlisting}

La bibliothèque officielle ainsi que l'ensemble de sa documentation est disponible à l'adresse \url{https://pymodbus.readthedocs.io/en/latest/} \newline 

\section{Pyuic5}

\subsection{Présentation}

Une fois l'interface graphique est réalisée à l'aide de QtDesigner, il faut utiliser \bold{pyuic5} pour convertir le fichier \ePath{.ui} en fichier python exploitable.\newline
%Même si 1/4 de l'espace de la page est dédié aux notes de pied de page ?
%J'exagère, 1/6 * Page 15 docteur
%non, c'est assez pratique
%Ah !! ouf, oh ca va, quelle page ?  merci
La commande pour transformer le fichier UI est la suivante (via un terminal/console) :
\begin{lstlisting}
pyuic5.py -x fichier_qtdesigner_entree.ui -o fichier_python_sortie.py
\end{lstlisting}

Vous n'aurez jamais besoin de modifier le fichier python généré par pyuic5. En effet, \textbf{ce fichier doit être généré à chaque fois que vous modifiez l'interface depuis QtDesigner}, auquel cas vous ne constaterez aucun changement.

\subsection{Améliorations}

Étant donné que le fait de régénérer le fichier .ui à chaque changment peut être pénible, nous avons rédigé un script python qui transforme tout les fichier QtDesigner présent dans un dossier en fichier python. \newline
Cette section sera abordée plus en détail par la suite.

\subsection{Installation}

La commande pour installer Pyuic5 ets la suivante : \newline

\begin{lstlisting}
pip install pyqt5-tools
\end{lstlisting}
Il est possible d'installer la bibliothèque avec Anaconda : \newline
\begin{lstlisting}
conda install -c anaconda pyqt
\end{lstlisting}


%###########################################################################################################################
%###################### NEW PAGE ###########################################################################################
%###########################################################################################################################


\chapter{Conclusions}

\section{Capacités du programme}

A ce jour, le code fourni est en mesure de gérer un programme de supervision comportant des vérins, moteurs et voyants. \\
L'utilisateur est libre d'ajouter des objets comme bon lui semble. \\

Les tables d'animations sont également opérationnelles et permettent de visualiser à tout instant l'état des variables de l'automate. \\
Il est possible de changer d'automate pour une même interface. \\
Cependant, il nous est encore impossible de distinguer une variable forcé par l'utilisateur ou une variable à un état de repos. \\
Cet objectif sera, dans la mesure du possible, à prendre en compte. \\


\section{Améliorations}

L'un des objectifs serait de récupérer les données des variables sur une base de temps afin de les exploiter dans le futur (traitement puis graphique d'exploitation). Il faudrait donc sauvegarder les valeurs des variables soit dans un simple fichier au format CSV ou XML ou encore dans une base de données (SQL, SQlite...). \\
Il faut être conscient que la dernière méthode est plus à longue à configurer car une base de données n'est pas forcément installé sur les ordinateurs. \\

Le programme que nous avons codé permet d'utiliser le logiciel seulement sur un ordinateur. \\
Une version améliorée serait de faire une interface pour smartphone afin de la rendre plus portable. \\
On pourrait donc prendre le contrôle de l'automate à travers une interface WEB. Cependant, cela peut poser de graves problèmes de sécurité car la communication Interface-Automate ne se fait plus à travers un seul réseau local.


\section{Problèmes rencontrés}

Tout au long de ce semestre, de nombreux problèmes ont été rencontrés, notamment au niveau du langage en lui-même. \\
Nous devions nous baser sur le langage Python, cependant celui-ci a apporté de nombreux problèmes.
\newline
En effet, le premier était tout simplement notre mauvaise maîtrise du langage. Python est un langage simple à aborder mais dans le cadre d'une utilisation plus ambitieuse telle que notre projet, les compétences à posséder deviennent bien particulières et le langage expose des aspects qui sont loin d'être simples si nous n'y avons pas été formé. \newline
Une bonne partie du temps dont nous disposions a donc été employé à la résolution de bugs inhérents à notre mauvaise maîtrise de Python.

\smallSkip

Ensuite, de nombreux choix ont dû être faits. En effet, avec la volonté de créer un module ré-utilisable de haut niveau, il a été nécessaire de faire des compromis entre performance et simplicité d'utilisation.\newline
Au début du projet, nous avions développés notre propre système de signaux/slots utilisant le multi-threading mais apportant des problèmes en terme de compatibilité avec Qt. C'est ceci qui nous a fait re-changer notre programme pour lui faire utiliser des signaux/slots Qt au dernier moment.

\smallSkip

Le manque de possibilités de tester notre programme a aussi été un frein au développement de notre projet. Nous avons testé celui-ci avec Unity-PRO extrêmement tardivement.\newline
Puisque nous programmions sous Linux et sans moyen d'accéder facilement à un système Windows.\newline
De plus, nous n'avions pas non plus accès à la salle d'automatismes, ce qui nous aurait offert un moyen simple de tester notre programme en conditions réelles et de perdre moins de temps en fin de semestre à corriger tous les bugs existants dans le programme (n'ayant pas pu être testé avant). Cela nous aurait aussi permis de nous rendre compte des différents problèmes conceptuels qui ont dû être résolu très rapidement\footnote{Même si ces problème auraient dû être envisagés avant la rédaction du programme}.

\smallSkip

Enfin, nous avons fait un mauvais usage de la bibliothèque PyModbus. En effet, celle-ci semble proposer de nombreuses fonctionnalités pour simplifier la gestion et la robustesse de la communication. Malheureusement, l'absence de documentation complète de la bibliothèque ne nous a pas permis de les découvrir avant la fin du projet. Sans compter que les documents qui nous ont été fournis comme base au début du semestre n'exploitaient pas ces fonctionnalités.


\end{document}
