
%#############################################################
%MAIN file for  project
%#############################################################
\documentclass[12pt]{report}
%############################################################
%###### Package 'name' 
%###### This package contains ...
%###### Author  : Nicolas LE GUERROUE
%###### Contact : nicolasleguerroue@gmail.com
%############################################################
\RequirePackage{lmodern}
\RequirePackage[T1]{fontenc}
\RequirePackage[utf8]{inputenc}     %UTF-8 encoding
\RequirePackage[many]{tcolorbox}
%1 : txt color
%2 : bg color 
%3 : text
\newtcbox{\badge}[3][red]{
  on line, 
  arc=2pt,
  colback=#3,
  colframe=#3,
  fontupper=\color{#2},
  boxrule=1pt, 
  boxsep=0pt,
  left=6pt,
  right=6pt,
  top=2pt,
  bottom=2pt
}

%################################################################%############################################################
%###### Package 'Chapter_1' 
%###### This package contains first model of chapter
%###### Author  : Nicolas LE GUERROUE
%###### Contact : nicolasleguerroue@gmail.com
%############################################################
% \RequirePackage[explicit]{titlesec} 


% \titleformat{\chapter}
%   {\gdef\chapterlabel{}
%    \normalfont\sffamily\Huge\bfseries\scshape}
%   {\gdef\chapterlabel{\thechapter)\ }}{0pt}
%   {\begin{tikzpicture}[remember picture,overlay]
%     \node[yshift=-2cm] at (current page.north west)
%       {\begin{tikzpicture}[remember picture, overlay]
%         \draw[fill=MediumBlue] (0,0) rectangle
%           (\paperwidth,2cm);
%         \node[anchor=east,xshift=.9\paperwidth,rectangle,
%               rounded corners=20pt,inner sep=11pt,
%               fill=white]
%               {\color{black}#1};%\chapterlabel
%           \node[anchor=west,yshift=1cm,xshift=2cm,inner sep=11pt,
%               fill=MediumBlue]
%               {\color{white}{\large Synthèse Latex}};%\chapterlabel
%        \end{tikzpicture}
%       };
%    \end{tikzpicture}
%   }
%   \titlespacing*{\chapter}{0pt}{50pt}{-60pt}%############################################################
%###### Package 'Color' 
%###### This package contains some colors
%###### Author  : Nicolas LE GUERROUE
%###### Contact : nicolasleguerroue@gmail.com
%############################################################
\RequirePackage{color}              %colors
%################################################################
\definecolor{pathColor}{rgb}{0.858, 0.188, 0.478}
\definecolor{fileColor}{rgb}{0.858, 0.188, 0.478}
\definecolor{water}{rgb}{0.858, 0.188, 0.478}
\definecolor{LightBlue}{RGB}{66, 163, 251}
\definecolor{DarkBlue}{RGB}{36, 100, 176}
\definecolor{LightGray}{gray}{.94}
\definecolor{DarkGray}{gray}{.172}
\definecolor{Orange}{RGB}{229, 133, 3}
\definecolor{MediumBlue}{RGB}{38, 119, 193}


\newcommand{\colors}[2]{
{\color{#1}{#2}}
}%############################################################
%###### Package 'Check' 
%###### This package contains soem tools to check data
%###### Author  : Nicolas LE GUERROUE
%###### Contact : nicolasleguerroue@gmail.com
%############################################################
%################################################################

  \makeatletter%
  \@ifpackageloaded{babel}
    {\typeout{>>> Utils : Babel package is loaded}}%
    {\typeout{>>> Utils : Babel package is not loaded}}%
  \makeatother%%############################################################
%###### Package 'Electronic' 
%###### This package contains some tools to generate electornic circuits
%###### Author  : Nicolas LE GUERROUE
%###### Contact : nicolasleguerroue@gmail.com
%############################################################

\RequirePackage{tikz-timing}

\RequirePackage{graphics} %include figures
\RequirePackage{graphicx} %include figures
\RequirePackage{pgf,tikz}
\RequirePackage{circuitikz}
\usetikzlibrary{babel}  %allow to use tikz library with babel

\RequirePackage{ifthen}    %use if then 
 

%############################ Settings ##############################
\tikzset{%
    timing/table/axis/.style={->,>=latex},
    timing/table/axis ticks/.style={},   
}

%Direction of some device such as resistor, led...
%1.5 is the minimum length of device according my runs
\newcommand{\Up}{0,1.5}
\newcommand{\Down}{0,-1.5}
\newcommand{\Right}{1.5,0}
\newcommand{\Left}{-1.5,0}


%###### Length of components
\newcommand{\bipolesLength}[1]{#1cm}     %begin default size
%Length update
\newcommand{\setBipolesLength}[1]{
    \renewcommand{\bipolesLength}{#1}
    \ctikzset{bipoles/length=\bipolesLength cm}
}%End \setBipolesLength

%############ Mirrors and inverting
\newcommand{\Mirror}{}
\newcommand{\Invert}{}

%Update
\newcommand{\setMirror}[1]{
    \renewcommand{\Mirror}{,mirror}
}%End \setMirror

\newcommand{\setNoMirror}[1]{
    \renewcommand{\Mirror}{}
}%End \setMirror

\newcommand{\setInvert}[1]{
    \renewcommand{\Invert}{,Invert}
}%End \setInvert

\newcommand{\setNoInvert}[1]{
    \renewcommand{\Invert}{}
}%End \setInvert


%############## Rotate ###########
\newcommand{\rotate}{0}

%Update
\newcommand{\setRotate}[1]{
    \renewcommand{\rotate}{#1}
}%End \setRotate


%####################### Colors
%default colors of border colors and background colors
\newcommand{\deviceBorderColor}{black}
\newcommand{\deviceBackgroundColor}{white}

%Update
\newcommand{\setDeviceBorderColor}[1]{
    \renewcommand{\deviceBorderColor}{#1}
    \renewcommand{\deviceBackgroundColor}{white} %reset bg color
}%End \setDeviceBorderColor

\newcommand{\setDeviceBackgroundColor}[1]{
    \renewcommand{\deviceBorderColor}{black} %reset bg color
    \renewcommand{\deviceBackgroundColor}{#1}
}%\setDeviceBackgroundColor

%Reset
\newcommand{\resetColors}{
    \renewcommand{\deviceBorderColor}{black} %reset border color
    \renewcommand{\deviceBackgroundColor}{white} %reset bg color
}
%####################################################################
%############## draw device #########################################

%rotation
%color
%anchor

%\ifthenelse{\equal{#1}{0}}{A.}{no A.}
%Init

\begin{comment}
    @begin
    @command \addLogicGate
    @des 
    Cette commande permet de dessiner une porte logique à double entrée. Pour dessinder une porte inverseuse, utiliser la commande \addNotGate
    @sed
    @input Coordonnées de la porte en (x,y) sans parenthèse
    @input Référence de la porte pour s'accrocher aux entrées et sorties
    @input Type de la porte [nand, nor, or, and, or, xor]
    @input Label de sortie (laisser vide si absence de label souhaité)
    @input Label de l'entrée 1 (laisser vide si absence de label souhaité)
    @input Label de l'entrée 2 (laisser vide si absence de label souhaité)
    @input Nom de la porte [NOR1, AND1...]
    @begin_example 
    \addLogicGate{5,5}{logicgate}{nand}{S}{A}{B}{L1}
    @end_example
    @end
    \end{comment}

\newcommand{\addLogicGate}[7] {
    %\Colors
    \raiseMessage{Adding logic gate device [type=#3]}
    \ifthenelse{\equal{\deviceBorderColor}  {black}}
    {\draw (#1)         node (#2) [rotate=\rotate,xshift=0cm,fill=\deviceBackgroundColor,#3 port] {#7}}%if equal to black
    {\draw (#1)         node (#2) [rotate=\rotate,xshift=0cm,color=\deviceBorderColor,#3 port] {#7}}

    (#2.out)  node      [anchor=south west, yshift=-0.3cm] {#4}
    (#2.in 1) node (A1)     [anchor=east,xshift=0cm,yshift=+0.3cm]       {#5}
    (#2.in 2) node (B1)     [anchor=east,xshift=0cm,yshift=+0.3cm]       {#6};
}


\newenvironment{schema}[1]
{
    \begin{center}
        \makeatletter
        \def\@captype{figure}
        \makeatother
        \newcommand{\TitleSchema}{#1}%use var to print title 
        %\shorthandoff{:;!?} %Compulsory if frenchb package is used (from babel)
        \raiseMessage{Creating new schema ['#1']}
        \begin{tikzpicture}
            %\setGraphic %command to display with frenchb babel
    }
    { 
        \end{tikzpicture}
   % \caption{\TitleSchema}
    \end{center}
}


\newenvironment{numeric}[1]
{
\begin{center}
    \makeatletter
    \def\@captype{figure}
    \makeatother
    \newcommand{\TitleNumeric}{#1}%use var to print title 
    \raiseMessage{Creating new chronogram ['#1']}
\begin{tikztimingtable}
}
{
\end{tikztimingtable}%
\caption{\TitleNumeric}
\end{center}
}


%cood
%name device
%type (npn, pnp)
%B
%C
%E
\newcommand{\addTransistor}[6] {

    \raiseMessage{Adding transistor device [type=#3]}
    \ifthenelse{\equal{\deviceBorderColor}  {black}}
    {\draw (#1)         node (#2) [xshift=0cm,fill=\deviceBackgroundColor,#3] {}}%if equal to black
    {\draw (#1)         node (#2) [xshift=0cm,color=\deviceBorderColor,#3] {}}

    (#2.B)  node      [anchor=south west, xshift=0cm, yshift=0cm] {#4} 
    (#2.C) node (A1)     [anchor=north,xshift=0.3cm,yshift=+0.1cm]       {#5}
    (#2.E) node (B1)     [anchor=south,xshift=0.3cm,yshift=0.1cm]       {#6};
}
%node 1
%node 2
%wire type
\newcommand{\addWire}[3] {
    \draw (#1) #3 (#2);
}%end addWire

\newcommand{\orthogonalWireA}{-|}
\newcommand{\orthogonalWireB}{|-}
\newcommand{\directWire}{--}

%coord
%label
%value
\newcommand{\addNode}[3] {
    \node (#2) at (#1) {#3};
}%end addWire


%posiition (x,y)
%color
%width in pt
\newcommand{\addPoint}[3] {
    \filldraw [#2] (#1) circle (#3pt);
}%end addWire

%posiition (x,y)
%name
%value
\newcommand{\addPower}[3] {
    \raiseMessage{Adding power device [name=#2, value=#3]}
    \draw (#1) node (#2) [vcc] {#3};
}%end addPower

%posiition (x,y)
%name
%value
\newcommand{\addGround}[3] {
    \draw (#1) node (#2) [ground] {#3};
}%end addGround

%posiition depart(x,y)
%orientation de départ
%position arrivée
\newcommand{\addResistor}[4] {
    \raiseMessage{Adding resistor device}
    \draw (#1) to[R,l=$R$] +(#2) #4 (#3);
}%end addResistor

%posiition depart(x,y)
%orientation de départ
%position arrivée
%type de liaison
%nom de la led
\newcommand{\addLed}[5] {
    \raiseMessage{Adding LED device [name=#5]}
    \ifthenelse{\equal{\deviceBorderColor}  {black}}
    {\draw (#1) to[leD,l_=#5,fill=\deviceBackgroundColor] +(#2) #4 (#3);}
    {\draw (#1) to[leD,l_=#5,color=\deviceBorderColor] +(#2) #4 (#3);}

    % {\draw (#1)         node (#2) [xshift=0cm,fill=\deviceBackgroundColor,#3] {}}%if equal to black
    % {\draw (#1)         node (#2) [xshift=0cm,color=\deviceBorderColor,#3] {}}

    %\draw (#1) to[leD,l_=#5] +(#2) #4 (#3);
}%end addResistor%############################################################
%###### Package 'Font' 
%###### This package contains some tools to set fonts
%###### Author  : Nicolas LE GUERROUE
%###### Contact : nicolasleguerroue@gmail.com
%############################################################
\RequirePackage{fontawesome}

   %############################################################
%###### Package 'Glossaries' 
%###### This package contains some tools to set glossaries
%###### Author  : Nicolas LE GUERROUE
%###### Contact : nicolasleguerroue@gmail.com
%############################################################
\RequirePackage[xindy, acronym, nomain, toc]{glossaries}
%################################################################
\makeglossaries

\renewcommand{\glossary}[1]{%
\gls{#1}
}%############################################################
%###### Package 'Graphics' 
%###### This package contains some tools to create graphics 2D or 3D
%###### Author  : Nicolas LE GUERROUE
%###### Contact : nicolasleguerroue@gmail.com
%############################################################
%\ProvidesPackage{Utils}[2013/01/13 Utils Package]
%############################################################
\RequirePackage[T1]{fontenc}
\RequirePackage[utf8]{inputenc}     %UTF-8 encoding
\RequirePackage{csvsimple} 
\RequirePackage{tikz,pgfplots,pgf}  
\RequirePackage{version}            %use commented code

\RequirePackage{graphics} %include figures
\RequirePackage{graphicx} %include figures
\RequirePackage{caption}
\RequirePackage{subcaption} %Add
\RequirePackage{version}            %use commented code
\pgfplotsset{compat=1.7}
%###### Checking if babel is loaded
\makeatletter
\@ifpackageloaded{babel}
{% if the package was loaded
\newcommand{\setGraphic}{\shorthandoff{:;!?}} %Compulsory if frenchb package is used (from babel)
\frenchbsetup{StandardLists=true} %to include if using \RequirePackage[french]{babel} -> rounded list
}
{%else:
\newcommand{\setGraphic}{} %Compulsory if frenchb package is used (from babel)
}
\makeatother
%############################################################
%### WARNING : USE \shorthandoff{:;!?} before \begin{tikzpicture} 
%### environment
%############################################################
\begin{comment}
@begin
@env graphics
@des 
Cet environnement permet de tracer des courbes en 2D à partir de points, d'équations ou de fichier CSV
@sed
@input Largeur du graphique ]0;1] 
@input Hauteur du graphique ]0;1+] 
@input Valeur minimale en abscisse 
@input Valeur maximale en abscisse
@input Valeur minimale en ordonnée 
@input Valeur maximale en ordonnée
@input Légende de l'axe des abscisse
@input Légende de l'axe des ordonnées
@input Titre du graphique
@begin_example 
\begin{graphics}{0.8}{0.5}{0}{100}{-10}{10}{ve}{vs}{Titre}
\addPointsFromCSV{raw.csv}
\end{graphics}
@end_example
@end
\end{comment}

\newenvironment{graphicFigure}[9]
{
    \raiseMessage{Creating new graphic figure [title='#9']}
    \begin{center}
        \makeatletter
        \def\@captype{figure}
        \makeatother

        \newcommand{\TitleGraphic}{#9}%use var to print title 
        \begin{tikzpicture}
        \setGraphic %command to display with frenchb babel
    %\shorthandoff{:;!?} %Compulsory if frenchb package is used (from babel)
    \begin{axis}[width=#1\linewidth,height=#2\linewidth,xmin=#3,xmax=#4,  ymin=#5, ymax=#6, scale only axis,xlabel=#7,ylabel=#8] %grid=both
    }
    { 
    \end{axis}
        \end{tikzpicture}
        \captionof{figure}{\TitleGraphic}
    \end{center}
}


\newenvironment{graphic}[9]
{
    \raiseMessage{Creating new graphic [title='#9']}
    \newcommand{\TitleGraphic}{#9}%use var to print title 
        \begin{tikzpicture}
        \setGraphic %command to display with frenchb babel
    \begin{axis}[width=#1\linewidth,height=#2\linewidth,xmin=#3,xmax=#4,  ymin=#5, ymax=#6, scale only axis,xlabel=#7,ylabel=#8, title=#9] %grid=both
    }
    { 
    \end{axis}
        \end{tikzpicture}

}
    
%############################################################
\begin{comment}
@begin
@command addPoints
@des 
Cette commande permet de tracer une courbe en passant une liste de coordonnées de points : (x1,y1)(x2,y2)(...)
Il faut que cette commande soit utilisées dans l'environnement \textbf{graphics}.
@sed
@input Couleur de la courbe
@input Liste des coordonnées des points. Chaque point est entre parenthèse et les coordonnées sont séparées par une virgule
@begin_example 
\addPoints{red}{(0,0)(5,0)(5,5)(10,5)(10,0)}
@end_example
@end
\end{comment}   

\newcommand{\addPoints}[2]{
    \addplot+[thick,mark=none, color=#1] coordinates{#2};
}

%############################################################
\begin{comment}
@begin
@command addTrace
@des 
Cette commande permet de tracer une courbe en passant une équation en fonction de x
Il faut que cette commande soit utilisées dans l'environnement \textbf{graphics}.
@sed
@input Couleur de la courbe
@input Début du domaine de définition
@input Fin du domaine de définition
@input Équation à tracer
@begin_example 
\addTrace{blue}{4}{8}{8x^2}
@end_example
@end
\end{comment}

\newcommand{\addTrace}[4]{
    \addplot [#1, domain=#2:#3, samples=400] {#4};
}

%############################################################
\begin{comment}
@begin
@command addPointsFromCSV
@des 
Cette commande permet de tracer une courbe en passant un fichier de données CSV (ou format TXT)
Il faut que cette commande soit utilisées dans l'environnement \textbf{graphics}.
@sed
@input Couleur de la courbe
@input délimitateur des données (comma=virgule, semicolon=point-vigule)
@input Fichier des données
@begin_example 
\addPointsFromCSV{blue}{semicolon}{raw.csv}
@end_example
@end
\end{comment}

\newcommand{\addPointsFromCSV}[3]{
    \IfFileExists{#3}{
    \addplot+[thick, mark=none, color=#1] table[mark=none,col sep=#2] {#3};
    \raiseMessage{File '#3' loaded !}
    }
    {\raiseError{[import failed]'#3' \stop}
    }
}

%############################################################
\begin{comment}
@begin
@command addLegend
@des 
Cette commande permet d'ajouter une légende au graphique
Il faut que cette commande soit utilisées dans l'environnement \textbf{graphics}.
@sed
@input Légendes de chaque courbe séparées par une virgule
@begin_example 
\addLegend{sig1, sig2, sig3}
@end_example
@end
\end{comment}

\newcommand{\addLegend}[1]{
    \legend{#1}
}

%############################################################%############################################################
%###### Package 'Layout' 
%###### This package contains some tools to set page layout or text
%###### Author  : Nicolas LE GUERROUE
%###### Contact : nicolasleguerroue@gmail.com
%############################################################
\RequirePackage{lmodern}
\RequirePackage[T1]{fontenc}
\RequirePackage[utf8]{inputenc}     %UTF-8 encoding
\RequirePackage{graphicx}           %Images
\RequirePackage{caption}            %légende
\RequirePackage{textcomp}           %special characters
\RequirePackage{fancyhdr}           %headers & footers
\RequirePackage{lastpage}           %page counter

\RequirePackage{float}              %image floating
\RequirePackage{wrapfig}
\RequirePackage{subcaption}         %Subcaption

\RequirePackage{geometry}

%############################################################

\begin{comment}
@begin
@command \setHeader
@des 
Cette commande permet de créer une page de garde minimaliste
@sed
@input Titre du document
@input Auteur(s) - Les retours à la ligne se font en utilisant la commande \\
@input La date
@begin_example 
\setHeader{Titre}{Auteur 1 \\ Auteur 2}{XX/XX/XXXX}
@end_example
@end
\end{comment}

\newcommand{\setHeader}[3]{
\title{#1}
\author{#2}
\date{#3}
\maketitle
}






%###############################################################
\begin{comment}
@begin
@command \partImg
@des 
Cette commande permet de créer une page de partie avec une image
@sed
@input Titre de la partie
@input Source de l'image \\
@input Ratio
@begin_example 
\partImg{Partie}{Images/file.png}{0.2}
@end_example
@end
\end{comment}
\newcommand{\partImg}[3]{
    \part[#1]{#1 \\\vspace*{2cm} \makebox{\centering \includegraphics[width=#3\textwidth]{#2}}}
}


%###############################################################
\begin{comment}
@begin
@command \setHeaderImage
@des 
Cette commande permet de créer une page de garde avec une image centrée, un titre, sous titre en plus
@sed
@input Titre du document
@input Auteur(s) - Les retours à la ligne se font en utilisant la commande \\
@input La date
@begin_example 
\setHeaderImage{Titre}{Auteur 1 \\ Auteur 2}{XX/XX/XXXX}
@end_example
@end
\end{comment}

\newcommand{\setHeaderImage}[6]{
\begin{titlepage}
  \begin{sffamily}
  \begin{center}
    \includegraphics[scale=#2]{#1} \sn \sn
    \hfill
%\HRule \\[0.4cm]
\begin{center}
    {\Huge \textbf{#3}} \sn
    \textbf{#4}\sn \sn
\end{center}
\sn \sn
 #5 \sn
   \vfill
   {\large #6}
  \end{center}
  \end{sffamily}
\end{titlepage}
}


%### Définition du style de page 'classic' si report

\newcommand{\addPresentation}[6]{
\fancypagestyle{classic}{
    \rhead{#3}  
    \lhead{#1}
    \chead{#2}
    \rfoot{#6}  %Page courante / Nombre de page
    \cfoot{#5}
    \lfoot{#4}
}

\@ifclassloaded{report}{
\makeatletter
\renewcommand\chapter{\if@openright\cleardoublepage\else\clearpage\fi
                      \thispagestyle{classic} %Thème 'classic'
                      \global\@topnum\z@
                      \@afterindentfalse
                      \secdef\@chapter\@schapter}
\makeatother
}%End renew chapter

\@ifclassloaded{book}{
\makeatletter
\renewcommand\chapter{\if@openright\cleardoublepage\else\clearpage\fi
                      \thispagestyle{classic} %Thème 'classic'
                      \global\@topnum\z@
                      \@afterindentfalse
                      \secdef\@chapter\@schapter}
\makeatother
}%End renew chapter

\pagestyle{classic}
}


\newcommand{\setRightHeader}[1]{\rhead{#1}}
\newcommand{\setCenterHeader}[1]{\chead{#1}}
\newcommand{\setLeftHeader}[1]{\lhead{#1}}

\newcommand{\setRightFooter}[1]{\rfoot{#1}}
\newcommand{\setCenterFooter}[1]{\cfoot{#1}}
\newcommand{\setLeftFooter}[1]{\lfoot{#1}}



\newcommand{\setHeaderLine}[1]{ 
\renewcommand{\headrulewidth}{#1pt} 
}
\newcommand{\setFooterLine}[1]{ 
\renewcommand{\footrulewidth}{#1pt} 
}
  
\newcommand{\currentChapter}{\leftmark}

%Raccourcis
\@ifclassloaded{report)}{

\newcommand{\setAliasChapter}[1]{
\makeatletter
\renewcommand{\@chapapp}{#1}   %Le mot 'Chapitre' est remplacé par 'Section'
\makeatother
}
}%End if
{
  \newcommand{\setAliasChapter}[1]{
}
}

\@ifclassloaded{book)}{
\newcommand{\currentChapter}{\leftmark}
\newcommand{\setAliasChapter}[1]{
\makeatletter
\renewcommand{\@chapapp}{#1}   %Le mot 'Chapitre' est remplacé par 'Section'
\makeatother
}
}%End if


\newcommand{\currentPage}{\thepage/\pageref{LastPage}}%############################################################
%###### Package 'name' 
%###### This package contains ...
%###### Author  : Nicolas LE GUERROUE
%###### Contact : nicolasleguerroue@gmail.com
%############################################################
\RequirePackage{lmodern}
\RequirePackage[T1]{fontenc}
\RequirePackage[utf8]{inputenc}     %UTF-8 encoding
\RequirePackage{graphicx}           %Images
\RequirePackage{caption}            %légende
\RequirePackage{float}              %image floating
\RequirePackage{wrapfig}
\RequirePackage{subcaption}         %Subcaption

%################################################################


\newcounter{imgCounter}  %Number of img

%##################
\begin{comment}
    @begin
    @command \img
    @des 
    Cette commande permet de mettre une image centrée avec un titre et une taille
    @sed
    @input Source et nom de l'image
    @input Titre de l'image
    @input Taille de l'image ]0;1+]
    @begin_example 
    \img{Image.png}{Titre de l'image}{0.5}
    @end_example
    @end
\end{comment}
    
\newcommand{\img}[3]{\IfFileExists{#1}{\begin{figure}[H]\centering\ \includegraphics[scale=#3]{#1}\caption{#2}\end{figure} \addtocounter{imgCounter}{1} \raiseMessage{Image '#1' [size=#3,id \arabic{imgCounter}] loaded !}  }{\raiseWarning{Image '#1' no loaded}}}


\begin{comment}
      @begin
      @command \imgr
      @des 
      Cette commande permet de mettre une image centrée avec un titre et une taille et un angle
      @sed
      @input Source et nom de l'image
      @input Titre de l'image
      @input Taille de l'image ]0;1+]
      @input angle en degres
      @begin_example 
      \img{Image.png}{Titre de l'image}{0.5}{90}
      @end_example
      @end
\end{comment}
\newcommand{\imgr}[4]{\IfFileExists{#1}{\begin{figure}[H]\centering\ \includegraphics[scale=#3,angle=#4]{#1}\caption{#2}\end{figure} \addtocounter{imgCounter}{1} \raiseMessage{Image '#1' [size=#3,id \arabic{imgCounter},angle=#4] loaded !} }{\raiseWarning{Image '#1' no loaded}}}

    %################################################################
    \begin{comment}
    @begin
    @command \imgf
    @des 
    Cette commande permet de mettre une image centrée sans titre et une taille de manière flottante avec le contenu
    @sed
    @input Source et nom de l'image
    @input Titre de l'image
    @input Taille de l'image ]0;1+] par rapport au texte
    @input Taille de l'image par rapport à l'espace qui lui est reservé
    
    @begin_example 
    \img{Image.png}{Titre de l'image}{0.5}{0.5}
    @end_example
    @end
    \end{comment}
    



    %################################################################%############################################################
%###### Package 'Index' 
%###### This package contains some tools to set index
%###### Author  : Nicolas LE GUERROUE
%###### Contact : nicolasleguerroue@gmail.com
%############################################################
%################################################################
\RequirePackage{makeidx}            %make index

\makeindex%############################################################
%###### Package 'name' 
%###### This package contains ...
%###### Author  : Nicolas LE GUERROUE
%###### Contact : nicolasleguerroue@gmail.com
%############################################################
\RequirePackage{enumitem}
\RequirePackage{pifont}

%################################################################

\newcommand{\setTriangleList}{\renewcommand{\labelitemi}{$\blacktriangleright$}}
\newcommand{\setCircleList}{\renewcommand{\labelitemi}{$\circ$}}
\newcommand{\setBulletList}{\renewcommand{\labelitemi}{$\bullet$}}
\newcommand{\setDiamondList}{\renewcommand{\labelitemi}{$\diamond$}}

\newcommand{\Triangle}{$\blacktriangleright$}
\newcommand{\Circle}{$\circ$}
\newcommand{\Bullet}{$\bullet$}
%\newcommand{\Diamond}{$\diamond$}


\newenvironment{items}[2]
{      
        \begin{itemize}[font=\color{#1}, label=#2]  
    }
    { 
        \end{itemize}
}%############################################################
%###### Package 'name' 
%###### This package contains ...
%###### Author  : Nicolas LE GUERROUE
%###### Contact : nicolasleguerroue@gmail.com
%############################################################
\tcbuselibrary{listings,breakable, skins}
%\RequirePackage{amsmath}

%################################################################

\newtcbox{\lorange}[1]{enhanced, nobeforeafter,tcbox raise base,boxrule=0.4pt,top=0mm,bottom=0mm,
  right=0mm,left=4mm,arc=1pt,boxsep=2pt,before upper={\vphantom{dlg}},
  colframe=orange!50!black,coltext=orange!25!black,colback=orange!10!white,
  overlay={\begin{tcbclipinterior}\fill[orange!75!white] (frame.south west)
    rectangle node[text=white,font=\sffamily\bfseries\tiny,rotate=90] {#1} ([xshift=4mm]frame.north west);\end{tcbclipinterior}}}

\newtcbox{\lred}[1]{enhanced, nobeforeafter,tcbox raise base,boxrule=0.4pt,top=0mm,bottom=0mm,
  right=0mm,left=4mm,arc=1pt,boxsep=2pt,before upper={\vphantom{dlg}},
  colframe=red!50!black,coltext=red!25!black,colback=red!10!white,
  overlay={\begin{tcbclipinterior}\fill[red!75!white] (frame.south west)
    rectangle node[text=white,font=\sffamily\bfseries\tiny,rotate=90] {#1} ([xshift=4mm]frame.north west);\end{tcbclipinterior}}}

\newtcbox{\lgreen}[1]{enhanced, nobeforeafter,tcbox raise base,boxrule=0.4pt,top=0mm,bottom=0mm,
  right=0mm,left=4mm,arc=1pt,boxsep=2pt,before upper={\vphantom{dlg}},
  colframe=green!50!black,coltext=green!25!black,colback=green!10!white,
  overlay={\begin{tcbclipinterior}\fill[green!75!white] (frame.south west)
    rectangle node[text=white,font=\sffamily\bfseries\tiny,rotate=90] {#1} ([xshift=4mm]frame.north west);\end{tcbclipinterior}}}

\newtcbox{\lmagenta}[1]{enhanced, nobeforeafter,tcbox raise base,boxrule=0.4pt,top=0mm,bottom=0mm,
  right=0mm,left=4mm,arc=1pt,boxsep=2pt,before upper={\vphantom{dlg}},
  colframe=magenta!50!black,coltext=magenta!25!black,colback=magenta!10!white,
  overlay={\begin{tcbclipinterior}\fill[magenta!75!white] (frame.south west)
    rectangle node[text=white,font=\sffamily\bfseries\tiny,rotate=90] {#1} ([xshift=4mm]frame.north west);\end{tcbclipinterior}}}
  
\newtcbox{\lpurple}[1]{enhanced, nobeforeafter,tcbox raise base,boxrule=0.4pt,top=0mm,bottom=0mm,
  right=0mm,left=4mm,arc=1pt,boxsep=2pt,before upper={\vphantom{dlg}},
  colframe=purple!50!black,coltext=purple!25!black,colback=purple!10!white,
  overlay={\begin{tcbclipinterior}\fill[purple!75!white] (frame.south west)
    rectangle node[text=white,font=\sffamily\bfseries\tiny,rotate=90] {#1} ([xshift=4mm]frame.north west);\end{tcbclipinterior}}}
    
\newtcbox{\lblue}[1]{enhanced, nobeforeafter,tcbox raise base,boxrule=0.4pt,top=0mm,bottom=0mm,
  right=0mm,left=4mm,arc=1pt,boxsep=2pt,before upper={\vphantom{dlg}},
  colframe=blue!50!black,coltext=blue!25!black,colback=blue!10!white,
  overlay={\begin{tcbclipinterior}\fill[blue!75!white] (frame.south west)
    rectangle node[text=white,font=\sffamily\bfseries\tiny,rotate=90] {#1} ([xshift=4mm]frame.north west);\end{tcbclipinterior}}}
   
   
\newtcbox{\lcyan}[1]{enhanced, nobeforeafter,tcbox raise base,boxrule=0.4pt,top=0mm,bottom=0mm,
  right=0mm,left=4mm,arc=1pt,boxsep=2pt,before upper={\vphantom{dlg}},
  colframe=cyan!50!black,coltext=cyan!25!black,colback=cyan!10!white,
  overlay={\begin{tcbclipinterior}\fill[cyan!75!white] (frame.south west)
    rectangle node[text=white,font=\sffamily\bfseries\tiny,rotate=90] {#1} ([xshift=4mm]frame.north west);\end{tcbclipinterior}}}

\newtcbox{\lbrown}[1]{enhanced, nobeforeafter,tcbox raise base,boxrule=0.4pt,top=0mm,bottom=0mm,
    right=0mm,left=4mm,arc=1pt,boxsep=2pt,before upper={\vphantom{dlg}},
    colframe=brown!50!black,coltext=brown!25!black,colback=brown!10!white,
    overlay={\begin{tcbclipinterior}\fill[brown!75!white] (frame.south west)
      rectangle node[text=white,font=\sffamily\bfseries\tiny,rotate=90] {#1} ([xshift=4mm]frame.north west);\end{tcbclipinterior}}}
   
\newtcbox{\lyellow}[1]{enhanced, nobeforeafter,tcbox raise base,boxrule=0.4pt,top=0mm,bottom=0mm,
  right=0mm,left=4mm,arc=1pt,boxsep=2pt,before upper={\vphantom{dlg}},
  colframe=yellow!50!black,coltext=yellow!25!black,colback=yellow!10!white,
  overlay={\begin{tcbclipinterior}\fill[yellow!75!white] (frame.south west)
    rectangle node[text=white,font=\sffamily\bfseries\tiny,rotate=90] {#1} ([xshift=4mm]frame.north west);\end{tcbclipinterior}}}
        
\newtcbox{\lblack}[1]{enhanced, nobeforeafter,tcbox raise base,boxrule=0.4pt,top=0mm,bottom=0mm,
    right=0mm,left=4mm,arc=1pt,boxsep=2pt,before upper={\vphantom{dlg}},
    colframe=black!50!black,coltext=black!25!black,colback=black!10!white,
    overlay={\begin{tcbclipinterior}\fill[black!75!white] (frame.south west)
      rectangle node[text=white,font=\sffamily\bfseries\tiny,rotate=90] {#1} ([xshift=4mm]frame.north west);\end{tcbclipinterior}}}%############################################################
%###### Package 'Layout' 
%###### This package contains some tools to set page layout or text
%###### Author  : Nicolas LE GUERROUE
%###### Contact : nicolasleguerroue@gmail.com
%############################################################
\RequirePackage{lmodern}
\RequirePackage[T1]{fontenc}
\RequirePackage[utf8]{inputenc}     %UTF-8 encoding

\RequirePackage{xcolor}             %define new colors
\RequirePackage{xparse}
\RequirePackage{amssymb,amsthm}     %math


%################################################################
\begin{comment}
@begin
@command \bold
@des 
Cette commande permet de mettre le texte en gras
@sed
@input Texte à mettre en gras
@begin_example 
\bold{texte en gras}
@end_example
@end
\end{comment}

\renewcommand{\bold}[1]{\textbf{#1}}

%################################################################
\begin{comment}
@begin
@command \italic
@des 
Cette commande permet de mettre le texte en italique
@sed
@input Texte à mettre en italique
@begin_example 
\italic{texte en italique}
@end_example
@end
\end{comment}

\newcommand{\italic}[1]{\textit{#1}}

%################################################################
\begin{comment}
@begin
@command \ib
@des 
Cette commande permet de mettre le texte en italique et en gras
@sed
@input Texte à mettre en italique et gras
@begin_example 
\ib{texte en italique et gras}
@end_example
@end
\end{comment}

\newcommand{\ib}[1]{\textit{\textbf{#1}}}

%################################################################
\begin{comment}
@begin
@command \bi
@des 
Cette commande permet de mettre le texte en italique et en gras
@sed
@input Texte à mettre en italique et gras
@begin_example 
\bi{texte en italique et gras}
@end_example
@end
\end{comment}

\newcommand{\bi}[1]{\textit{\textbf{#1}}}


\begin{comment}
@begin
@command \n
@des 
Cette commande permet de faire un saut de ligne
@sed
@begin_example 
Text \n Next text
@end_example
@end
\end{comment}

\newcommand{\n}{\\}

%################################################################
\begin{comment}
@begin
@command \sn
@des 
Cette commande permet de faire un espace vertical
@sed
@begin_example 
Text \sn Next text
@end_example
@end
\end{comment}

\newcommand{\sn}{\vskip 0.5cm}

%################################################################

%#######################################%############################################################
%###### Package 'name' 
%###### This package contains tools to use links, url and other
%###### Author  : Nicolas LE GUERROUE
%###### Contact : nicolasleguerroue@gmail.com
%############################################################
\RequirePackage{hyperref}           %Url

%################################################################
\begin{comment}
@begin
@command \setParameters
@des 
Cette commande permet de définir les propriétés du document PDF (auteur, titre...)
@sed
@input Titre du PDF
@input Auteur(s)
@input Sujet du fichier PDF (courte phrase)
@input Créateur du fichier PDF
@input Mots-clés (liste)
@input Couleurs des liens
@input Couleurs des citations dans la bibliographie
@input Couleurs des liens de fichier
@input Couleurs des liens externe
@begin_example 
\setParameters {Tutoriel Latex} {Nicolas LE GUERROUE} {Tutoriel Latex pour la mise en place des outils} {\author}{Latex}{green}{blue}{blue}
@end_example
@end
\end{comment}

\newcommand{\setParameters}[8]{
   \typeout{>>> Utils - [MData] : title='#1}
   \typeout{>>> Utils - [MData] : author(s)='#2'}
   \typeout{>>> Utils - [MData] : subject='#3'}
   \typeout{>>> Utils - [MData] : creator='#4'}
   \typeout{>>> Utils - [MData] : keywords='#5'}
   \typeout{>>> Utils - [MData] : link colors='#6'}
   \typeout{>>> Utils - [MData] : bib links colors='#7'}
   \typeout{>>> Utils - [MData] : link file colors='#8'}
\hypersetup{
    bookmarks=true,         % show bookmarks bar?
    unicode=true,          % non-Latin characters in Acrobat’s bookmarks
    pdftoolbar=true,        % show Acrobat’s toolbar?
    pdfmenubar=true,        % show Acrobat’s menu?
    pdffitwindow=false,     % window fit to page when opened
    pdfstartview={1},    % fits the width of the page to the window
    pdftitle={#1},    % title
    pdfauthor={#2},     % author
    pdfsubject={#3},   % subject of the document
    pdfcreator={#4},   % creator of the document
    pdfproducer={#4}, % producer of the document
    pdfkeywords={#5}, % list of keywords
    pdfnewwindow=true,      % links in new PDF window
    colorlinks=true,       % false: boxed links; true: colored links
    linkcolor=black,          % color of internal links (change box color with linkbordercolor)
    citecolor=#6,        % color of links to bibliography
    filecolor=#7,         % color of file links
    urlcolor=#8        % color of external links
} }%############################################################
%###### Package 'Maths' 
%###### This package contains some tools to set mathematic tools
%###### Author  : Nicolas LE GUERROUE
%###### Contact : nicolasleguerroue@gmail.com
%############################################################
%Vecteur à 3 composantes  
\newcommand{\evec}[3]{\left (\begin{array}{ccc} #1 \\ #2 \\ #3\end{array} \right )}
%Matrice à 3 composantes
\newcommand{\emat}[3]{\left [\begin{array}{ccc} #1 \\ #2 \\ #3\end{array} \right ]}
% \emat{a & b & c}{d & e & f}{g & h & i}%############################################################
%###### Package 'MessageBox' 
%###### This package contains some tools to set messageBox
%###### Author  : Nicolas LE GUERROUE
%###### Contact : nicolasleguerroue@gmail.com
%############################################################
\RequirePackage[many]{tcolorbox}
\RequirePackage{color}              %colors
\RequirePackage{geometry}
%################################################################
\newcounter{messageBoxCounter}  %Number of msgBox
%###############################################################

\newcommand{\messageBox}[5]{
% 1: title
% 2: color frame
% 3: color bg
% 4: content
% 5: titlecolor
%\messageBox{Remarque}{green}{white}{Erreur 0xff58}{black}
%\messageBox{Installation}{gray}{white}{DE}{black}
%\messageBox{Avertissement}{orange}{white}{dede}{black}
%\messageBox{Erreur}{red}{white}{dede}{black}
\addtocounter{messageBoxCounter}{1}
\raiseMessage{MessageBox '#1' [id \arabic{messageBoxCounter}] created !}
 \begin{tcolorbox}[title=#1,
colframe=#2!80,
colback=#3!10,
coltitle=#5!100,  
]
#4
\end{tcolorbox}
}%############################################################
%###### Package 'Glossaries' 
%###### This package contains some tools to set glossaries
%###### Author  : Nicolas LE GUERROUE
%###### Contact : nicolasleguerroue@gmail.com
%############################################################
\RequirePackage{nomencl}  %nomenclature
%################################################################


\makenomenclature

%##### Convention [Nomenclature]
  \renewcommand{\nomgroup}[1]{%
  \item[\bfseries
  \ifthenelse{\equal{#1}{P}}{Constantes physiques}{%
  \ifthenelse{\equal{#1}{O}}{Autres symboles}{%
  \ifthenelse{\equal{#1}{N}}{Nombres spéciaux}{ 
  \ifthenelse{\equal{#1}{A}}{Amplificateurs Opérationnels}{ 
  \ifthenelse{\equal{#1}{M}}{Mécanique}{ 
  \ifthenelse{\equal{#1}{E}}{Électronique}{}}}}}}%
  ]}


  % Add unit on convention [nomenclature]
%----------------------------------------------
\newcommand{\addUnit}[1]{%
\renewcommand{\nomentryend}{\hspace*{\fill}#1}}
%----------------------------------------------%############################################################
%###### Package 'name' 
%###### This package contains ...
%###### Author  : Nicolas LE GUERROUE
%###### Contact : nicolasleguerroue@gmail.com
%############################################################

\RequirePackage{csvsimple} 
\RequirePackage{tikz,pgfplots,pgf}  
\RequirePackage{version}            %use commented code

\RequirePackage{graphics} %include figures
\RequirePackage{graphicx} %include figures
\RequirePackage{caption}
\RequirePackage{subcaption} %Add
\RequirePackage{version}            %use commented code
\pgfplotsset{compat=1.7}

%############################################################
%################################################################
%1 : color
%2 : diameter
\newcommand{\ball}[2]{
    \tikz\path[shading=ball,
    ball color=#1] circle (#2mm);
}





%############################################################
\begin{comment}
    @begin
    @command plot
    @des 
    Cette commande permet de tracer des surfaces 3D
    @sed
    @input Iitre du graphique
    @input Equation à deux variables (x et y)
    @begin_example 
    \plot{Titre}{x*x+y*5}
    @end_example
    @end
    \end{comment}   
        
    \newcommand{\plot}[2]{
        \raiseMessage{Creating new plot [title='#1']}
    \begin{tikzpicture}
    \setGraphic %command to display with frenchb babel
     \begin{axis}[title={#1}, xlabel=x, ylabel=y]
     \addplot3[surf,domain=0:360,samples=50]
     {#2};
     \end{axis}
     \end{tikzpicture}
     }
        %############################################################
%###### Package 'Titles' 
%###### This package contains some tools to set title color
%###### Author  : Nicolas LE GUERROUE
%###### Contact : nicolasleguerroue@gmail.com
%############################################################
\RequirePackage{pdfpages} 
%################################################################

%\includepdf[⟨options⟩]{⟨fichier⟩}\RequirePackage{color}              %colors
\RequirePackage{listings}           %new env such as Bash, Python
\RequirePackage{xcolor}             %define new colors
\RequirePackage{graphicx}             %define new colors
\RequirePackage{xparse}
\RequirePackage{amssymb,amsthm}     %math
\RequirePackage{multirow}
\RequirePackage{tabularx}           %use for csv file [table]
\RequirePackage{textcomp}           %special characters

%###############################################################
%### Couleurs programmation
\definecolor{green}{rgb}{0,0.6,0}
\definecolor{gray}{rgb}{0.5,0.5,0.5}
\definecolor{purple}{rgb}{0.58,0,0.82}
\definecolor{bg}{rgb}{0.98,0.98,0.98}
%###############################################################
%### définition du style 'Python' [begin{Python}]
\lstdefinestyle{Python}{
    backgroundcolor=\color{bg},   
    commentstyle=\color{green},
    keywordstyle=\color{blue},
    numberstyle=\tiny\color{gray},
    stringstyle=\color{purple},
    basicstyle=\ttfamily\footnotesize,
    breakatwhitespace=false,         
    breaklines=true,                 
    captionpos=b,                    
    keepspaces=true,                 
    numbers=none,   %left, right, none 
    numbersep=5pt,                  
    showspaces=false,                
    showstringspaces=false,
    showtabs=false,  
    tabsize=2
}
\lstset{style=Python}
\lstset{literate=
  {á}{{\'a}}1 {é}{{\'e}}1 {í}{{\'i}}1 {ó}{{\'o}}1 {ú}{{\'u}}1
  {Á}{{\'A}}1 {É}{{\'E}}1 {Í}{{\'I}}1 {Ó}{{\'O}}1 {Ú}{{\'U}}1
  {à}{{\`a}}1 {è}{{\`e}}1 {ì}{{\`i}}1 {ò}{{\`o}}1 {ù}{{\`u}}1
  {À}{{\`A}}1 {È}{{\'E}}1 {Ì}{{\`I}}1 {Ò}{{\`O}}1 {Ù}{{\`U}}1
  {ä}{{\"a}}1 {ë}{{\"e}}1 {ï}{{\"i}}1 {ö}{{\"o}}1 {ü}{{\"u}}1
  {Ä}{{\"A}}1 {Ë}{{\"E}}1 {Ï}{{\"I}}1 {Ö}{{\"O}}1 {Ü}{{\"U}}1
  {â}{{\^a}}1 {ê}{{\^e}}1 {î}{{\^i}}1 {ô}{{\^o}}1 {û}{{\^u}}1
  {Â}{{\^A}}1 {Ê}{{\^E}}1 {Î}{{\^I}}1 {Ô}{{\^O}}1 {Û}{{\^U}}1
  {Ã}{{\~A}}1 {ã}{{\~a}}1 {Õ}{{\~O}}1 {õ}{{\~o}}1 {’}{{'}}1
  {œ}{{\oe}}1 {Œ}{{\OE}}1 {æ}{{\ae}}1 {Æ}{{\AE}}1 {ß}{{\ss}}1
  {ű}{{\H{u}}}1 {Ű}{{\H{U}}}1 {ő}{{\H{o}}}1 {Ő}{{\H{O}}}1
  {ç}{{\c c}}1 {Ç}{{\c C}}1 {ø}{{\o}}1 {å}{{\r a}}1 {Å}{{\r A}}1
  {€}{{\euro}}1 {£}{{\pounds}}1 {«}{{\guillemotleft}}1
  {»}{{\guillemotright}}1 {ñ}{{\~n}}1 {Ñ}{{\~N}}1 {¿}{{?`}}1
}
\lstnewenvironment{Python}[1]{\lstset{language=Python, title=#1}}{}
%###############################################################
%###############################################################
%### définition du style 'Cpp' [begin{Cpp}]
\lstdefinestyle{Cpp}{
    basicstyle=\ttfamily,
    columns=fullflexible,
    keepspaces=true,
    upquote=true,
    showstringspaces=false,
    commentstyle=\color{olive},
    keywordstyle=\color{blue},
    identifierstyle=\color{violet},
    stringstyle=\color{purple},
    numbers=none,
    language=c,
}
\lstnewenvironment{Cpp}[1]{\lstset{style=Cpp, title=#1}}{}

%###############################################################
%### définition du style 'Bash' [begin{Bash}]
\lstdefinestyle{Bash}{
    basicstyle=\ttfamily,
    columns=fullflexible,
    keepspaces=true,
    upquote=true,
    frame=trBL, %frame=trBL, shadowbox, trrb
    %frameround=fttt,
    numbers=none,
    showstringspaces=false,
    commentstyle=\color{olive},
    keywordstyle=\color{blue},
    identifierstyle=\color{black},
    stringstyle=\color{gray},
    language=bash,
    morekeywords={sudo,apt-get,install, autoremove, update, upgrade}
}
\lstnewenvironment{Bash}[1]{\lstset{style=Bash, title=#1}}{}%############################################################
%###### Package 'Theorems' 
%###### This package contains some tools to set theorems
%###### Author  : Nicolas LE GUERROUE
%###### Contact : nicolasleguerroue@gmail.com
%############################################################
\RequirePackage{amssymb,amsthm}     %math
%################################################################
%###############################################################
\newtheorem{question}{Question}
\newtheorem{reponse}{$>>>$}
\newtheorem{propriete}{Propriété}
\newtheorem{proposition}{Proposition}
\newtheorem{remarque}{Remarque}
\newtheorem{exemple}{Exemple}
\newtheorem{definition}{Definition}%############################################################
%###### Package 'Titles' 
%###### This package contains some tools to set title color
%###### Author  : Nicolas LE GUERROUE
%###### Contact : nicolasleguerroue@gmail.com
%############################################################
\RequirePackage[explicit]{titlesec} 
%################################################################

% \@ifclassloaded{report}{
%     \titleformat{\chapter}[display] {\fontsize{17pt}{12pt}\selectfont \bfseries}{\textcolor{blue} {\chaptertitlename\ \thechapter: #1}}{20pt}{\Huge}
% }

\titleformat{\section}[display] {\fontsize{17pt}{12pt}\selectfont \bfseries}{\textcolor{DarkBlue} {#1}}{20pt}{\Huge}
\titleformat{\subsection}[display] {\fontsize{15pt}{12pt}\selectfont \bfseries}{\textcolor{orange} {#1}}{20pt}{\large}
\titlespacing*{\section}{0pt}{20pt}{-30pt}
\titlespacing*{\subsection}{0pt}{20pt}{-20pt}
%#######################################################
%### Settings file
%### Project : Rapport_etonnement
%#######################################################
\documentclass[12pt]{report}
\usepackage[frenchb]{babel} %Select langage
\usepackage{Utils/Utils} %Load Utils library
%#############################################################
%Geometry
%#############################################################
\geometry{hmargin=3cm,vmargin=3cm}
%#############################################################
%#############################################################
%Rename chapter name
%#############################################################
\setAliasChapter{Section}
%#############################################################
%#############################################################
%If you want to add presentation, modify the next bloc
%The firt line is about the header :
%{left content}{center content}{right content}
%The second line is about the footer :
%{left content}{center content}{right content}
%####
%to get the current chapter name, use \currentChapter command as content
%#############################################################
\addPresentation
{} {} {}
{Club de Robotique et d'Electronique\\ Programmable de Ploemeur} {} {\currentPage}
%#############################################################
%#############################################################
%Change the width of footer line and header line
%To delete it, set value to 0
%#############################################################
\setHeaderLine{0}
\setFooterLine{0.2}
%#############################################################
%#############################################################
%URL and data settings
%#############################################################
\setParameters{}{Nicolas LE GUERROUE}{}{}{}{blue}{blue}{green}
%#############################################################



\titleformat{\chapter}
  {\gdef\chapterlabel{}
   \normalfont\sffamily\Huge\bfseries\scshape}
  {\gdef\chapterlabel{\thechapter)\ }}{0pt}
  {\begin{tikzpicture}[remember picture,overlay]
    \node[yshift=-2cm] at (current page.north west)
      {\begin{tikzpicture}[remember picture, overlay]
        \draw[fill=MediumBlue] (0,0) rectangle
          (\paperwidth,2cm);
        \node[anchor=east,xshift=.9\paperwidth,rectangle,
              rounded corners=20pt,inner sep=11pt,
              fill=white]
              {\color{black}#1};%\chapterlabel
          \node[anchor=west,yshift=1cm,xshift=2cm,inner sep=11pt,
              fill=MediumBlue]
              {\color{white}{\large MySensors}};%\chapterlabel
       \end{tikzpicture}
      };
   \end{tikzpicture}
  }
  \titlespacing*{\chapter}{0pt}{50pt}{-60pt}
\begin{document}
\setHeader{A}{A}{A}
\tableofcontents
\setcounter{page}{2}
\part{Préparation du projet}

\chapter{Introduction}

\section{Présentation}

Ce document a pour but d'expliquer la mise en place d'une passerelle et d'une sonde MySensors.

\subsection{Organigramme}

\begin{figure}[h]
  \centering
\begin{tikzpicture}[node distance={30mm}, thick, main/.style = {draw, circle}]
  \node[main] (1) [color=green] {$Sonde$}; 
  \node[main] (2) [above right of=1, color=cyan] {$Passerelle$}; 
  \node[main] (3) [left of=1] {$DHT$}; 
  \node[main] (4) [below of=3] {$Capteur_2$}; 
  \node[main] (5) [below of=1] {$Capteur_3$}; 
  \node[main] (6) [right of=2, color=blue] {$Domoticz$};
  
  \draw[->] (3) -- (1);
  \draw[->] (4) -- (1);
  \draw[->] (5) -- (1);

  \draw[->] (1) -- (2);
  \draw[->] (2) -- (6);
\end{tikzpicture} 
\caption{Les différents composants du projet}
\end{figure}

  \subsection{Principe}

  Les capteurs vont être analysés par la sonde MySensors.\\
  Cette dernière enverra à distance les informations vers la passerelle qui se chargera d'envoyer les informations au serveur Domoticz via une liaison USB.\\


  Une sonde représente un endroit physique, un lieu de mesure. \\
  Si vous souhaitez par la suite faire d'autres relevés dans un endroit différent, il suffira d'ajouter une sonde et de garder la passerelle.\\
  Chaque sonde est caractérisée par un identifiant de noeud (NODE\_ID) et chaque capteur possède un identifiant enfant sur la sonde qui lui est rattachée (CHILD\_ID)
  \begin{figure}[h]
    \centering
  \begin{tikzpicture}[node distance={30mm}, thick, main/.style = {draw, circle}]
    \node[main] (1) {$Passerelle$}; 
    \node[main] (2) [below left of=1] {$Sonde_1$}; 
    \node[main] (3) [left of=1] {$Sonde_2$}; 
    \node[main] (4) [above left of=1] {$Sonde_3$}; 

    \node[main] (5) [right of=1] {$Domoticz$};
    
    \draw[->] (2) -- (1);
    \draw[->] (3) -- (1);
    \draw[->] (4) -- (1);
  
    \draw[->] (1) -- (5);
    \end{tikzpicture} 
    \caption{Une extension possible}
  \end{figure}





  \section{Structure du projet}

  Nous vous invitons à garder la structure suivante pour le projet : 
  
  Dans un dossier \lblue{DIR}{Domoticz\_Crepp}, placez deux dossiers appelés \lblue{DIR}{Sonde\_MySensors} et \lblue{DIR}{Passerelle\_MySensors}
  
  Ces deux derniers dossiers contiendront respectivement le programme de la sonde et de la passerelle.
  
  
  \begin{figure}[h!]
    \centering
  \usetikzlibrary{trees}
  
  \tikzstyle{every node}=[draw=black,thick,anchor=west]
  \tikzstyle{selected}=[draw=blue,fill=blue!30]
  \tikzstyle{optional}=[dashed,fill=gray!50]
  \begin{tikzpicture}[%
    grow via three points={one child at (0.5,-0.7) and
    two children at (0.5,-0.7) and (0.5,-1.4)},
    edge from parent path={(\tikzparentnode.south) |- (\tikzchildnode.west)}]
    \node {Domoticz\_MySensors\_Crepp}
      child { node {Sonde\_MySensors}
        child { node [selected] {Sonde\_MySensors.ino}}
      }
      child [missing] {}				
      child { node {Passerelle\_MySensors}
        child { node [selected] {Passerelle\_MySensors.ino}}
      };
  \end{tikzpicture}
  
  \tikzstyle{every node}=[]
  \tikzstyle{selected}=[]
  \tikzstyle{optional}=[]
  \caption{Arborescence du projet}
  \end{figure}\chapter{Programmation}

\section{Programmation de la Pro-Mini}

Programmer une carte Arduino pro-mini avec une carte Arduino évite d'acheter un module FTDI.\\
De plus, la carte Arduino Uno pourra être réutilisée pour d'autres projets.\\

L'objectif est de programmer la carte Pro-mini sur la sonde MySensors.

\section{Liste du matériel}

\begin{items}{blue}{\Triangle}
    \item 5 câbles Dupont mâles-femelles\footnote{Il est possible de faire des liaisons mâles-femelles avec des cables mâles-mâles et femelles-femelles}
    \imgr{\rootImages/wire.jpg}{Les câbles de connexion}{0.3}{0}

    \item Une carte Arduino Pro-Mini
    \imgr{\rootImages/promini.jpeg}{La carte Arduino Pro-mini}{0.5}{0}

    \item Une carte Arduino Uno
    \imgr{\rootImages/a1.jpg}{La carte Arduino Uno}{0.05}{0}

    \end{items}

  \section{Branchements}

  \messageBox{Attention}{red}{white}{La carte Arduino Pro-Mini doit être alimentée en \bold{3.3V} et non en 5V !\\\bold{La carte Arduino Pro-Mini ne doit pas être placée sur son support de sonde}}{white}
 

  Voici les connexions à faire pour programmer la Pro-Mini : 

  Le mot \lorange{PIN}{XXXX\_UNO} représente une broche de la carte Arduino UNO et \\ 
  \lgreen{PIN}{XXXX\_PRO-MINI} représente une broche de la carte Arduino Pro-Mini.\\
  \bold{XXXX} est l'indication du nom de la broche.

  \begin{items}{orange}{\Triangle}
    \item \lorange{PIN}{RESET\_UNO} vers \lgreen{PIN}{RST\_PRO-MINI}
    \item \lorange{PIN}{+3.3V\_UNO} vers \lgreen{PIN}{VCC\_PRO-MINI}
    \item \lorange{PIN}{GND\_UNO} vers \lgreen{PIN}{GND\_PRO-MINI}
    \item \lorange{PIN}{RX\_UNO} vers \lgreen{PIN}{RX\_PRO-MINI}
    \item \lorange{PIN}{TX\_UNO} vers \lgreen{PIN}{TX\_PRO-MINI}
  \end{items}

  \imgr{\rootImages/pinout.png}{Les broches du Pro-Mini}{0.3}{0}

  \messageBox{Remarque}{orange}{white}{Ici, la liaison série (\bold{RX} et \bold{TX}) n'est pas croisée, le \bold{RX} de la carte Uno  va sur le \bold{TX} de la Pro-Mini, idem pour le TX}{white}
 




  Vous pouvez ouvrir le programme Arduino que vous désirez charger\footnote{Vous pouvez charger le programme de clignotement de la led pour l'exemple} sur la carte Arduino Pro-Mini.
  Voici un programme minimal pour faire clignoter la Led du pro-mini. \\
  Ce programme est disponible en allant, dans le logiciel Arduino, dans la section \bold{Fichiers > Exemples > basics > Blink}\\
  
  \begin{Cpp}{Programme d'exemple Blink}
  void setup() {
  // initialize digital pin LED_BUILTIN as an output.
  pinMode(LED_BUILTIN, OUTPUT);
}

// the loop function runs over and over again forever
void loop() {
  digitalWrite(LED_BUILTIN, HIGH);   // turn the LED on (HIGH is the voltage level)
  delay(1000);                       // wait for a second
  digitalWrite(LED_BUILTIN, LOW);    // turn the LED off by making the voltage LOW
  delay(1000);                       // wait for a second
}
  \end{Cpp}
  
Une fois le programme ouvert, voici les étapes pour compiler le programme.

  \section{Téléversement}

  \begin{items}{blue}{\Triangle}
    \item 1) Sélectionner la carte Arduino Pro-mini dans \bold{Outils > Types de carte}
    \imgr{\rootImages/type.png}{Type de carte}{0.3}{0}

    \item 2) Sélectionne le processeur dans \bold{Outils > Processeur}
    \imgr{\rootImages/processeur.png}{Type de processeur}{0.4}{0}
  \end{items}

  \messageBox{Avertissement}{orange}{white}{N'oubliez pas de séléctionner le port de communciation de l'Arduino}{black}
 
  Il ne vous reste plus qu'à cliquer sur le bouton de téléversement du programme.\\
  La led de la carte Pro-Mini devrait clignoter.



  \messageBox{Information}{green}{white}{Ici, nous avons chargé un programme de test, par la suite, il conviendra de charger le programme \bold{Sonde\_MySensors.ino}.\\Cette étape de chargement de programme sera nécéssaire à chaque modification du code de la sonde.}{black}
 

  \section{Programmation de la Nano}

  La carte Nano étant reliée à l'ordinateur par un câble USb, sa programmation sera plus aisée. 
  On alimente la carte via l'ordinateur, on séléctionne le type de carte (\bold{Type de carte > Arduino Nano}),\\
   le type de processeur (\bold{Outils > Processeur > Old bootloader}), le port puis on téléverse le programme désiré.\\


%PB : 

%Canal différent pour chaque personne

%vérifier branchement RF24 - > carré masse, voisin +3.3V -> ohmetre
\part{Montage des circuits imprimés}

\chapter{La passerelle}

\section{Rappels}

La passerelle reçoit sur sa barrette connecteurs femelles un \bold{Arduino Nano} et est reliée par une nappe à 8 conducteurs à un \bold{module transmetteur NRF24}.

L'Arduino Nano est relié au Raspberry Pi de votre plateforme Domoticz par un câble USB (transmission d'infos et alimentation 5v) et alimentera la passerelle en 5v.

Le module transmetteur NRF24 assure la liaison radio avec les sondes. 

\section{Liste du matériel de la passerelle}

\begin{itemize}
    \item 3 Led
    \item 3 résistances de $270~\Omega$
    \item 2 condensateurs électrolytiques ( $100~\mu F, 16~V$ – cylindriques noirs)
    \item 2 condensateurs céramiques monolithiques ( $100nF, 50~V$ – couleur jaune foncé )
    \item 1 régulateur 3.3v HT7533-1
    \item 1 module transmetteur NRF24
    \item 1 bouton poussoir
    \item 1 circuit imprimé
    \item 1 barrette connecteurs femelle ({\color{red}{déjà montée sur le circuit imprimé}})
    \item 1 nappe 8 conducteurs ({\color{red}{dont l'un porte un liseré rouge}} ),
    \item 2 jumpers
\end{itemize}

\section{Placement des composants}

\subsection{Vue de dessus du circuit}
\img{\rootImages/pin.png}{Circuit imprimé vu de dessus, coté composants}{0.5}\label{TEST}
\subsection{Étapes}
\begin{enumerate}
\item Souder la barrette déjà en place


\messageBox{Remarque}{red}{white}{Il faudra faire très attention au placement de la carte Arduino nano par la suite : La broche D13 de la Nano doit être impérativement dans le trou le plus avancé (coté nappe de fils).\\Un décalage de 1 trou lors du placement de la carte Nano dans sa barrette pourra endommager la carte Nano et le module RF24 !}{black}
 
\item Souder les 8 brins de la nappe en respectant les consignes suivantes : 

\begin{items}{blue}{\Triangle}
\item Séparer les 8 conducteurs sur 2 cm environ.	
\item Chaque conducteur étant multibrins, s'assurer qu'ils sont bien torsadés puis étamer.
\item Respecter l'ordre de soudage -> Fil au liseré rouge en 1 puis conducteurs suivants en 2,3 	   etc.
\end{items}
\img{\rootImages/nappe.png}{Emplacement de la nappe}{0.5}

\messageBox{Remarque}{red}{white}{Ne pas se tromper sur la soudure de la nappe\\
 En gardant la vue de la Figure 3.1, le câble rouge de la nappe est en bas à gauche, le n°2 est en bas à droite, etc...}{black}
 
\end{enumerate}

Puis dans l'ordre que vous souhaitez : 
\begin{itemize}
    \item Led rouge en D1 – Led jaune en D2 – Led verte en D3 ({\color{red}{sont polarisées}}, patte longue au + ,  patte courte, méplat sur la led au - ),
    \item Les 3 résistances sont à souder en R1, R2 et R3
    \item Les 2 condensateurs électrolytiques sont à souder en C1 et C2 ({\color{red}{sont polarisées}}, le corps du condo est noir avec une bande grisée, la patte de ce côté est le - ),
    \item Les 2 condensateurs céramiques sont à souder en C3 et C4
    \item Le bouton poussoir en SW1
    \item Le jumper D13 (Arduino Nano) en J12
    \item Le jumper 5v (Arduino Nano) en J10
    \item Le régulateur 3.3v HT7533 est à souder suivant les conseils de la section suivante
\end{itemize}

\section{Mise en place du régulateur de tension}

Comme vous l'avez surement remarqué, l'implantation précédente correspond à un régulateur LE33 et non à un HT7533

Si vous voulez utiliser un HT7533, il faut adapter le brochage du HT7533 au circuit.

\begin{itemize}
    \item Vin : ENtrée 5v
    \item GND : la masse
    \item Vout : sortie 3.3V
\end{itemize}

\img{\rootImages/to-92.png}{Broches du HT7533}{0.5}

\subsection{Adaptation des broches du HT7533 au schéma de la passerelle}

Il faut que les broches \bold{GND, Vin et Vout} rentrent dans les mêmes broches que celle du schéma de la passerelle. Même si les broches ne sont pas dans le même ordre, c'est assez simple à faire en tordant les broches du HT7533 avec une petite pince plate.\\

Sur le HT7533, sans que les broches se touchent, \bold{on tord Vin vers l'avant, Vout vers l'arrière et on ramène GND au milieu.}

\img{\rootImages/ht.png}{Insertion du HT7533}{0.35}

\section{Rendus}

\img{\rootImages/side.png}{Vue de coté}{0.4}
\img{\rootImages/passerelle_emetteur.png}{Vue de la passerelle et de l'émetteur}{0.4}
\img{\rootImages/dessus.png}{Vue de dessus}{0.4}

\chapter{La sonde}


\section{Rappels}


La sonde reçoit sur sa barrette connecteurs femelles un \bold{Arduino Pro-Mini} et est reliée par une nappe à 8 conducteurs à un \bold{module transmetteur NRF24}.
Le module transmetteur NRF24 assure la liaison radio avec la passerelle. 

\section{Liste du matériel de la sonde}

\begin{itemize}
\item 1 pcb
\item 2 barrettes mâle-femelle 12 plots		
\item 1 barrette mâle-femelle 3 plots	
\item 1 barrette mâle-femelle 2 plots	
\item 3 barrettes mâle-mâle 5 plots		
\item 1 barrette mâle-mâle 4 plots		
\item 1 barrette mâle-mâle 3 plots		
\item 3 barrettes mâle-mâle 2 plots		
\item 1 barrette mâle-mâle 1 plot		
\item 2 condensateurs céramiques 100 nF
\item 2 condensateurs chimiques    10 µF
\item 1 résistance 1/4 w  330 K$\Omega$
\item 1 résistance 1/4 w 1M$\Omega$
\item 1 régulateur HT7533
\item 1 longueur de fil d'acier pour shunts
\item 1 longueur de fil isolé pour shunt
\item 1 carte Pro mini
\item 1 émetteur NRF24
\item 1 nappe 8 conducteurs
\end{itemize}

\section{Placement des composants}

\subsection{Vue de dessus du circuit}
\img{\rootImages/pcb.png}{Circuit imprimé vu de dessus, coté composants}{0.5}\label{TEST}


\subsection{Étapes}
\begin{enumerate}
\item Souder les shunts. Voir schéma général et la photo platine sonde shunts
\item Souder les 8 conducteurs de la nappe en suivant les mêmes instructions que pour le montage de la passerelle
\img{\rootImages/nappe.png}{Emplacement de la nappe}{0.5}
\end{enumerate}
Puis dans l'ordre que vous souhaitez : 
\begin{itemize}
    \item Les 2 condensateurs électrolytiques sont à souder en C1 et C2 ({\color{red}{sont polarisées}}, le corps du condensateur est noir avec une bande grisée, la patte de ce côté est le - ),
    \item Les 2 condensateurs céramiques sont à souder en C3 et C4
    \item Le régulateur 3.3v HT7533\\
    \bold{A la différence de son implantation sur la passerelle MySensor, ici il ne faut pas croiser les pattes, implantez le module tel quel en respectant son positionnement sur le pcb grâce au méplat}

    \item Les 2 condensateurs céramiques de 100 nF
    \item Les 2 résistances de 330 k$\Omega$ et de 1 M$\Omega$    (Elles ont le même aspect extérieur vérifier à l’ohmmètre avant la pose)
    \item les 2 condensateurs chimiques de 10 µF	(Le moins est du côté grisé sur le corps du condensateur)
    \item Souder les connecteurs mâle-mâle sur la carte Pro mini, ils sont livrés dans la pochette.\\ Attention à ne pas trop chauffer les points de soudure.
    \item Faire de même avec la carte Arduino Nano nécessaire à la passerelle.
\end{itemize}

\section{Rendus}

\img{\rootImages/s4.jpg}{Sonde vue de dessus sans la nappe}{0.25}
\img{\rootImages/s3.jpg}{Sonde vue de dessus avec la nappe}{0.25}
\chapter{Vérification}

\section{Les court-circuit}

\subsection{Le multimètre} 

Le meilleur allié contre les courts circuit est le multimètre.\\
Sur les rangées de barrettes, nous allons regarder la résistance entre deux broches voisines. \\
Si la résistance est infinie (\bold{Un 1 affiché sur l'écran}), il n'y a pas de court-circuit et si elle tend vers 0, il y a un risque.

\subsubsection{Réglage}

On règle le multimètre en mode \bold{Ohmmètre}, c'est à dire avec le fil noir sur \bold{COM}, le rouge sur \bold{$\Omega$} et le curseur réglé sur la résistance la plus élevée de l'appareil.
\imgr{\rootImages/multi.jpg}{un multimètre bien réglé}{0.05}{-90}


On regarde la résistance entre les broches 1 et 2 par exemple pour commencer puis ensuite entre la broche 2 et 3, etc...

\imgr{\rootImages/c.jpg}{Une vérification}{0.08}{0}


\subsection{L'alimentation} 

Le problème le plus grave peut survenir si un court circuit a lieu entre la broche +VCC\footnote{Alimentation positive, ici +3.3V pour la sonde et +5V pour la passerelle}
et la masse.\\
Il convient donc de trouver ces deux broches (VCC et GND) et de regarder la valeur de la résistance entre ces deux broches.
Cette valeur doit être infinie (\bold{1 sur l'afficheur})

\section{Les sondes NRF24}

Les sondes NRF24 viennent s'insérer dans la nappe de fils (8 brins). Il ne faut pas se tromper de sens sous peine de détruire le module NRF24 lors de sa mise sous tension.

Pour cela, il faut que le coté avec les broches 1 et deux du NRF24 (repéré avec le carré blanc sur la broche 1) soit du même coté que le fils rouge de la nappe.

\imgr{\rootImages/nrf24.jpg}{Insertion du NRF24 dans son connecteur}{0.05}{-90}


Une fois que la connexion electrique est exacte, on peut alimenter le montage et vérifier la tension au bornes de la sonde NRF24.
Si une tension inférieure à $3.2V$ ou supérieure à $3.4V$ apparait, coupez l'alimentation et reprenez les vérifications.

Puis procédez de la même manière pour la sonde.


\part{Configuration de Domoticz}

\chapter{La passerelle}

Une fois que la passerelle est fonctionnelle, nous allons configurer Domoticz pour que la plateforme recoive 
les données en provenance de la passerelle.

\section{Ajout de la passerelle}

Tout d'abord, allez dans la section \bold{Configuration > Matériel}

\img{\rootImages/hardware.png}{Emplacement du matériel}{0.5}

Ensuite, saississez les informations suivantes :

\img{\rootImages/add_gateway.png}{Paramétrage de la passerelle}{0.5}

Le port série séléctionné sera celui où est raccordé la passerelle en liaison USB.
Il ne faut pas prendre les noms simplifiés des ports USB (\italic{COM\_XXX}) mais le nom le plus complet.\\
Pour plus de simplicité, veuillez déconnectez tous les autres périphériques du Raspberry-Pi\\.


\section{Affichage des données}

Visualisons les données en provenance de la sonde en allant dans \\
\bold{Configuration > Matériel}\\.

L'ensemble de vos dispositif apparait. En cas de liste trop longue, saisissez \bold{Gateway} dans la barre de recherche.


\img{\rootImages/search.png}{Recherche de la passerelle}{0.5}


\messageBox{Remarque}{orange}{white}{Si le dispositif n'apparait pas immdiatemment, 
c'est normal, Domoticz fait des mesures par défaut toutes les 5 min.... 
Donc la passerelle devrait apparaitre au bout de 10 min environ.}{black}

\img{\rootImages/see.png}{La passerelle est détectée}{0.5}

\messageBox{Précision}{green}{white}{Dans notre programme \bold{Sonde\_MySensors.ino}, 
nous avons défini la valeur du noeud à 1.\\On retrouve bien cette valeur dans la colonne \bold{Idx}}{black}

Pour visualiser les données, il suffit de cliquer sur le bouton \bold{logs}

\img{\rootImages/cursor_log.png}{Affichage des données}{0.5}
%PB : 
\chapter{Questions}

il conviendra de vérifier toutes les asserelles et les sondes, ces dernières devant avoir le branchement suivant : 



\begin{question} 
  Pourquoi ma passerelle n'est pas détectée sur Domoticz ?
\end{question}

\begin{reponse}
 Une erreur fréquence est de sélectionner le mauvais port lors de la configuration de la passerelle dans Domoticz.\\
\end{reponse}


\begin{question} 
  Pourquoi la passerelle allume sa led rouge ?
\end{question}

\begin{reponse}
 La led rouge veut dire que des erreurs de communication sont survenues entre la passerelle et la sonde. Vérifier les branchements de la sonde. Un fonctionnement normal de la sonde est la led alluumée toutes les 6 secondes.
\end{reponse}


%Canal différent pour chaque personne

%vérifier branchement RF24 - > carré masse, voisin +3.3V -> ohmetre

%Faire attention au placement de la nano, un des plots ne sert pas !!!

%résistance entre +3.3V et GND
%résistance entre 2 plots consécutifs

\appendix

\part{Annexes}
\chapter{Installation}
\section{Installation de Domoticz sur Linux}

Veuillez ouvrir un terminal puis saisir les commandes suivantes : 

\begin{Bash}{Installation de domoticz}
sudo apt-get -y install cmake make gcc g++ libssl-dev git libcurl4-openssl-dev libusb-dev python3-dev curl zlib1g-dev zlib1g
\end{Bash}

Puis lancez le script d'installation avec la commande suivante

\begin{Bash}{Installation de domoticz}
  sudo curl -L https://install.domoticz.com | bash
\end{Bash}

Le terminal devrait afficher un contenu similaire : 

\imgr{\rootImages/load.png}{Vérification des bilbiothèques}{0.3}{0}

Ensuite, une interface utilisateur se lance dans le terminal :


\imgr{\rootImages/l1.png}{Présentation de Domoticz}{0.4}{0}

Il faut saisir la touche \lmagenta{KEY}{ENTREE} pour afficher la fenêtre suivante.
Une deuxième fenêtre apparait. Veuillez séléctionner le service HTTP (par défaut) puis \lmagenta{KEY}{ENTREE}

\imgr{\rootImages/l2.png}{Choix du protocole par défaut}{0.4}{0}

Nous allons ensuite choisir le port 8080 pour communiquer sur le réseau (par défaut : 8080)

\imgr{\rootImages/l3.png}{Choix du port}{0.4}{0}

Nous utiliserons le port 443 (HTTPS) pour un protocole plus sécurisé.

\imgr{\rootImages/l4.png}{Protocole HTTPS}{0.4}{0}

Il ne vous reste plus qu'à chosir l'emplacement du logiciel Domoticz.\\
Par défaut, Domoticz le place dans vos documents personnales (/home/nom\_utilisateur)

\imgr{\rootImages/l5.png}{Emplacement des fichiers Domoticz}{0.4}{0}

Il ne vous reste plus qu'à valider l'installation : 

\imgr{\rootImages/l6.png}{Validation de l'installation}{0.4}{0}

\end{document}
%#############################################################
