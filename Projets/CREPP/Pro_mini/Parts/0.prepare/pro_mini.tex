\chapter{Programmation}

\section{Programmation de la Pro-Mini}

Programmer une carte Arduino pro-mini avec une carte Arduino évite d'acheter un module FTDI.\\
De plus, la carte Arduino Uno pourra être réutilisée pour d'autres projets.\\

L'objectif est de programmer la carte Pro-mini sur la sonde MySensors.

\section{Liste du matériel}

\begin{items}{blue}{\Triangle}
    \item 5 câbles Dupont mâles-femelles\footnote{Il est possible de faire des liaisons mâles-femelles avec des cables mâles-mâles et femelles-femelles}
    \imgr{\rootImages/wire.jpg}{Les câbles de connexion}{0.3}{0}

    \item Une carte Arduino Pro-Mini
    \imgr{\rootImages/promini.jpeg}{La carte Arduino Pro-mini}{0.5}{0}

    \item Une carte Arduino Uno
    \imgr{\rootImages/a1.jpg}{La carte Arduino Uno}{0.05}{0}

    \end{items}

  \section{Branchements}

  \messageBox{Attention}{red}{white}{La carte Arduino Pro-Mini doit être alimentée en \bold{3.3V} et non en 5V !\\\bold{La carte Arduino Pro-Mini ne doit pas être placée sur son support de sonde}}{white}
 

  Voici les connexions à faire pour programmer la Pro-Mini : 

  Le mot \lorange{PIN}{XXXX\_UNO} représente une broche de la carte Arduino UNO et \\ 
  \lgreen{PIN}{XXXX\_PRO-MINI} représente une broche de la carte Arduino Pro-Mini.\\
  \bold{XXXX} est l'indication du nom de la broche.

  \begin{items}{orange}{\Triangle}
    \item \lorange{PIN}{REST\_UNO} vers \lgreen{PIN}{RST\_PRO-MINI}
    \item \lorange{PIN}{+3.3V\_UNO} vers \lgreen{PIN}{VCC\_PRO-MINI}
    \item \lorange{PIN}{GND\_UNO} vers \lgreen{PIN}{GND\_PRO-MINI}
    \item \lorange{PIN}{RX\_UNO} vers \lgreen{PIN}{RX\_PRO-MINI}
    \item \lorange{PIN}{TX\_UNO} vers \lgreen{PIN}{TX\_PRO-MINI}
  \end{items}

  \imgr{\rootImages/pinout.png}{Les broches du Pro-Mini}{0.3}{0}

  \messageBox{Remarque}{orange}{white}{Ici, la liaison série (\bold{RX} et \bold{TX}) n'est pas croisée, le \bold{RX} de la carte Uno  va sur le \bold{TX} de la Pro-Mini, idem pour le TX}{white}
 




  Vous pouvez ouvrir le programme Arduino que vous désirez charger\footnote{Vous pouvez charger le programme de clignotement de la led pour l'exemple} sur la carte Arduino Pro-Mini.
  Voici un programme minimal pour faire clignoter la Led du pro-mini. \\
  Ce programme est disponible en allant, dans le logiciel Arduino, dans la section \bold{Fichiers > Exemples > basics > Blink}\\
  
  \begin{Cpp}{Programme d'exemple Blink}
  void setup() {
  // initialize digital pin LED_BUILTIN as an output.
  pinMode(LED_BUILTIN, OUTPUT);
}

// the loop function runs over and over again forever
void loop() {
  digitalWrite(LED_BUILTIN, HIGH);   // turn the LED on (HIGH is the voltage level)
  delay(1000);                       // wait for a second
  digitalWrite(LED_BUILTIN, LOW);    // turn the LED off by making the voltage LOW
  delay(1000);                       // wait for a second
}
  \end{Cpp}
  
Une fois le programme ouvert, voici les étapes pour compiler le programme.

  \section{Téléversement}

  \begin{items}{blue}{\Triangle}
    \item 1) Sélectionner la carte Arduino Pro-mini dans \bold{Outils > Types de carte}
    \imgr{\rootImages/type.png}{Type de carte}{0.3}{0}

    \item 2) Sélectionne le processeur dans \bold{Outils > Processeur}
    \imgr{\rootImages/processeur.png}{Type de processeur}{0.4}{0}
  \end{items}

  \messageBox{Avertissement}{orange}{white}{N'oubliez pas de séléctionner le port de communciation de l'Arduino}{black}
 
  Il ne vous reste plus qu'à cliquer sur le bouton de téléversement du programme.\\
  La led de la carte Pro-Mini devrait clignoter.



  \messageBox{Information}{green}{white}{Ici, nous avons chargé un programme de test, par la suite, il conviendra de charger le programme \bold{Sonde\_MySensors.ino}.\\Cette étape de chargement de programme sera nécéssaire à chaque modification du code de la sonde.}{black}
 

  \section{Programmation de la Nano}

  La carte Nano étant reliée à l'ordinateur par un câble USb, sa programmation sera plus aisée. 
  On alimente la carte via l'ordinateur, on séléctionne le type de carte (\bold{Type de carte > Arduino Nano}),\\
   le type de processeur (\bold{Outils > Processeur > Old bootloader}), le port puis on téléverse le programme désiré.\\


%PB : 

%Canal différent pour chaque personne

%vérifier branchement RF24 - > carré masse, voisin +3.3V -> ohmetre

