\chapter{Vérification}

\section{Les court-circuit}

\subsection{Le multimètre} 

Le meilleur allié contre les courts circuit est le multimètre.\\
Sur les rangées de barrettes, nous allons regarder la résistance entre deux broches voisines. \\
Si la résistance est infinie (\bold{Un 1 affiché sur l'écran}), il n'y a pas de court-circuit et si elle tend vers 0, il y a un risque.

\subsubsection{Réglage}

On règle le multimètre en mode \bold{Ohmmètre}, c'est à dire avec le fil noir sur \bold{COM}, le rouge sur \bold{$\Omega$} et le curseur réglé sur la résistance la plus élevée de l'appareil.
\imgr{\rootImages/multi.jpg}{un multimètre bien réglé}{0.05}{-90}


On regarde la résistance entre les broches 1 et 2 par exemple pour commencer puis ensuite entre la broche 2 et 3, etc...

\imgr{\rootImages/c.jpg}{Une vérification}{0.08}{0}


\subsection{L'alimentation} 

Le problème le plus grave peut survenir si un court circuit a lieu entre la broche +VCC\footnote{Alimentation positive, ici +3.3V pour la sonde et +5V pour la passerelle}
et la masse.\\
Il convient donc de trouver ces deux broches (VCC et GND) et de regarder la valeur de la résistance entre ces deux broches.


\section{Les sondes NRF24}

Les sondes NRF24 viennent s'insérer dans la nappe de fils (8 brins). Il ne faut pas se tromper de sens sous peine de détruire le module NRF24 lors de sa mise sous tension.

Pour cela, il faut que le coté avec les broches 1 et deux du NRF24 (repéré avec le carré blanc sur la broche 1) soit du même coté que le fils rouge de la nappe.

\imgr{\rootImages/nrf24.jpg}{Insertion du NRF24 dans son connecteur}{0.05}{-90}
