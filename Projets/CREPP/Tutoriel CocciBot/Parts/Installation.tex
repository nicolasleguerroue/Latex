\chapter{Installation de \\ l'application}

\section{Installation de MIT App Inventor}

Pour installer l'application que vous avez réalisé, il est préférable d'installer l'application MIT AI2 Companion disponible sur Play Store. \\
Cette application (relativement lègère (50 Mo) va faire le lien entre le site App Inventor et votre téléphone. \\
Une fois que votre application est terminée, voici la façon la plus simple pour l'installer : 


\begin{itemize}
    \item Installer MIT AI2 Companion sur votre téléphone 
     \img{Images/download.png}{Logo de l'application}{0.5}
    \item Lancer l'application MIT AI2 Companion lorsque'elle est installée \bold{sur votre téléphone}
     \img{Images/mit.png}{Rendu de l'application MIT AI2 Companion}{0.4}
    \item Connecter vous au réseau Wifi de votre maison \bold{depuis votre téléphone}
    \item Ouvrez votre application (Cocci-Bot) dans App Inventor (\url{http://appinventor.mit.edu/}) \bold{depuis votre ordinateur}
    \item Rendez-vous dans le menu du haut, sélectionner \bold{Construire} puis \bold{App ( Donnez le code QR pour fichier .apk )}
    \img{Images/build.png}{Emplacement du menu "Construire"}{1}
    Une barre de progression devrait apparaître.
    \img{Images/progressbar.png}{Barre de progression}{0.6}
    Une fois la barre de progression disparue, un QR Code apparaît. \\ {\color{red}Ne surtout pas cliquer sur \bold{ok}}
    \img{Images/qrcode.png}{QR code}{0.6}
    
   \item Il est temps de scanner le code avec l'application MIT AI2 Companion sur votre téléphone ("Scan with QRCode"). Une fois le processus terminé, vous pouvez cliquer sur \bold{ok}, il faudra suivre les consignes sur votre téléphone portable (gestion des autorisations...)
   \item Et voila !
    
\end{itemize}

Nous pouvons passer au code Arduino et au traitement des données.