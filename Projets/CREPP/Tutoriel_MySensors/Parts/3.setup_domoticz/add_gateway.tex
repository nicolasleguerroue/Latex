

\part{Configuration de Domoticz}

\chapter{La passerelle}

Une fois que la passerelle est fonctionnelle, nous allons configurer Domoticz pour que la plateforme recoive 
les données en provenance de la passerelle.

\section{Ajout de la passerelle}

Tout d'abord, allez dans la section \bold{Configuration > Matériel}

\img{\rootImages/hardware.png}{Emplacement du matériel}{0.5}

Ensuite, saississez les informations suivantes :

\img{\rootImages/add_gateway.png}{Paramétrage de la passerelle}{0.5}

Le port série séléctionné sera celui où est raccordé la passerelle en liaison USB.
Il ne faut pas prendre les noms simplifiés des ports USB (\italic{COM\_XXX}) mais le nom le plus complet.\\
Pour plus de simplicité, veuillez déconnectez tous les autres périphériques du Raspberry-Pi\\.


\section{Affichage des données}

Visualisons les données en provenance de la sonde en allant dans \\
\bold{Configuration > Matériel}\\.

L'ensemble de vos dispositif apparait. En cas de liste trop longue, saisissez \bold{Gateway} dans la barre de recherche.


\img{\rootImages/search.png}{Recherche de la passerelle}{0.5}


\messageBox{Remarque}{orange}{white}{Si le dispositif n'apparait pas immdiatemment, 
c'est normal, Domoticz fait des mesures par défaut toutes les 5 min.... 
Donc la passerelle devrait apparaitre au bout de 10 min environ.}{black}

\img{\rootImages/see.png}{La passerelle est détectée}{0.5}

\messageBox{Précision}{green}{white}{Dans notre programme \bold{Sonde\_MySensors.ino}, 
nous avons défini la valeur du noeud à 1.\\On retrouve bien cette valeur dans la colonne \bold{Idx}}{black}

Pour visualiser les données, il suffit de cliquer sur le bouton \bold{logs}

\img{\rootImages/cursor_log.png}{Affichage des données}{0.5}


