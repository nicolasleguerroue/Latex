Le rapport d’étonnement1 est simple à mettre en œuvre, mais aussi très puissant dans sa
capacité à développer une culture de la curiosité sur le long terme. Il permet de conserver la mémoire
de faits et de prendre du recul sur ce que l’on vit, de distinguer le principe de plaisir du principe de
réalité.
Concrètement, le rapport d’étonnement, c’est la transcription sur papier de vos observations,
tant avec des mots qu’avec des dessins. Il porte sur ce « qui vous a étonné », « surpris », « ému » durant
votre intersemestre.
Cahier des charges :
- Le rapport est constitué de 5000 caractères minimum
- Le rapport est explicite : il doit pouvoir être lu et compris par des lecteurs étrangers à l’école
et à votre cursus.
- Le rapport est construit et suit un ordre chronologique ou thématique
- Le rapport atteste d’une dimension critique suffisante : outre quelques passages narratifs,
sont consignées des observations, des analyses de ce que vous avez vécu, de ce qui vous a
semblé utile-inutile/ plu-déplu : attention il est très inspirant de distinguer les deux.
Conseils
- Maintenir en éveil la faculté d’étonnement
- Etre curieux, collecter des informations (Pendant les ateliers mais aussi dans les déplacements,
les à-côtés)
- Pas de « procrastination » : N’attendez pas les derniers jours pour rédiger, relevez régulièrement
des faits sur un calepin à la manière d’un ethnographe.
Sources d’inspiration :
- Carnets de terrain IS2
- Carnets de voyages (Site de la Bibliothèque Nationale de France)
- Rapport d’étonnements entreprise
- « ESTONNER » : Surprendre par quelque chose d’inopiné […]
- « CURIOSITE » : vient de cura, qui veut dire la cure, comme dans cure ou curatif. Donc le curieux
est celui, ou celle, qui prend soin. « EMERVEILLER » dérive du préfixe « e » et de « merveille »,
mot issu du latin « mirabilia » , « les choses étonnantes, admirables ».
DOCUMENT A DEPOSER IMPERATIVEMENT LE 1 FEVRIER A LA SCOLARITE
\newpage



\newcommand{\pd}{Petits Débrouillards }

\section{La Fresque du Climat}

Cette activité a été très intéressante car du début de la séance à la fin, les émotions ressenties étaient totalement différentes.
Je suis arrivé dans la salle en me disant : "Je vais apprendre des choses sur le climat" et en sortant de la salle, j'ai pris
pleinement conscience de notre impact sur le climat." Le fait d'avoir des exemples de consommation de $CO_2$ m'a permis de me rendre 
compte de ma consommation de $CO_2$


\section{La Fresque du Climat}

Cette activité a été très intéressante car du début de la séance à la fin, les émotions ressenties étaient totalement différentes.
Je suis arrivé dans la salle en me disant : "Je vais apprendre des choses sur le climat" et en sortant de la salle, j'ai pris
pleinement conscience de notre impact sur le climat." Le fait d'avoir des exemples de consommation de $CO_2$ m'a permis de me rendre 
compte de ma consommation de $CO_2$


\section{Les \pd}

Le temps laissé entre le premier contact et l'expérience est extrement court. 
De plus, l'enseignante à recu le kit des \pd la veille.
