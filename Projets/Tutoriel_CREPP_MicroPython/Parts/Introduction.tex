\chapter{Introduction}     

\section{Présentation}

Ce document a pour but de configurer un ESP8266-12E (NodeMCU) afin que ce dernier puisse être accessible en tant que réseau Wifi.

Ce tutoriel s'adresse également dans le cas où vous avez perdu vos mots de passe d'accès (réseau wifi ou WebRepl) ou bien que vous souhaitez partir sur des bases saines.\\

\messageBox{Information}{green}{white}{Le temps estimé pour configurer l'ESP8266 est de 25 min}{black}


\section{Conventions}


\subsection*{Commandes}

Les commandes à saisir sont dans des encadrés similaires : \\
\begin{Bash}{Exemple de commande}
sudo apt-get update
\end{Bash}

\subsection*{Références et repères}

Dans un souci de clarté : 

\begin{itemize}


	\item Les fichiers sont indiqués par le repère \file{nom du fichier}
	\item Les dossiers sont indiqués par le repère \dir{nom du dossier}
	\item Les touches du clavier et le texte à saisir au clavier sont indiqués par le repère \shortcut{Raccourci clavier ou texte à saisir}
	\item Les bibliothèques, logiciels et utilitaires sont indiqués par le repère \lib{nom de l'utilitaire}

\end{itemize}


\chapter{Pré-requis}

\section{Matériel}

Pour réaliser ce tutoriel, vous aurez besoin de 

\begin{itemize}
    \item Un ordinateur (Linux, Apple ou Windows)
    \item Un ESP8266 (NodeMCU)
    \item Un câble USB Micro Type-B
    \img{Images/usb.png}{Un câble USB Micro type-B}{0.2}
    
\end{itemize}

\section{Mise à jour des systèmes UNIX}

Avant toute chose, il convient de mettre à jour la liste des paquets et de mettre à jour les logiciels déjà présents sur votre ordinateur si ce dernier est sous LINUX (UNIX). \\
Les commandes suivantes sont à saisir dans un terminal.

\subsection{Mise à jour de la liste des paquets}

\begin{Bash}{Mise à jour de la liste des paquets}
sudo apt-get update
\end{Bash}


\subsection{Mise à jour des logiciels}
\begin{Bash}{Mise à jour des logiciels}
sudo apt-get -y upgrade
\end{Bash}

\textit{Le -y sert à accepter automatiquement la mise à jour.}

\subsection{Mise à jour de Python}

Il conviendra d'installer au minimum la version 3.6 de Python. \\
Pour vérifier votre version, ouvrez un terminal et saisissez la commande 

\begin{Bash}{Vérification de la version de python}
python3
\end{Bash}

Si l'invité de commande Python suivant apparaît, la version est présente. \\
Pour quitter l'interpréteur python, il suffit de saisir \shortcut{exit()} dans l'interpréteur ou bien de faire \shortcut{Ctrl  +z}

\img{Images/python.png}{Invité de commande Python}{0.6}
Le cas échéant, je vous invite à saisir la commande suivante :

\begin{Bash}{Installation de Python 3.7}
sudo apt-get -y install python3.7
\end{Bash}

