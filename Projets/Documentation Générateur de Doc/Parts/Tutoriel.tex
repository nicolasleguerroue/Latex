
\chapter{Introduction}

Cet article a pour but de détailler la génération d'une documentation sur le site \url{http://electron-56.ddns.net}

\section{Principe}

La documentation repose sur des mot-clés précis qui permettent de définir les propriétés de la documentation. Ces mot-clés seront appelés \textbf{balises}. \newline
Certaines balises permettront de délimiter du contenu de code tandis que d'autres permettrons de stocker des informations de type, de description, etc... \newline

\bold{Chaque balise commence obligatoirement avec le caractère @ (arobase)} et doit être considérée comme un commentaire dans le langage à documenter. \newline


Voici un exemple de syntaxe : 


\begin{Bash}
@project Nom du projet
@author Auteur
@language Langage
@lv version_du_langage
@version version_du_projet
@date date_du_projet
\end{Bash}



\section{Remarques importantes}

Ce modèle de documentation présenté est valable uniquement pour les langages proposant des fonctions et procédures ayant un nom, le tout encapsulé dans une classe. \newline
Les balises de classes et de méthode ne seront pas valide pour documenter un fichier CSS. \footnote{Pour une documentation CSS, se reporter ici}
Il est obligatoire de limiter à une classe chaque fichier analysé.
Un fichier qui contiendrait deux classes soulèverait une erreur de la part du site.

\chapter{Les types de balise}

Il existe deux types de balises : 


\begin{enumerate}
    \item les balises \bold{uniques}
    \item les balises \bold{de classe et méthode (ou fonctions)}
\end{enumerate}


\chapter{Les balises uniques}

Dans le cas des balises uniques, ces balises peuvent être situées \bold{n'importe où dans le fichier source}, à condition que les balises ne soit pas exécutées (ou interprétées) comme du code.

\section{Présentation des balises}

\begin{enumerate}
    \item 
    \begin{Bash}{Le nom du projet }
    @project Nom du projet
    \end{Bash}
    
    \item 
    \begin{Bash}{Le ou les auteurs}
    @author Auteur
    \end{Bash}

    \item 
    \begin{Bash}{Le langage}
    @language Langage
    \end{Bash}
    
    \item 
    \begin{Bash}{La version du langage}
    @lv Version du langage
    \end{Bash}
    
    \item 
    \begin{Bash}{La date du projet}
    @date XX/XX/XXXX
    \end{Bash}
    
    \item 
    \begin{Bash}{Le nom de la classe}
    @class Nom de la classe
    \end{Bash}
    
    \item La description du projet \newline
    
    Il est possible de faire une description sur plusieurs ligne en ajoutant une balise @presentation à chaque nouvelle ligne.
    \begin{Bash}
    @presentation Description du projet
    @presentation Suite de la description du projet
    @presentation Suite de la description du projet
    \end{Bash}
  
\end{enumerate}


\section{Gestion de l'héritage}

Il est possible d'afficher les méthodes de la classe parent dans la documentation. Pour cela, dans la classe Fille, il faut rajouter la balise \bold{@parent} avec comme argument le nom du fichier comportant la classe Mère.
\begin{Python}
@parent File.py 
\end{Python}
Cette déclaration se fait n'importe où dans le fichier.

\section{Exemple}


Voici un exemple complet de l'utilisation des balises uniques (Exemple avec le langage Python)

\begin{Python}
"""
@project générateur de fichier LTSpice
@author Nicolas LE GUERROUE
@language PYTHON
@lv 3.7
@version 1.0
@date 05 juin 2020
@presentation La bibliothèque LTSpice permet de générer des fichiers LTSpice en commandes Python.
@presentation L'unité de placement des composants est une valeur entière qui, unitairement, représente le coefficient commun de taille des composants LTSpice
@presentation Le curseur se déplacant de 16 en 16, cette unité de mesure introduite par la bibliothèque permet de toujours bien placer les composants pour qu'ils soient
@presentation accessibles par le curseur
@class LTSpice
"""
\end{Python}


\section{Levée d'avertissement}

Une absence constatée d'une des balises soulèvera un avertissement de la part du site. La documentation va néanmoins s'afficher sur le site, sans les informations absentes.

\img{Images/error.png}{Levée d'erreur avec une balise @date absente}{0.8}

\newpage

\chapter{Les balises de classe et \\de méthode}

Pour les balises de classe et de méthode, ces balises doivent être situées \bold{entre une balise @begin et une balise @end.}

\section{Présentation des balises}

\begin{enumerate}
    \item Le \bold{nom de la méthode ou de la fonction}
\begin{Bash}{nom de la méthode ou de la fonction}
@method Nom_de_la_methode
\end{Bash}
    
   Pour documenter un constructeur, il faut remplacer \bold{@method} par \bold{@constructor}
    
    \item La \bold{portée de la méthode ou fonction}
        
        A ce propos, 3 portées sont disponibles :
        
        \begin{enumerate}
            \item portée \bold{publique} \newline
            La méthode est accessible dès lors de l'instanciation de la classe.
\begin{Bash}{Portée publique}
@range public
#ou bien
@range publique
\end{Bash}
Il est possible de mettre d'autres variantes du moment que le contenu de la balise contienne au moins le mot 'publi'

            \item portée \bold{protégée} \newline
            La méthode est accessible en interne de la classe et par héritage.
            
\begin{Bash}{Portée protégée}
@range protected
#ou bien
@range protégé(e)
\end{Bash}
Il est possible de mettre d'autre variante du moment que le contenu de la balise contienne au moins le mot 'prote'

            \item portée \bold{privée} \newline
            La méthode est accessible uniquement au sein de la classe.
\begin{Bash}{Portée privée}
@range private
#ou bien
@range privé(e)
\end{Bash}
Il est possible de mettre d'autre variante du moment que le contenu de la balise contienne au moins le mot 'priv'

        \end{enumerate}

    \item La \bold{description de la méthode} \newline
    Le contenu de la description de la méthode doit être contenu entre les balises @des et @sed
    \begin{Bash}{Description}
    @des
    Description de la méthode
    Suite de la description de la méthode
    Suite de la description de la méthode
    @sed
    \end{Bash}

    \item Les \bold{arguments de la méthode}
    Pour déclarer un paramètre attendu, il convient de mettre une balise @input pour chaque argument.
    La ligne de balise @input doit contenir une balise de type. \newline
    Quelques types d'arguments\footnote{Tous les types pris en charge sont situés en annexes (Types supportés)} :
    
    \begin{enumerate}
    \item \bold{@integer} - Entier 
    \item \bold{@string} - Chaine de caractère 
    \item \bold{@float} - Nombre décimal (ou @double) 
    \item \bold{@list} - Liste 
    \end{enumerate}
    Enfin, on ajoute la description de l'argument
    
    \begin{Bash}{Arguments}
    @input @integer Nombre pris en argument
    @input @string Chaine de caractère prise en argument
    \end{Bash}
    L'ordre des balises est sans importance, cependant, dans un souci de lecture, il convient de mettre la balise @input en première.\newline
    \img{Images/werror.png}{Un type @string faisant parti des types autorisés}{0.7}
    
    Lorsque une balise de type n'est pas reconnu, le nom de la balise va s'afficher dans la documentation afin de mettre en valeur le type choisi (type erroné par exemple)
    \newpage
    En cas d'erreur, la documentation va afficher ceci : \newline \newline
    \img{Images/error2.png}{Un type @str ne faisant parti des types autorisés}{0.7}
    
    Lorsque il n'y a aucun argument, ne mettez pas de balise @input.
   
    \item Les \bold{types de retour} \newline
    Pour déclarer un type de retour attendu, il convient de mettre une balise \bold{@type\_return} puis de mettre une balise de type. \newline
    Lorsqu'il n'y a pas de type de retour, précisez le type comme \bold{@none}
\begin{Bash}{Retour}
@type_return @integer
\end{Bash}
    
    \item Les \bold{descriptions de retour} \newline
    Pour expliquer un retourde méthode, il convient de mettre une balise \bold{@return} puis de mettre l'explication. \newline
    Lorsqu'il n'y a pas de type de retour, précisez que la fonction ne retourne rien \bold{@none}
\begin{Bash}{Valeur de retour}
@return False en cas de succès, True en cas d'erreur
\end{Bash}


    \item Les \bold{exemples de code} \newline
    Il est possible d'ajouter un éxemple d'utilisation de la méthode. Pour cela, il suffut d'ajouter la balise \bold{@example} puis le code associé (sur la même ligne, pas de retour à la ligne).
\begin{Bash}{Exemple}
@example x = ltpice.getX()
\end{Bash}

    \item Le \bold{code intégral de la méthode} \newline
    Afin de pouvoir récuperer le code de la méthode, il convient de placer le contenu de la méthode entre les balises \bold{@code} et \bold{@end}

    
\end{enumerate}


\newpage
\section{Exemple}


\begin{Python}
  #documentation complète d'une méthode
	"""
	@begin
	@method getX
	@range public
	@des 
	Cette méthode permet de retourner la position courante en x du curseur dans le fichier.
	@sed
	@type_return @integer
	@return La position du curseur en unité standard (unité=16)
	@example position_x = ltspice.x()
	@code
	"""
	def getX(self):
		return int(self.__x/16)
	"""
	@end
	"""
\end{Python}

\section{Levée d'avertissement}

Une absence constatée d'une des balises soulèvera un avertissement de la part du site. La documentation va néanmoins s'afficher sur le site, sans les informations absentes. \newline
\bold{Il faut être conscient qu'une absence de balise va impacter la suite de la documentation, des décalages dans les noms, arguments et autres risquent d'apparaître.}

    \img{Images/error3.png}{Deux balises manquantes}{0.7}
    

\chapter{Annexe}

\section{Types supportés}

Voici la liste des types pris en charges.

\begin{enumerate}
    \item caractère - \bold{@char}
    \item caractère non signé- \bold{@uchar}
    \item Chaîne de caractère - \bold{@string}
    \item Entier signé - \bold{@integer}
    \item Entier non signé - \bold{@uinteger}
    \item Booléen - \bold{@bool}
    \item Nombre décimal - \bold{@double} ou \bold{@float}
    \item Type automatique (C++)- \bold{@auto}
    \item Entier signé sur 8 bits - \bold{@int8\_t}
    \item Entier non signé sur 8 bits - \bold{@uint8\_t}
    \item Entier signé sur 16 bits - \bold{@int16\_t}
    \item Entier non signé sur 16 bits - \bold{@uint16\_t}    
    \item Vecteur (C++) - \bold{@vector}    
    \item Tableau (C++, PHP) - \bold{@array}  
    \item Iterateur (C++) - \bold{@iterator}  
    
    \item Pointeur sur caractère - \bold{@*char}
    \item Pointeur sur caractère non signé- \bold{@*uchar}
    \item Pointeur sur chaîne de caractère - \bold{@*string}
    \item Pointeur sur entier signé - \bold{@*integer}
    \item Pointeur sur entier non signé - \bold{@*uinteger}
    \item Pointeur sur booléen - \bold{@*bool}
    \item Pointeur sur nombre décimal - \bold{@*double} ou \bold{@*float}
    \item Pointeur sur entier signé sur 8 bits - \bold{@*int8\_t}
    \item Pointeur sur entier non signé sur 8 bits - \bold{@*uint8\_t}
    \item pointeur sur entier signé sur 16 bits - \bold{@*int16\_t}
    \item Pointeur sur entier non signé sur 16 bits - \bold{@*uint16\_t}    
    \item Pointeur sur Iterateur (C++) - \bold{@*iterator}    
    
    \item Référence sur caractère - \bold{@\&char}
    \item Référence sur caractère non signé- \bold{@\&uchar}
    \item Référence sur chaîne de caractère - \bold{@\&string}
    \item Référence sur entier signé - \bold{@\&integer}
    \item Référence sur entier non signé - \bold{@\&uinteger}
    \item Référence sur booléen - \bold{@\&bool}
    \item Référence sur nombre décimal - \bold{@\&double} ou \bold{@\&float}
    \item Référence sur entier signé sur 8 bits - \bold{@\&int8\_t}
    \item Référence sur entier non signé sur 8 bits - \bold{@\&uint8\_t}
    \item Référence sur entier signé sur 16 bits - \bold{@\&int16\_t}
    \item Référence sur entier non signé sur 16 bits - \bold{@\&uint16\_t}    
    \item Référence sur Iterateur (C++) - \bold{@\&iterator}  
    
\end{enumerate}
