
\section{Introduction}

Ceci n'est pas un tutoriel mais simplement un recueil de conseils et de renseignements notés au cours de réunions du mardi soir au Fablab \#4 et qui ont permis de mener à bien l'assemblage des composants électroniques sur le circuit imprimé.

\section{Rappels}

La passerelle reçoit sur sa barrette connecteurs femelles un \bold{Arduino Nano} et est reliée par une nappe à 8 conducteurs à un \bold{module transmetteur NRF24}.

L'Arduino Nano est relié au Raspberry Pi de votre plateforme Domoticz par un câble USB (transmission d'infos et alimentation 5v) et alimentera la passerelle en 5v.

Le module transmetteur NRF24 assure la liaison radio avec les sondes. 

\section{Liste du matériel de la passerelle}

\begin{itemize}
    \item 3 Led
    \item 3 résistances de $270~\Omega$
    \item 2 condensateurs électrolytiques ( $10~\mu F, 16~V$ – cylindriques noirs)
    \item 2 condensateurs céramiques monolithiques ( $100nF, 50~V$ – couleur jaune foncé )
    \item 1 régulateur 3.3v HT7533-1
    \item 1 module transmetteur NRF24
    \item 1 bouton poussoir
    \item 1 circuit imprimé
    \item 1 barrette connecteurs femelle ({\color{red}{déjà montée sur le circuit imprimé}})
    \item 1 nappe 8 conducteurs ({\color{red}{dont l'un porte un liseré rouge}} ),
    \item 2 jumpers
\end{itemize}

\section{Placement des composants}

\subsection{Vue de dessus du circuit}
\img{\rootImages/pin}{Circuit imprimé vu de dessus, coté composants}{0.5}
\subsection{Étapes}
\begin{enumerate}
\item Souder la barrette déjà en place
\item Souder les 8 brins de la nappe en respectant les consignes suivantes :
\img{\rootImages/nappe}{Emplacement de la nappe}{0.5}
\end{enumerate}

Puis dans l'ordre que vous souhaitez : 
\begin{itemize}
    \item Led rouge en D1 – Led jaune en D2 – Led verte en D3 ({\color{red}{sont polarisées}}, patte longue au + ,  patte courte, méplat sur la led au - ),
    \item Les 3 résistances sont à souder en R1, R2 et R3
    \item Les 2 condensateurs électrolytiques sont à souder en C1 et C2 ({\color{red}{sont polarisées}}, le corps du condo est noir avec une bande grisée, la patte de ce côté est le - ),
    \item Les 2 condensateurs céramiques sont à souder en C3 et C4
    \item Le bouton poussoir en SW1
    \item Le jumper D13 (Arduino Nano) en J12
    \item Le jumper 5v (Arduino Nano) en J10
    \item Le régulateur 3.3v HT7533 est à souder suivant les conseils de la section suivante
\end{itemize}

\section{Mise en place du régulateur de tension}

Comme vous l'avez surement remarqué, l'implantation précédente correspond à un régulateur LE33 et non à un HT7533

Si vous voulez utiliser un HT7533, il faut adapter le brochage du HT7533 au circuit.

\begin{itemize}
    \item Vin : ENtrée 5v
    \item GND : la masse
    \item Vout : sortie 3.3V
\end{itemize}

\img{\rootImages/pin}{Broches du HT7533}{0.5}

\subsection{Adaptation des broches du HT7533 au schéma de la passerelle}

Il faut que les broches \bold{GND, Vin et Vout} rentrent dans les mêmes broches que celle du schéma de la passerelle. Même si les broches ne sont pas dans le même ordre, c'est assez simple à faire en tordant les broches du HT7533 avec une petite pince plate.\\

Sur le HT7533, sans que les broches se touchent, \bold{on tord Vin vers l'avant, Vout vers l'arrière et on ramène GND au milieu.}

\img{\rootImages/ht}{Insertion du HT7533}{0.35}

\section{Rendus}

\img{\rootImages/side}{Vue de coté}{0.5}
\img{\rootImages/passerelle_emetteur}{Vue de la passerelle et de l'émetteur}{0.5}
\img{\rootImages/dessus}{Vue de dessus}{0.45}