 \documentclass[12pt]{report}  
%book, report, beamer, article

\usepackage[frenchb]{babel}     %french language
\usepackage{Utils/Utils}        %Utils package
%\usepackage{layout}

%#############################################################
%#### Settings
%#############################################################
%#############################################################
%If you want to set on fullpage, change this bloc
%#############################################################
\geometry{hmargin=4cm,vmargin=4cm}
%#############################################################
%#############################################################
%If you want rename the chapter name, change the value of argument
%#############################################################
\setAliasChapter{Chapitre}
%#############################################################
%#############################################################
%If you want to add presentation, modify the next bloc
%The firt line is about the header : 
% {left content}{center content}{right content}
%The second line is about the footer : 
% {left content}{center content}{right content}
%####
%to get the current chapter name, use \currentChapter command as content
%to get the current number page, use \currentPagecommand as content
\addPresentation
{Bluetooth \& Arduino} {} {\currentChapter}
{Nicolas LE GUERROUE} {} {\currentPage}
%#############################################################
%Change the width of footer line and header line
%To delete it, set value to 0
\setHeaderLine{0.2}
\setFooterLine{0.2}
%#############################################################
%Change the name of the section "Nomenclature"
%To delete it, set value to 0
\renewcommand{\nomname}{Conventions}
  







\begin{document}

\setHeader{Projet Voltaire}{Nicolas LE GUERROUÉ}{juin 2020}
\maketitle
\tableofcontents

\section{« ça », « çà » ou « sa » ?}

Si l'on peut remplacer le mot qui pose problème par « cela », c'est qu'il s'agit du pronom démonstratif « ça » : Ça me semble facile (= Cela me semble facile). \newline

Dans les autres cas, il s'agit du déterminant possessif « sa », qui précède un nom féminin et indique l'appartenance : sa remarque, sa veste. \newline

Précision de l'expert
« Çà », avec un accent grave, ne s'emploie guère aujourd'hui que dans l'expression « çà et là ».


\section{« J'ENVOIE », « TU ENVOIES, « IL ENVOIE »}
« J'ENVOIE », « TU ENVOIES, « IL ENVOIE »
Au présent de l'indicatif, les terminaisons des verbes du premier groupe – c'est‑à‑dire tous ceux qui, à l'infinitif, se terminent par « ‑er » (sauf « aller », qui est un verbe irrégulier) – sont : ‑e, ‑es, ‑e, ‑ons, ‑ez, ‑ent.
« Regarder » : je regarde, tu regardes, il regarde, nous regardons, vous regardez, ils regardent.

Les verbes qui, à l'infinitif, se terminent par « ‑oyer » (comme « envoyer ») sont eux aussi des verbes du premier groupe et ont donc les mêmes terminaisons.
Remarquez bien le « e », notamment aux trois personnes du singulier.
« Envoyer » : j'envoie, tu envoies, il envoie, nous envoyons, vous envoyez, ils envoient.

Aux trois personnes du singulier du présent de l'indicatif, la règle est la même pour les verbes qui se terminent par « ‑ayer » (comme « payer » : je paie, tu paies, il paie), « ‑uyer » (comme « s'ennuyer » : je m'ennuie, tu t'ennuies, il s'ennuie), « ‑ier » (comme « vérifier » : je vérifie, tu vérifies, il vérifie) et « ‑uer » (comme « continuer » : je continue, tu continues, il continue), puisque ce sont aussi des verbes du premier groupe.

Précisions de l'expert
Pour les verbes en « ‑ayer » uniquement, deux formes sont possibles : avec un « i » (je paie, tu paies…), orthographe la plus courante aujourd'hui – et que nous employons donc dans tous les exercices –, ou avec un « y » (je paye, tu payes…).

Au présent du subjonctif, les terminaisons des verbes en « ‑ayer », « ‑oyer », « ‑uyer », « ‑ier » et « ‑uer » sont les mêmes qu'au présent de l'indicatif, sauf pour « nous » et « vous » : Il faut que j'envoie, que tu envoies, qu'il envoie, que nous envoyions, que vous envoyiez, qu'ils envoient.



\section{« dans » ou « d'en » ?}

Astuce Voir la vidéo Voir le cours
« D'en », contraction de « de » et de « en », est suivi d'un verbe à l'infinitif : Il n'a pas envie d'en parler.

Dans la quasi‑totalité des autres cas, il s'agit de la préposition « dans » : dans l'immeuble.






\section{« sans », « s'en » ou « c'en » ?}

Le remplacement par « cela en » est possible ? Alors il faut écrire « c'en », qui est la contraction de « ce » et de « en ».
C'en est trop, c'est insupportable ! => Cela en est trop, c'est insupportable !

Le remplacement par « m'en » ou « t'en » est possible ? Alors il faut écrire « s'en », qui est la contraction de « se » et de « en ».




\section{« mieux »}

Astuce Voir la vidéo Voir le cours
« Mieux » prend toujours un « x » : Le mieux est l'ennemi du bien.
Astuce : « Mieux, c'est mieux avec un “x” ! »



\section{« ils arrivent »}


Lorsqu'il est conjugué à la 3e personne du pluriel (« ils », « elles », etc.), un verbe a toujours une terminaison en « nt » : Ils arrivent à 9 heures.
Il ne faut donc jamais écrire avec un « s » les verbes conjugués à la 3e personne du pluriel.

\section{« TU MANGES », « TU MANGERAS »}

Dans la grande majorité des cas, un verbe conjugué à la 2e personne du singulier (« tu ») se termine par un « s » : tu commences, tu finiras, tu prenais.

Précisions de l'expert
Font exception, à la 2e personne du singulier du présent de l'indicatif, les formes suivantes : tu peux, tu veux, tu vaux.

À l'impératif – mode que l'on emploie pour exprimer un ordre, un conseil ou une défense –, les formes qui, à la 2e personne du singulier, se terminent par un « e » muet ne prennent généralement pas de « s » non plus : Ferme la fenêtre ! Ouvre la bouche !



\section{« FERME LA FENÊTRE ! »}

À l'impératif – mode qui permet d'exprimer un ordre, un conseil ou une défense –, les formes se terminant par un « e » muet à la 2e personne du singulier ne prennent pas de « s » : Ferme la fenêtre ! Ouvre la porte !

En revanche, si elles sont immédiatement suivies de « en » ou de « y », on ajoute un « s », que l'on entend : Laisses‑en un peu ! Penses‑y !

Précision de l'expert
Si « en » est suivi d'un nom (en silence) ou d'un verbe au participe présent (en fermant) – c'est‑à‑dire d'un verbe qui se termine par « ant » –, c'est alors une préposition. Le verbe à l'impératif, à la 2e personne du singulier, qui précède ce « en » et qui se termine par un « e » muet ne prend pas de « s » : Mange en silence ! Mange en fermant la bouche !


\section{« connexion »}

« Connexion » et « déconnexion » s'écrivent avec un « x », bien que ces noms soient de la même famille que « déconnecter » et « connectique » : une connexion à Internet.
C'est en anglais qu'on écrit connection.

Astuce : la « connexion » est un croisement d'informations, que représente le « x ».



\section{« avenir » ou « à venir » ?}

Astuce Voir la vidéo Voir le cours
Le nom « avenir » est généralement précédé d'un déterminant (l'avenir, un brillant avenir, son avenir).

Mais, lorsque le terme sur lequel on s'interroge est introduit par un verbe (hésiter à venir, tarder à venir) ou lorsqu'il suit un nom (les générations à venir, c'est‑à‑dire « les générations qui vont venir »), il faut écrire « à venir », en deux mots.

\section{« je peux », « tu peux », mais « il peut »}

Le « x » correspond à la 1re et à la 2e personne du singulier du verbe « pouvoir » conjugué au présent de l'indicatif : je peux, tu peux.

Le « t » est la marque de la 3e personne du singulier (« il », « elle », « on », « cela », etc.) : il peut.

Précision de l'expert
Même règle pour « vouloir » et « valoir » conjugués au présent de l'indicatif : je veux, tu veux, il veut ; je vaux, tu vaux, il vaut.



\section{« est » ou « et » ?}

Si l'on peut remplacer le mot en question par « était », c'est qu'il s'agit d'une forme du verbe « être », qui s'écrit « est » : Il est dans son bureau.

Dans le cas contraire, il s'agit de la conjonction de coordination « et » : La directrice et son adjoint viennent d'arriver.





\section{« je savais », « tu savais », « il savait »}

Un verbe conjugué à l'imparfait se prononce de la même façon aux trois personnes du singulier (« je », « tu », « il »), mais les terminaisons ne sont pas les mêmes.

Aux deux premières personnes du singulier (« je » et « tu »), le verbe prend un « s » : je savais, tu savais.

En revanche, à la 3e personne du singulier (« il », « elle », « on », « cela », etc.), il prend un « t » : il savait.


\section{« LA », « L'A(S) » OU « LÀ » ?}

Si l'on peut remplacer le mot par « les », il faut écrire « la ».
La lettre, il la lit => Les lettres, il les lit.

Si l'on peut remplacer le mot par « l'avais » ou « l'avait », il faut écrire « l'as » (2e personne du singulier) ou « l'a » (3e personne du singulier), en fonction du sujet.
Tu l'as signé => Tu l'avais signé.
Il l'a signé => Il l'avait signé.

Sinon, il faut écrire « là », qui marque le lieu ou renforce un démonstratif.
Il est là.
Ce jour‑là.

\section{« -é » ou « -er » ?}

Si l'on peut remplacer le mot par « finir », « faire », « dire » ou tout autre infinitif qui ne soit pas du premier groupe, c'est qu'il s'agit d'un infinitif, qui se termine par « ‑er » : Je vais acheter cette voiture (on peut dire : Je vais finir).

Sinon, c'est qu'il s'agit du participe passé : J'ai acheté cette voiture.



\section{« NOTRE » OU « NÔTRE », « VOTRE » OU « VÔTRE » ?}

Lorsque le mot précède un nom, il ne faut pas mettre d'accent circonflexe. Il s'agit du déterminant possessif « notre » ou « votre ».
C'est notre entreprise.
Voici votre nouveau bureau.

Mais, quand le mot ne précède pas un nom, le « o » prend un accent circonflexe.
Dans ce cas, le mot est très souvent précédé de « le », « la », « les », « du » ou « des » (il s'agit du pronom possessif).
Cette entreprise, c'est la nôtre.
Ce bureau est le vôtre.

Précision de l'expert
Dans les deux exemples ci‑dessous, le mot n'est pas précédé de « le », « la », « les », « du » ou « des ». Mais, comme il ne précède pas non plus un nom, il faut bien mettre un accent circonflexe.
Amicalement vôtre.
Ces idées, nous les avons faites nôtres.
Ces deux tournures, plus anciennes ou plus littéraires, sont moins fréquentes.



\section{« avoir à faire » ou « avoir affaire » ?}

Astuce Voir la vidéo Voir le cours
Si l'on peut remplacer l'expression par « avoir à réaliser (quelque chose) » ou « avoir à refaire », il faut écrire « à faire » : J'ai beaucoup de choses à faire ce matin.

Sinon, il faut écrire « avoir affaire », qui est souvent suivi de la préposition « à » : J'ai eu affaire à lui.


\section{}
« nous nous amusons »
Astuce Voir la vidéo Voir le cours
Conjugué à la 1re personne du pluriel (« nous »), un verbe se termine toujours par « s » (jamais par « t ») : Nous attendons le train.



\section{« nous nous amusons »}

Astuce Voir la vidéo Voir le cours
Conjugué à la 1re personne du pluriel (« nous »), un verbe se termine toujours par « s » (jamais par « t ») : Nous attendons le train.

\section{« quand » ou « qu'en » ?}

Si oralement on peut décomposer le mot en « que en », il faut écrire « qu'en » : Il n'a qu'en partie raison.

Sinon, il faut écrire « quand » : Quand il arrive le matin.

Précision de l'expert
« Quant », avec un « t », signifie « en ce qui concerne », « pour ce qui est de », et est suivi de « à » ou de « aux ».

\section{« A » OU « À » ?}

Si l'on peut remplacer le mot par « avait », c'est qu'il s'agit d'une forme du verbe « avoir », qui s'écrit « a », sans accent : Il a trente ans (on peut dire : Il avait trente ans).

Sinon, c'est qu'il s'agit de la préposition « à », qui prend toujours un accent grave : Ils se sont mis à l'abri.

Précisions de l'expert
La forme « a » correspond au verbe « avoir » comme à l'auxiliaire « avoir ».
On parle d'« auxiliaire » lorsque le verbe « avoir » permet de construire les temps composés des autres verbes, c'est‑à‑dire les temps formés de cet auxiliaire et du participe passé des verbes en question (« j'ai regardé », « tu avais fini », « il a vu », « pour que nous ayons pris », etc.).
Il n'y a que deux auxiliaires en français : « avoir » et « être ».

\section{« IL SE DÉTEND », « IL RÉPOND »}

À la 3e personne du singulier (« il », « elle », « on », « cela », etc.) du présent de l'indicatif, les verbes dont l'infinitif se termine par « ‑endre » (comme « vendre ») ou « ‑ondre » (comme « répondre ») prennent un « d » (et non un « t »).
Il vend sa voiture.
Elle répond à cette question.

Précisions de l'expert
Même règle pour les verbes en « ‑andre » (comme « répandre »), en « ‑erdre » (comme « perdre »), en « ‑ordre » (comme « mordre ») et en « ‑oudre » (comme « coudre »).

Les verbes en « ‑soudre » (comme « résoudre ») ne se conjuguent pas comme les verbes en « ‑oudre », puisque, à la 3e personne du singulier du présent de l'indicatif, ils prennent, eux, un « t », comme les verbes en « ‑indre » (« atteindre », par exemple).
Ces verbes en « ‑soudre » et en « ‑indre » sont vus dans le module Excellence.

\section{« une qualité », « l'amitié »}

Astuce Voir la vidéo Voir le cours
Les noms féminins se terminant par « ‑té » ou « ‑tié » ne prennent pas de « e » : Elle a de l'autorité. C'est une preuve d'amitié.

Sauf « butée », « compotée », « députée », « dictée », « jetée », « montée », « nuitée », « pâtée », « portée », « potée », « remontée », « tétée », « tripotée », ainsi que les noms exprimant un contenu (« assiettée », « brouettée », « pelletée », etc.).

\section{« cauchemar »}

Astuce Voir la vidéo Voir le cours
Le nom « cauchemar » se termine par un « r » : faire des cauchemars.
Pas de « d » à la fin, donc, malgré le verbe « cauchemarder » et l'adjectif « cauchemardesque ».

\section{accent ou pas ?}

Astuce Voir la vidéo Voir le cours
Il n'y a jamais d'accent lorsque le « e » est suivi :
– d'un « x » (exact, examen) ;
– de deux consonnes (essence, belle, espoir, destin), sauf si la première consonne est suivie d'un « h », d'un « l » ou d'un « r » (échange, éclairer, écrit) ;
– de plus de 2 consonnes (technique, perspicace).

\section{« qu'il ait » ou « qu'il est » ?}

Astuce Voir la vidéo Voir le cours
Si l'on peut remplacer la forme qui pose problème par « que nous ayons », il faut écrire « qu'il ait », avec un « t », qui est une forme du verbe « avoir » (« ait » et « avoir » commencent par la même lettre) : pour qu'il ait le temps => pour que nous ayons le temps.

Sinon, il faut écrire « qu'il est », qui est une forme du verbe « être » (« est » et « être » commencent par la même lettre) : J'espère qu'il est encore là.

\section{« QUAND » OU « QUANT » ?}

« Quand » introduit presque toujours une notion de temps. On peut remplacer ce mot par « lorsque » (ou par « à quel moment », « le moment où », etc.).
Quand il pleut => Lorsqu'il pleut.
Quand partez‑vous ? => À quel moment partez‑vous ?

« Quant », avec un « t », signifie « pour ce qui est de » et est suivi de « à » ou de « au(x) » : Quant aux artistes, ils sortent par cette porte.

Précision de l'expert
« Quand même », qui signifie « malgré tout », « tout de même », s'écrit avec un « d » : Ce serait quand même plus simple !

\section{« ENVOIE » OU « ENVOI » ?}

Au présent de l'indicatif, le verbe « envoyer » se conjugue ainsi aux trois personnes du singulier : j'envoie, tu envoies, il envoie, avec un « e » à la fin, comme tous les verbes du premier groupe (c'est‑à‑dire tous ceux qui, à l'infinitif, se terminent par « ‑er », sauf « aller », qui est un verbe irrégulier).

Au présent du subjonctif, les terminaisons sont les mêmes à ces trois personnes du singulier : il faut que j'envoie, que tu envoies, qu'il envoie.

Lorsqu'il s'agit du verbe « envoyer » conjugué à l'une de ces trois personnes du singulier, il peut toujours être précédé de « je », « tu », « il » ou « elle » (c'est‑à‑dire d'un pronom sujet). Il faut mettre un « e » à la fin (ou « es », si le verbe est conjugué à la 2e personne du singulier).
J'envoie ce message. Tu envoies ce message. (Les pronoms sont déjà exprimés.)
La directrice envoie un message. (On peut remplacer « la directrice » par le pronom « elle ».)

Sinon, c'est qu'il s'agit du nom « envoi », qui ne prend jamais de « e » à la fin (des frais d'envoi) et qui se termine donc par « ois » au pluriel (des envois de marchandises).

Précision de l'expert
Même règle pour « renvoie » (verbe) et « renvoi » (nom).

\section{« malgré »}


« Malgré » est invariable, il ne prend donc jamais de « s » : Malgré les échecs, il ne désespère pas.

\section{« ni » ou « n'y » ?}

Astuce Voir la vidéo Voir le cours
Si le mot est immédiatement suivi d'un verbe, il s'agit sans doute de « n'y », contraction de « ne » et de « y ». La présence, peu après, d'un terme renforçant la négation (« pas », « jamais », « plus », « guère », etc.) le confirme : Il n'y a jamais cru.

Sinon, il s'agit de la conjonction de coordination « ni », qui est souvent répétée : ni l'un ni l'autre, ni oui ni non.

\section{« tous les »}

Astuce Voir la vidéo Voir le cours
On écrit forcément « tous », avec un « s », devant un nom masculin au pluriel : tous les jours, tous nos amis.

Précisions de l'expert
On écrit « tout », avec un « t », devant un nom masculin au singulier : tout le travail, tout le temps.

On écrit « toute » devant un nom féminin au singulier (toute l'entreprise) et « toutes » devant un nom féminin au pluriel (toutes les entreprises).

\section{« LA PLUPART SONT »}

L'accord du verbe se fait, en genre et en nombre, avec le complément de « la plupart » : La plupart des vacanciers sont déjà partis.

Si « la plupart » est employé seul, c'est‑à‑dire sans complément, le verbe se met au pluriel : La plupart sont déjà là.

Précision de l'expert
Lorsque le complément de « la plupart » est au singulier (ce qui est rare), le verbe se met lui aussi au singulier : La plupart du travail se fait dans cet atelier.

\section{« ou » ou « où » ?}

Astuce Voir la vidéo Voir le cours
Si l'on peut remplacer le mot par « ou bien », c'est qu'il s'agit de « ou » (conjonction de coordination), sans accent : Thé ou café ?

Sinon, il s'agit de « où » (adverbe qui marque le lieu ou le temps, ou pronom relatif), avec un accent grave : Où est‑il ? La maison où j'ai vécu. L'année où nous avons déménagé.

\section{« CONSEIL » OU « CONSEILLE », « TRAVAIL » OU « TRAVAILLE » ?}

« Conseille », « détaille », « réveille » et « travaille » sont des verbes conjugués.

« Conseil », « détail », « réveil » et « travail » sont des noms.

Pour savoir s'il s'agit d'un verbe conjugué, il faut essayer de mettre au futur le terme qui pose problème.
Il m'a donné ce conseil => « Il m'a donné ce conseillera » ne veut rien dire. Ici, il faut donc écrire « conseil » (= nom).
Il détaille son programme => « Il détaillera son programme » est une phrase correcte. Ici, il faut donc écrire « détaille » (= verbe).

\section{« JE LEUR DIS », « LEURS CLIENTS », « LES LEURS »}

Si l'on peut remplacer « leur » par « lui » en mettant la phrase au singulier, alors « leur » ne prend jamais de « s », même s'il renvoie à… plusieurs personnes ! Il s'agit d'un pronom personnel invariable.
Je vais leur parler => Je vais lui parler.

Lorsqu'il précède un nom, « leur » s'accorde (en nombre uniquement) avec ce nom (il s'agit alors d'un déterminant possessif) : leur appartement, leur maison, leurs dents, leurs jouets.
Impossible, dans ce cas, de remplacer « leur(s) » par « lui ».
Leurs jouets sont rangés => « Lui jouets » = IMPOSSIBLE : on met donc un « s » à « leurs », puisque le nom « jouets » est au pluriel.
Leur maison est à vendre => « Lui maison » = IMPOSSIBLE : on ne met donc pas de « s » à « leur », puisqu'il n'y a qu'une maison.

Enfin, « leur » peut être immédiatement précédé de « le », « la », « les », « du » ou « des ». Il fait alors partie d'un pronom possessif et s'accorde, toujours en nombre uniquement, avec l'article qui le précède : Notre voiture est ici, la leur est là‑bas. Nous serons des leurs.

\section{« ENTRETIEN », « ENTRETIENS » OU « ENTRETIENT » ?}

Le nom « entretien » ne prend ni « s » ni « t » au singulier : Son entretien s'est bien passé.
Mais il prend bien sûr un « s » au pluriel : Les entretiens annuels commencent.

Le verbe « entretenir », lui, se conjugue ainsi au présent de l'indicatif : j'entretiens, tu entretiens, il entretient.
S'il s'agit du verbe conjugué, on peut changer le temps et mettre le verbe au futur.
Tu entretiens bien ton appartement => Tu entretiendras bien ton appartement.

Même règle pour « soutien » (nom = pas de « s » au singulier) et « soutiens / soutient » (verbe conjugué), ainsi que pour « maintien » (nom = pas de « s » au singulier) et « maintiens / maintient » (verbe conjugué).

\section{« développer », « envelopper »}

Astuce Voir la vidéo Voir le cours
Dans les verbes « développer » et « envelopper », ainsi que dans les mots de la même famille (« développement », « développeur », « enveloppe », « enveloppement », etc.), seule la consonne « p » est doublée : Les pays en voie de développement.

\section{« ils sont debout », « ils sont ensemble »}

Astuce Voir la vidéo Voir le cours
Les adverbes « ensemble » et « debout », comme tous les adverbes, sont invariables. Ils ne prennent donc jamais de « s ».
Ils sont restés debout dans le bus.
Elles travaillent ensemble.

Précision de l'expert
Il ne faut pas confondre l'adverbe « ensemble » avec le nom « ensemble », qui, lui, prend un « s » au pluriel : Ces grands ensembles ont été construits dans les années 1970.


\section{« parce que » ou « par ce que » ?}

Astuce Voir la vidéo Voir le cours
« Parce que », en deux mots, exprime la cause, le motif, et répond à la question : « Pourquoi ? »
Il n'est pas venu parce qu'il a encore de la fièvre => « parce que » introduit ici la réponse à la question : « Pourquoi n'est‑il pas venu ? »

« Par ce que », en trois mots, répond à la question : « Par quoi ? »
Il est surpris par ce que son collègue lui a dit => « par ce que » introduit ici la réponse à la question : « Par quoi est‑il surpris ? »

\section{« langage »}

Astuce Voir la vidéo Voir le cours
En français, le mot « langage » ne prend jamais de « u », contrairement au terme anglais language : Surveillez votre langage !

\section{« PRÊT » OU « PRÈS » ?}

« Prêt », avec un « t », est un adjectif : il peut donc être mis au féminin.
Souvent suivi de la préposition « à », il signifie « disposé », « préparé ».
Il est prêt à tout pour cela => Elle est prête à tout pour cela.

« Près », avec un « s », est un adverbe : il est donc invariable.
Souvent suivi de la préposition « de », il indique la proximité (Ils habitent près de chez nous) et peut aussi signifier « environ », « presque » (Il est près de minuit. Près de mille euros. Près du tiers de la population).

Quand il est suivi d'un verbe à l'infinitif, l'adverbe « près » – invariable, donc – indique qu'une action est sur le point de se produire.
Elle est près d'y arriver (et non : « Elle est prête d'y arriver »).

\section{« COMMENT VA-T-IL ? », MAIS « VA-T'EN ! »}

Trait d'union ou apostrophe : comment savoir ce qu'il faut mettre ?

Le « t » que l'on utilise pour faciliter la prononciation entre un verbe conjugué, se terminant par une voyelle, et un pronom personnel (« il », « elle », « on ») est encadré par deux traits d'union : Comment va‑t‑il ? Aura‑t‑on le temps ? « C'est parfait », déclare‑t‑elle.

En revanche, le « t » qui résulte de l'élision du pronom « toi » (c'est‑à‑dire de la suppression des voyelles « o » et « i ») après un verbe conjugué à l'impératif est précédé d'un trait d'union, mais suivi d'une apostrophe : Va‑t'en ! Remets‑t'en !
L'apostrophe est en effet un petit signe qui indique toujours qu'une voyelle a été supprimée – voire deux, comme dans les deux exemples ci‑dessus.
Rappelons que l'impératif est le mode qui nous permet d'exprimer un ordre, un conseil ou une défense.

Précision de l'expert
Dans « Donne‑m'en un peu ! », on met également un trait d'union entre le verbe et le pronom qui s'y rattache, mais une apostrophe après le « m » pour indiquer l'élision du pronom « moi ».

\section{« elle s'est fait faire »}

Astuce Voir la vidéo Voir le cours
S'il est suivi d'un verbe à l'infinitif, « fait » – participe passé du verbe « faire » ou « se faire » – est invariable.
Les clients, elle les a fait entrer dans son bureau.
Elle s'est fait faire des mèches.
Ils se sont fait longtemps attendre.


\section{}
On peut dire : « Est‑ce que la directrice est là ? » ou « La directrice est‑elle là ? »

En revanche, il est incorrect de dire : « Est‑ce que la directrice est‑elle là ? » On ne peut en effet associer les deux tournures, « est‑ce que » + inversion du sujet (« est‑elle »).

\section{« CRÉE » OU « CRÉÉE » ?}

Le participe passé des verbes qui, à l'infinitif, finissent par « ‑er » se termine par « é » au masculin singulier.
Regarder (infinitif) => j'ai regardé (participe passé).

Le participe passé du verbe « créer », qui se termine par « ‑er », s'écrit donc « créé » au masculin singulier, c'est‑à‑dire avec deux « é » : Le poste vient d'être créé.
Au féminin, on ajoute logiquement un « e ». Il y a donc trois « e » successifs, avec deux accents aigus sur les deux premiers : L'entreprise a été créée il y a deux ans.

La règle est la même pour les verbes « agréer » et « suppléer ».

Précision de l'expert
La forme « crée » – avec un accent aigu sur le premier « e » uniquement – existe néanmoins. Elle correspond à la 1re et à la 3e personne du singulier du présent de l'indicatif (je crée, tu crées, il crée) et du présent du subjonctif (il faut que je crée, que tu crées, qu'il crée), et à la 2e personne du singulier de l'impératif, mode qui permet d'exprimer un ordre, un conseil ou une défense (crée !).

\section{« certes »}

Astuce Voir la vidéo Voir le cours
« Certes » est invariable et s'écrit toujours avec un « s » : Certes, tu as raison.


\section{« -AMMENT » OU « -EMMENT » ?}

Lorsqu'on entend le son [aman] à la fin d'un adverbe, on doit écrire « ‑amment » ou « ‑emment », avec toujours deux « m ».
Pour savoir s'il y a un « a » ou un « e » avant ces deux « m », il faut partir de l'adjectif correspondant.

Si l'adjectif se termine par « ‑ant », l'adverbe s'écrit « ‑amment ».
Courant => couramment.
Méchant => méchamment.
Voici quelques adjectifs se terminant par « ‑ant » : abondant, bruyant, complaisant, constant, courant, élégant, étonnant, indépendant, méchant, nonchalant, obligeant, pesant, suffisant, vaillant.

Si l'adjectif se termine par « ‑ent », l'adverbe s'écrit « ‑emment ».
Intelligent => intelligemment.
Prudent => prudemment.
Voici quelques adjectifs se terminant par « ‑ent » : apparent, ardent, conscient, décent, différent, évident, fréquent, imprudent, intelligent, négligent, pertinent, prudent, violent.

« Notamment », « précipitamment » et « sciemment » sont formés sur des radicaux différents ou dérivés d'adjectifs disparus.

Précision de l'expert
Si l'on n'entend pas le son [aman] à la fin de l'adverbe, il n'y a qu'un « m » : aimablement, lentement, poliment, vraiment…


\section{« elle s'est fait faire »}

Astuce Voir la vidéo Voir le cours
S'il est suivi d'un verbe à l'infinitif, « fait » – participe passé du verbe « faire » ou « se faire » – est invariable.
Les clients, elle les a fait entrer dans son bureau.
Elle s'est fait faire des mèches.
Ils se sont fait longtemps attendre.

\section{}


\section{}


\section{}

\section{}


\section{}


\section{}

\section{}


\section{}


\section{}

\section{}


\section{}


\section{}

\section{}


\section{}


\section{}

\section{}


\section{}


\section{}

\section{}


\section{}


\section{}

\section{}


\section{}


\section{}

\section{}


\section{}


\section{}

\section{}


\section{}


\section{}

\section{}


\section{}


\section{}

\section{}


\section{}


\section{}

\section{}


\section{}


\section{}

\section{}


\section{}


\section{}

\section{}


\section{}


\section{}

\section{}


\section{}


\section{}

\section{}


\section{}


\section{}

\section{}


\section{}


\section{}

\section{}


\section{}

\end{document}
