\chapter{Appairage du module} \label{appairage}

\begin{messageBox}{Point-clé}{orange}{white}{Avant de créer l'application, il faut tout d'abord que le portable puisse reconnaître le module Bluetooth.\\
Pour cela, il faut «l'appairer», c'est à dire que l'on va définir que le module sera apte à recevoir les données envoyées par le portable. \\Cette étape est très rapide et ne sera effectuée qu’à chaque changement de module Bluetooth.}{white}
\end{messageBox}

1) Tout d'abord, branchez le module : la broche +5V du module est reliée à la broche 5V de la carte Arduino et la broche GND du module est reliée à la GND de la carte. \\


2) Démarrez le Bluetooth sur votre téléphone portable puis allez dans \bold{paramètres > Bluetooth} \\
Faites « \bold{rechercher} » afin de trouver un périphérique. 
Vous devriez trouver un périphérique « BT Crius ». Cliquez dessus et faites « \bold{associer} ». \\
A ce moment là, on vous demandera un mot de passe, qui par défaut, est \bold{0000} (ou \bold{1234}). \\

Lorsqu'il est entré, faites « \bold{valider} » et au bout de quelques secondes, le clignotement du module devrait se stopper pour qu'il n'y ait qu'une lumière continue 



Et voila, le module est appairé.