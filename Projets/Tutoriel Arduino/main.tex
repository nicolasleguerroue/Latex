\documentclass[12pt]{report}
\usepackage[utf8]{inputenc}
\usepackage{hyperref}
\usepackage{color}
\usepackage{eurosym}
\usepackage{graphicx}
\usepackage[francais]{babel}

\hypersetup{
	colorlinks=true,
	linkcolor=blue,
	filecolor=magenta,
	urlcolor=blue,
}
\urlstyle{same}



\begin{document}

\title{Présentation Arduino}

\author{Nicolas LE GUERROUÉ}  

\begin{center}
\includegraphics{arduino.png}
\end{center}

\maketitle
\tableofcontents

%################################################################
%################ CHAPITRE 1 ####################################
%################################################################

\chapter{Introduction}



Ce docment vise à présenter le projet Arduino et ses supports dans le cadre du cours de Construction Electronique. \newline
Ce tutoriel a pour but également de présenter certaines possibilités d'Arduino en terme de langage et de ressources. \newline
Bien évidemment, cette section n'est pas du tout exaustive.

\section{origines}

Arduino est née en 2004 sous l'impulsion d'étudiants italiens souhaitant promouvoir l'accès à  l'électronique. Ils se rencontraient fréquement dans un bar (le ...) pour développer leur projet. \newline
Aujourd'hui, \underline{Arduino} c'est :
\begin{enumerate}
\item Un langage de programmation basé sur le C++
\item Une communauté
\item Un projet Open-Source
\end{enumerate}

\section{Supports}

Arduino disposant d'une communauté assez vaste, de nombreux supports existent. \newline
Nous avons notamment le site officiel d'Arduino à l'adresse suivante :

\underline{arduino.cc/Reference/en} \newline

Le langage Arduino est compatible avec les instructions du C++ dans la mesure ou le compilateur pour Arduino est g++.
Ainsi, les types compososés comme les structures et classes sont supportés, tout comme le mot clé auto par exemple.





\chapter{Présentation}

Dans le cadre du cours de Construction Electronique, nous utilisons des cartes Arduino Uno, basées sur les microcontroleurs Atmega-328 du fabricant ATMEL. \newline
Les microcontroleurs sont des unités contenant dans un seul boitier une mémoire, un processeur et des interfaces entrées-sorties pour ne citer que ces élements.
 \newline
 Cela permet notamment de dialoguer avec des périphériques \footnote{Voire section Protocoles de communication}

\section{Le microcontroleur}


\subsection {Alimentation}

\subsubsection {Tension d'alimentation}

Le microcontroleur doit être alimenté entre 1.8 V et 5.5 V. \newline
Il existe deux façon d'alimenter la carte Arduino : 

\begin{enumerate}
\item Via le port USB 
Le port USB délivre du 5V régulé avec un courant maximal de 500 mA (cas général)
\item Via la broche Vin (connectique Jack femelle)
La carte Arduino possède un régulateur intégré de tension en 5 V, ce qui permet d'alimenter la carte entre 7V et 20V

\end{enumerate}

\subsubsection {Courants d'entrées-sortie}

\subsection {Fréquence d'horloge}

La carte Arduino comporte un oscillateur de 16 MHz même si en interne du microcontroleur, un oscillateur de 8 Mhz est intégré.
Cela donne une idée des performances maximales de l'Arduino.



\subsection {mémoire}

Ce microcontroleur dispose de clé \textbf{32 ko} de clé \textbf{mémoire flash}, c'est à dire la mémoire pour stocker le programme téléversé vers la carte. \newline


Quand à la clé \textbf{mémoire vive (SRAM)}, elle est de \textbf{2 ko} et est utilisée pour les variables du programme en cours d'éxécution. \newline
Cette mémoire peut être donc vite saturée lors de l'utilisation de grands tableaux par exemple. \newline


Enfin, le  possède une mémoire effacable electriquement, appelée \textbf{EEPROM}\footnote{Référence : arduino.cc/Reference/EEPROM}, lors de l'éxécution du programme. \newline
Cette mémoire occupe \textbf{1 ko} et chaque registre de cette mémoire, pouvant stocker un nombre codé sur 8 bits (type byte ou char), peut être modifiée 100 000 fois avant son arrêt définitif.\newline


\section{Caractéristiques électriques}

Le microcontrolleur dispose d'entrée sorties permettant d'intéragir avec des périphériques (Diodes electroluminescentes, capteurs, modules de communication...).
Les entrées sont deux types : 

\begin{enumerate}
\item clé \textbf{numérique} : la valeur lue sera perçue comme un niveau logique 0 ou 1 sur les broches allant de 1 à 13
\item clé \textbf{analogique} : Un Convertisseur Analogique-Numérique 10 bits est intégrés sur les broches A0, A1, A2, A3, A4 et A5
De ce fait, le CAN est possède une résolution de 4.84 mV avec une référence de tension à 5V \footnote{Il est également possible de changer la tension de référence de la carte Arduino 
arduino.cc/Reference/en/language/function/analog-io/analogreference}
\item Un projet Open-Source
\end{enumerate}

Astuce : Il est possible de configurer les broches analogique en broches digitales. 

%################################################################
%################ CHAPITRE 3 ####################################
%################################################################

\chapter{Le langage}

\section {Les types}


Par défaut les types sont signés, c'est à dire que la plage de valeur pour un nombre codé sur \emph{n} bits est compris entre $ \frac{-2^n}{2} $ et $ \frac{2^n+1}{2} $ 
Pour définir un type non signé, c'est à dire pour agrandir la plage positive, il suffit d'ajouter le mot clé \textbf{unsigned} avant les types concernés.

\subsubsection {Les types supportés}

\begin{enumerate}
\item \textbf{byte} \textit{[1 octet]} \newline
Désigne la plus petite unité de mémoire allouable, permettant de stocker un nombre entier compris entre -127 et + 127.
\item \textbf{unsigned byte} \textit{[1 octet]} \newline
Idem mais la plage de valeur strictement positive

\item \textbf{int}    \textit{[2 octets]} \newline
Permet de stocker un nombre entier compris entre -32536 et +32536
\item \textbf{unsigned int} \textit{[2 octet]} \newline
Idem mais la plage de valeur strictement positive

\item \textbf{float}  \textit{[4 octets]} \newline
Permet de manipuler des réels
\item \textbf{double} \textit{[4 octets]} \newline
Idem, mais ce type est plus précis que le type float et demande plus de ressources au microcontroleur. 

\item \textbf{char} \textit{[1 octets]} \newline
Ce type est utilisé pour stcoker et traiter des caractères de la table ASCII.
\item \textbf{String} \textit{[4 octets]} \newline
String est un type élaboré qui permet de traiter des chaines de caractères.
\end{enumerate}

\subsubsection {Les types non supportés}

Les types \textbf{vector}, \textbf{array} et \textbf{tuple} ne sont pas supportés par le langage Arduino.




\section {Les fonctions}

\subsection{Les fonctions mathématiques}

En ce qui concerne les fonctions mathématiques, les fonctions trogonométriques sont incluses 


\begin{tabular}{ l c r }
   1 & 2 & 3 \\
   4 & 5 & 6 \\
   7 & 8 & 9 \\
 \end{tabular}




\chapter {Les broches d'interruption}

Dans certains cas, il est souhaitable de récuperer la valeur d'une broche à tout moment du programme, même quand celui ci est occupé dans une tâche et même dans une fonction de temporisation \footnote{Voir delay(), delayMicroseconds()}. \newline
Prenont le cas d'un bouton de comptage : on souhaite acquérir toutes les mesures afin de ne pas manquer d'appui de la part de l'utilisateur.\newline
Pour remédier à ce problème, on peut utiliser les \textcolor{blue} {\textbf{broches d'interruption}} qui permettent de récuperer la main sur l'ensemble du programme lorsqe'un évènement survient sur une broche.\newline
Concrètement, lorsque un évènement  \emph{e} survient sur la broche \emph{b}, la fonction \emph{f} est appelée, quelque soit l'état du programme principal. \newline

\begin{verbatim}

//Déclare un bouton sur la broche 2 de l'Arduino
#define PB 2  						
//Déclare une variable pouvant être modifiée par une fonction d'interruption
volatile int nombre_appui = 0;  	

void setup() {
		attachInterrupt(, RISING, add());
}

void loop() {
	//...
	//code loop
	//...
}

void add() {
	
	nombre_appui++;
}

\end{verbatim}
Ici, quelque soit l'action effectuée dans la fonction loop, dès qu'un front montant est détecté sur la broche BP (2), la fonction add() sera exécuté et incrémentera la variable de comptage.







\chapter{Synthèse}

\section{Caractéristiques}

\subsection {Matériel}

\begin{enumerate}
\item Mémoire Flash : 32 ko
\item Mémoire Vive 'SRAM) : 2 ko 
\item Fréquence d'horloge : 16 MHz 
\end{enumerate}



\subsection {Électriques}

\begin{enumerate}
\item Impédance d'entrée : > 1MOhm
\item Courant de sortie par broche : 40 mA maximum
\item Courant de sortie pour toutes les broches entrée-sorties: 200 mA
\end{enumerate}


\begin{center}
   \begin{tabular}{ | c | c | c | c | }
     \hline
     Fonction & description & Argument & Valeur de retour \\ \hline
     map & 
     Permet de convertir une valeur d'un interval à un autre &  
     value [int] : valeur de l'interval\\ \hline
     7 & 8 & 9 \\ \hline
     \hline
   \end{tabular}
 \end{center}

\begin{thebibliography}{9}

\bibitem{}
	  \emph{Documentation de référence Arduino}.

	  \url{https://www.arduino.cc/reference/en}

\bibitem{}
	  \emph{Siet Locoduino}.

	  \url{https://pymodbus.readthedocs.io/en/latest/}

\bibitem{}
	  \emph{Ressource ENIB}.

	  \url{https://enib.fr/~mecatro/}


\end{thebibliography}


%$$ log{10*e{5*x}} $$

\end{document}
