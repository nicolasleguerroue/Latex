 

\section{Introduction}


Il existe deux grandes familles d'alimentations : 

\begin{items}{orange}{\Triangle}
    \item Les alimentations linéaires
    \item Les alimentations à découpages
\end{items}

Les alimentations linaires sont réputées pour avoir une tension stable et propre (tension continue).\\
Regardons de plus près le principe de ces alimentations.

\section{L'objectif}

On souhaite transformer une tension sinusoïdale\footnote{De valeur efficace 230V et de fréquence 50 Hz} en une tension continue, stable et régulée.

\vfill
\begin{graphicFigure}{0.8}{1}{0}{0.08}{-230}{230}{t(s)}{Tension (V)}{Les tensions d'entrée et de sortie de notre alimentation}
  \addTrace{blue}{0}{0.08}{230*sin(50*360*\x)}
  \addTrace{orange}{0}{0.08}{12}
  \addTrace{black}{0}{0.08}{0}
  \addLegend{Entrée, Sortie, 0V}
  \end{graphicFigure}

  \newpage
  \section{Le choix du transformateur}

  Pour notre circuit, nous devons connaître la tension finale en sortie de notre alimentation.\\
  Au vue des composants demandés (régulateurs, pont de diode), il faudra une tension efficace de $12V$\footnote{Nous verrons la justification de cette valeur par la suite.}
  Le rapport des bobines de notre transformateur sera donc au minimum de \bold{19}. \\Cela veut dire que pour 19 enroulements dans la bobine primaire, il en faudra une seule dans la bobine secondaire.

  La puissance demandée pour le transformateur n'est pas critique, la puissance absorbée par le circuit en aval est négligeable.

  Notre transformateur va donc abaisser la tension du secteur en une tension plus faible.

  \begin{graphicFigure}{0.6}{0.6}{0}{0.08}{-230}{230}{t(s)}{Tension (V)}{Les tensions d'entrée et de sortie du transformateur}
    \addTrace{blue}{0}{0.08}{230*sin(50*360*\x)}
    \addTrace{red}{0}{0.08}{14*sin(50*360*\x)}
    \addTrace{black}{0}{0.08}{0}
    \addLegend{Entrée, Sortie, 0V}
    \end{graphicFigure}

    \newpage 
  \section{Le pont redresseur de tension}

  \subsection{Principe}
  Notre objectif est maintenant de supprimer la composante négative de notre signal. Pour cela, on va utiliser la propriété des diodes qui ont la faculté de laisser passer le courant dans un seul sens.\\
  Un montage existant est le Pont de Graetz.

  \img{\rootImages/diode.png}{Le pont de diode}{0.5}

  Observons le résultat lorsque nous mettons notre tension de sortie du transformateur sur notre entrée de pont.

  \begin{graphicFigure}{0.6}{0.6}{0}{0.08}{-20}{20}{t(s)}{Tension (V)}{Les tensions d'entrée et de sortie du pont de diode}
    \addTrace{blue}{0}{0.08}{12*sin(50*360*\x)}
    \addTrace{red}{0}{0.08}{abs(15.64*sin(50*360*\x))}
    \addTrace{black}{0}{0.08}{0}
    \addLegend{Entrée, Sortie, 0V}
    \end{graphicFigure}

  On constate que la tension de sortie est plus élevée que notre tension d'entrée.\\
  Cette valeur s'explique par les diodes. 
  
  En sortie, nous voulons une tension de $15V$. Or, les $12V$ en entrée de notre pont sont une valeur efficace, ce qui veut dire que la valeur réelle vaut $17.04V$
    En effet : 
  $$V_{réelle} = V_{efficace}\cdot \sqrt{2} = 12\cdot 1.41 = 17.04 V$$

  Or, à chaque changement de polarité dans notre pont (signal alternatif [positif et négatif]), le courant passe à travers deux diodes.\\

  Chaque diode engendre une chute de tension de $0.7V$, d'où une perte de $1.4V$ en sortie.

  Nous avons comme tension finale en sortie de pont: 

  $$V_{s} = V_{efficace}\cdot \sqrt{2} - 0.7\cdot 2 = 12\cdot 1.41 -1.4 = 15.64 V$$


  \subsection{Un exemple de boîtier}

    La plupart des boîtiers possèdent 4 broches

    \begin{items}{blue}{\Triangle}
      \item deux broches $V_{~}$ pour la tension alternative
      \item Une broche $V_-$ pour la masse du circuit en aval
      \item Une broche $V_+$ pour la tension positive du circuit en aval
   \end{items}

  \img{\rootImages/case.png}{Un exemple de boîtier}{0.2}


  \section{Les condensateurs}

  \subsection{Principe}
  Comme vu à la section précédente, notre signal n'est pas continu et possède des oscillations élevées.
  Nous cherchons à produire une tension continue.\\

  Pour cela, nous allons utiliser des condensateurs.\\

  Leur rôle sera d'accumuler de l'énergie lorsque le signal monte en amplitude et de restituer cette énergie lorsque le signal sera sur sa phase descendante.

  \img{\rootImages/signal.png}{Le rôle du condensateur}{0.6}

  \subsection{Choix de la valeur}
  Plus la valeur du condensateur est élevée, plus notre signal sera lisse en sortie, ce qui est l'effet recherché.\\
  En effet, la capacité du condensateur exprime la quantité d'énergie emmagasinée dans le condensateur.\\

  Un condensateur de $1mF$ et plus est une valeur normale.\\
   Une fois notre tension assez propre (même si elle présente des oscillations), nous allons utiliser les régulateurs de tensions.

  \section{Les Régulateurs de tension}

  \subsection{Principe et branchements}
  Ces composants à trois broches servent à stabiliser une tension pseudo-continue en une tension continue et stable, c'est à dire ne présentant pas d'oscillations.\\


  Le branchement est le suivant : \\

  \img{\rootImages/lm.png}{Un régulateur de tension}{0.8}

  \begin{items}{blue}{\Triangle}
    \item L'entrée INPUT est la tension à réguler
    \item La broche GROUND est la masse du circuit
    \item la sortie OUTPUT est la tension regulée (stable)
\end{items}

\subsection{Les familles des régulateurs}

  Il existe différentes familles de régulateurs linaires, les plus connus étant les LM78XX\footnote{XX prend la tension régulée de sortie (5,9,12,15,18V...)} ou les LM317 qui sont des régulateurs de tension réglables.\\
  Pour des tensions négatives, se référer à la famille des LM79XX.\\

  Pour que ces composants fonctionnent correctement, il convient de leur mettre en entrée une tension supérieure à leur tension de régulation.\\

  Concrètement, si on utilise un LM7812 (régulateur 12 V), il faudra mettre en entrée au moins 15 V pour que le composant fonctionne dans la plage idéale.\\

  \bold{Cette tension de $15V$ justifie la valeur de tension du transformateur.}

  Ces composants sont linéaires, c'est à dire que la différence de tension entre l'entrée et la sortie ($V_{e}-V_s$) engendre une perte de puissance dans le système.\\
  Cette perte dépend directement du courant absorbé par le circuit en aval mais également de la différence de tension $(V_e-V_s)$.\\
  Cette puissance vaut : $$ P_{perdue} =(V_e-V_s)\cdot I_{charge} ~~~~(W)$$

  \subsection{Un exemple}

  On souhaite réguler un circuit en 12V, ce dernier consommant 150 mA ($I_{charge}$).

  La perte d'énergie vaut $$ P_{perdue} =(V_e-V_s)\cdot I_{charge} =(15-12)\cdot 0.150 = 0.45 W$$

  % \subsection{la résistance thermique des régulateurs}

  % En fontion de cette puissance perdue, il est possible de calculer l'élevation de la température du régulateur de tension.

  % La documentation technique des régulateurs de tension précise la valeur de la résistance thermique, c'est à dire de l'élévation de température lorsque le composant dissipe 1 Watt en pure chaleur.


  \appendix

  \section{Quelques fournisseurs de composants et matériels}

  Voici ma liste des fournisseurs.

  \begin{items}{blue}{\Circle}
    \item \bold{Reichelt} \\ Cet site possède un choix très élevé de circuits intégrés (Amplificateurs opérationnels, transistors...) mais également de diodes, de leds, résistances.\\
    La documentation est bien fournie.\\
    Vente d'outils pour l'électronique.\\
    Le site est disponible à l'adresse \url{https://www.reichelt.com/}

    \item \bold{Gotronic} \\ Cet site possède un choix très élevé de capteurs.\\
    La documentation est bien fournie.\\
    Vente d'outils pour l'électronique.\\
    Délais de livraison rapides.\\
    Le site est disponible à l'adresse \url{https://www.gotronic.fr/}

    \item \bold{Conrad} \\ Similaire à Gotronic.\\
    La documentation est bien fournie.\\
    Vente d'outils pour l'électronique.\\
    Le site est disponible à l'adresse \url{https://www.conrad.fr/}

    \item \bold{Semageek} \\ Similaire à Gotronic.\\
    Vente d'outils pour l'électronique.\\
    Le site est disponible à l'adresse \url{https://boutique.semageek.com/fr/}


    \item \bold{Dfrobot} \\ Un large choix de capteur pour l'embarqué.\\
    La documentation est bien fournie, cependant certains prix sont parfois un peu excessifs.\\
    Le site est disponible à l'adresse \url{https://www.dfrobot.com/}

    \item Puis \bold{Banggoods} pour les petites bricoles pas chères... \\
    Le site est disponible à l'adresse \url{https://www.banggood.com/}\\
    Délai de livraison au maximum de 3 semaines.


\end{items}

\end{document}