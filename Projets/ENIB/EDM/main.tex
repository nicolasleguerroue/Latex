\chapter{Bibliothèque Header}

\section{Mise en forme de la page de garde avec une image}

\lgreen{LOC}{Header}
\begin{Latex}{Code pour la mise en forme de la page de garde avec une image}
\setHeaderImage{Emplacement_image}{0.8}{Titre}{sous-titre}{Auteurs}{\today \\ \pageref{LastPage} pages}
\end{Latex}


\section{Mise en forme de la page de garde sans image}

\lgreen{LOC}{Header}
\begin{Latex}{Code pour la mise en forme de la page de garde sans image}
  \setHeader{Titre}{Auteur 1 \\ Auteur 2}{Date}
\end{Latex}

\section{Mise en forme de la page des parties}

\lgreen{LOC}{Body}
\begin{Latex}{Code pour la mise en forme de la page des parties}
  \partImg{Partie}{Images/file.png}{0.2}
\end{Latex}

\section{Ajout d'un trait entre l'en-tête et le corps de la page}

\lgreen{LOC}{Header}
\begin{Latex}{Code pour l'ajout d'un trait entre l'en-tête et le corps de la page}
  \setHeaderLine{0.2}
\end{Latex}

\section{Ajout d'un trait entre le corps de la page et le bas de page}

\lgreen{LOC}{Header}
\begin{Latex}{Code pour l'ajout d'un trait entre le corps de la page et le bas de page}
  \setFooterLine{0.2}
\end{Latex}





\section{Définition de la présentation globale des pages}

\lgreen{LOC}{Header}
\begin{Latex}{Code pour la définition de la présentation globale des page}
  \addPresentation
  {Titre} {Centre} {\currentChapter}
  {Gauche} {} {\currentPage}
\end{Latex}

\section{Redéfinition des titres des chapitres}

%\setAliasChapter{Section}
\lgreen{LOC}{Header}
\begin{Latex}{Code pour la redéfinition des titres des chapitres}
  \setAliasChapter{Section}
\end{Latex}


\section{Mettre le document en pleine page}
\lgreen{LOC}{Header}
\begin{Latex}{Code pour mettre le document en pleine page}
\setFullPage
\end{Latex}

\section{Récuperer le chapitre courant}
\lgreen{LOC}{Body}
\begin{Latex}{Code pour récuperer le chapitre courant}
\currentChapter
\end{Latex}


%%%%%%%%%%%%%%%%%%%%%%%%%%%%%%%%%%%%%%%%%%%%%%%%%%%%%

%dossier des images
\newcommand{\imRoot}{Images/}

\newcommand{\evec}[3]{\left (\begin{array}{ccc} #1 \\ #2 \\ #3\end{array} \right )}
%Vecteur à 3 composantes

\newcommand{\emat}[3]{\left [\begin{array}{ccc} #1 \\ #2 \\ #3\end{array} \right ]}
%Matrice à 3 composantes
% \emat{a & b & c}{d & e & f}{g & h & i}

%%%%%%%%%%%%%%%%%%%%%%%%%%%%%%%%%%%%%%%%%%%%%%%%%%%%%
\title{Projet EDM S4P }
\author{Antoine SCAVINER - Théo MAINGUENÉ - Mathieu CHARLES - Nicolas LE GUERROUÉ}
\date{Année 2019-2020}
\usepackage{titlesec}
\titlespacing*{\section}
{0}{0}{0}
\begin{document}

\makeatletter
\begin{titlepage}
\begin{tikzpicture}[overlay,remember picture]
    \coordinate (SW) at (current page.south west);
    \coordinate (SE) at (current page.south east);
    \coordinate (NW) at (current page.north west);
    \coordinate (NE) at (current page.north east);
    \node at ($(NW)!0.5!(SE)$)[anchor=north,yshift=6cm,align=center]
        {\huge \@title};
    \node at ($(NW)!0.5!(SE)$)[anchor=north,yshift=4cm,align=center]
        {\Large Notice de calcul};
    \node at ($(NW)!0.5!(SE)$) [anchor=north,yshift=2cm,align=center]
        {\Large \@date};
    \node at ($(NW)!0.5!(SE)$) [anchor=north,yshift=0cm,align=center]
        {\large {\@author}};
    \node at ($(NE)!0.92!(SW)$)[anchor=south,xshift=14cm,align=right]{\includegraphics[width=0.3\linewidth]{./img/ENIB.png}};
    \node at ($(NW)$)[align=left,xshift=10.77cm,yshift=-0.3cm,]{\includegraphics[width=1.35\linewidth]{./img/cover.png}};
\end{tikzpicture}
\end{titlepage}



\tableofcontents
	
\titlespacing{\chapter}{0pt}{*-5}{*5}
\chapter{Cahier des charges}

\section{Contexte}
On se propose de dimensionner un monte-charge permettant de soulever un poids de 10 kilogrammes.\newline
Ce dernier doit pouvoir déplacer la charge de 2.6m  en un maximum de 12 secondes.\newline
Le diamètre du bout d'enroulement est de 8mm. \newline
L'ensemble sera fixé sur une plaque de 200x280mm et de 5mm d'épaisseur.\newline
L'enroulement sera sur 8 tours +/-1.\newline
Le moteur devra être alimenté en 24V.\newline
Le système sera auto-bloquant en cas d'absence d'énergie d'alimentation.\newline

\endgroup% toutes les modifications vont disparaitre après


%\chapter{Calculs des puissances}
\chapter{Calculs des puissances}


\section{Paramétrisation}

Diamètre du bout : 8mm \newline
%Diamètre du tambour : 52 mm \newline
%Largeur du tambour : 64 mm \newline
Vitesse linéaire de charge minimale : 0.22 m/s \newline
Masse de la charge : 10 kg (15 kg avec le coefficient de sécurité appliqué) \newline
Coefficient de sécurité : \textbf{1.5} \newline
Accélération de la pesanteur (g) : 10 $m/s^2$
\sskip


\section{Déterminations des contraintes}


\subsection{Calcul du nombre d'enroulement}

La largeur du tambour correspond au nombre d'enroulements multiplié par le diamètre du bout.

$$Largeur_{tambour}=enroulements\cdot diamètre_{bout}= 8 * 8 =64 mm$$

\subsection{Calcul du rayon minimum}

Afin de déterminer le rayon minimum on divise la distance totale parle nombre d'enroulement, ce qui nous donne le périmètre \newline
$$Perimetre=\frac{Distance}{Enroulement}$$ 

On isole alors le rayon

$$R_{minimum}= \frac{Perimetre}{2 \pi} = \frac{0.325}{2 \pi}=52mm$$

Il y aura donc au total 1 enroulement de câble autour du tambour. \newline

\subsection{Calcul du rayon maximal}

Le rayon maximal correspond ici au rayon où sera exercé l'action de la charge. \newline

$$Rayon_{max} = Rayon_{tambour} + (Enroulement \cdot \phi_{câble}) - \frac{\phi_{câble}}{2} $$ \newline \newline (toutes les distances en mm) \newline

\iImage{Images/tambour.png}{Point d'application de la charge}{0.5}

$$Rayon_{max} = 52 + (1\cdot 8) - \frac{8}{2} = 56 mm = 0.056m$$
\n

\textbf{Le rayon maximal sera de 56 mm
}

\subsection{Calcul du couple de sortie}

Le couple de sortie correspond au produit du poids par la distance orthogonal à l'axe de rotation

$ Couple_{sortie} = R_{max} \cdot m_{charge} \cdot g$ \newline
(Unités de longueur en m) \newline

$ Couple_{sortie} =  0.056 \cdot 15 \cdot 10  $ \newline
 
D'où \newline

$ Couple_{sortie} =  8.4 Nm $\n

\textbf{Le couple développé sur l'arbre de sortie devra être supérieur à 8.4 N.m.}


\chapter{Rapport de réduction}

\section{Référence du moteur}
On choisira par conséquent de prendre le moteur suivant :
\n\n
Moteur \bold{EC050.24E ( S2 30' )} \newline
Lien d'achat : \href{https://www.technoindus.com/moteur-frein-a-courant-continu/moteur-a-courant-continu-ferrite-frein-350w-3000t-mn-24v-o11-bride-o90-ip40-3665948288223-4641.html}{\bold{ICI}} \newline
Tension nominale ($U_{nom}$): 24 V CC. - Courant nominal ($I_{nom}$): 29.4 A\newline
Couple nominal ($C_{nom}$) : 1.57 N.m \newline
Vitesse nominale ($V_{nom}$): 3000 tr/min ($\Omega_{nom} = 314.15 rad/s$) \newline
Documentation : \href{https://www.technoindus.com/index.php?controller=attachment&id_attachment=138}{\bold{Page 18, cycle de vie S1}} \n\n
On se propose de vérifier que le choix de ce moteur est cohérent avec le cahier des charges. 


\subsection{Calcul du rapport de réduction}


$ r_{réduction} = \frac{Couple_{entrée}}{Couple_{sortie}}$ \newline

D'ou : \newline

$ r_{réduction} = \frac{1.57}{8.4} = 0.186$

Il faudra donc que le rapport de réduction soit inférieur à 0.186

Nous choisirons le jeu suivant 17/114 qui donne un rapport de 0.149

\sskip
\section{Vérification du choix du jeu}
\subsection{Vérification du couple de sortie}
On a un couple en sortie de moteur de 1.57 Nm.
Or: 1.57*(1/0.149) = 10.53 Nm\n
Cela se conforme avec notre couple de sortie minimal de 8.4Nm.

$ Vitesse_{charge} = \Omega \cdot R_{max}$


%Données Roulement & RDM
%Diamètres roulement : 26 ou 30 mm
%Diamètre arbre :  17 mm ou 20 mm


\subsection{Calcul du module minimal}



Nous avons choisi de partir sur des roues dentées de module 1.25 et possédant 17 et 114 dents.

Il nous a fallu vérifier que ce module était suffisant pour que les roues puissent supporter les efforts imposés par la charge. \newline
Il nous faut pour cela calculer l'effort tangentiel auquel les dents des roues sont soumises : $ F_T = \frac{2C_{moteur}}{m_{choisi}*Z_{min}*10^{-3}} = \frac{2*1.57}{1.25*17*10^{-3}} = 147.76 N$\newline
Ainsi, nous pouvons vérifier que notre module rempli bien la condition $$ m \geq 2.34*\sqrt{\frac{F_T}{k*R_{pe}}} $$
Ce qui nous donne : $ 2.34 * \sqrt{\frac{147.76}{8*250}} = 0.636 $ or, $ 1.25 \geq 0.636 $ donc ce choix est valide.


Les références des dents avec un module de 1.25 sont disponible à l'adresse \newline
\url{https://www.michaud-chailly.fr/fr/roue-cylindrique-droite-acier-module-1-25-largeur-denture-10mm-a1-30/}


\chapter{Calculs PFS}



\section{Bilan des actions extérieures}

On isole le tambour. Ce système est appelé \{1\}
On considère le repère galiléen $R_0=(\v{x}, \v{y}, \v{z})$

\subsection{Action de l'engrenage au point I}

$\{T_{3 \rightarrow 1}\}_I = \{ {\v{R}_{3 \rightarrow 1} \v{M}(3 \rightarrow 1)}_I\}_I =
\left\{ \begin{array}{ccc}
0&0\\
Y_I&0\\
Z_I&0\\
\end{array}
\right\}_{I,R_0=(\v{x}, \v{y}, \v{z})}$


\subsection{Action du roulement au point A}
$\{T_{2 \rightarrow 1}\}_A = \{ {\v{R}_{2 \rightarrow 1} \v{M}(2 \rightarrow 1)}_A\}_A =
\left\{ \begin{array}{ccc}
0&0\\
Y_A&0\\
Z_A&0\\
\end{array}
\right\}_{A,R_0=(\v{x}, \v{y}, \v{z})}
$

\subsection{Action du roulement au point B}
$\{T_{2 \rightarrow 1}\}_B = \{ {\v{R}_{2 \rightarrow 1} \v{M}(2 \rightarrow 1)}_B\}_B =
\left\{ \begin{array}{ccc}
X_B&0\\
Y_B&0\\
Z_B&0\\
\end{array}
\right\}_{B,R_0(\v{x}, \v{y}, \v{z})}
$

\subsection{Action la charge au point P}
$\{T_{4 \rightarrow 1}\}_P = \{ {\v{R}_{4 \rightarrow 1} \v{M}(4 \rightarrow 1)}_P\}_P =
\left\{ \begin{array}{ccc}
0&0\\
-m\cdot g&M\\
0&0\\
\end{array}
\right\}_{P,R_0=(\v{x}, \v{y}, \v{z})}
$

\section{Principe Fondamental de la Statique}

Le système est à l'équilibre, on peut donc appliquer le PFS. \newline
Les moments seront calculés par rapport au point B car son torseur associé possède 3 inconnues.
\subsection{Énoncé du PFS}
$$\sum(\v{F}_{\bar 1 \rightarrow 1}) = 0$$
$$\sum\v{M}_B(\v{F}_{\bar 1 \rightarrow 1}) = 0$$
\textit{Nous obtenons donc les équations suivantes} \newline
$$ \left\{
\begin{array}{ccc}
\sum \v{F}_{\bar 1 \rightarrow 1}\cdot \v{x} & = 0 \\
\sum \v{F}_{\bar 1 \rightarrow 1}\cdot \v{y} & = 0 \\
\sum \v{F}_{\bar 1 \rightarrow 1}\cdot \v{z} & = 0
\end{array}
\right.
$$


$$ \left\{
\begin{array}{ccc}
\sum \v{M}_B(\v{F}_{\bar 1 \rightarrow 1})\cdot \v{x} & = 0 \\
\sum \v{M}_B(\v{F}_{\bar 1 \rightarrow 1})\cdot \v{y} & = 0 \\
\sum \v{M}_B(\v{F}_{\bar 1 \rightarrow 1})\cdot \v{z} & = 0 
\end{array}
\right.
$$



\subsection{Calculs des moments}


\subsection{Calcul au point I}
$$ 
\begin{array}{cc}
\v{M}_B(\v{T}_{3 \rightarrow 1}) = &  \v{M}_I + \v{BI} \wedge \v{R}_{3 \rightarrow 1} \\

\v{M}_B(\v{T}_{3 \rightarrow 1}) = &  \v{0} + 
                                \bigg(  
                               \begin{array}{ccc}
                                -L_B+L_I\\
                                -R_I\\
                                0
                                \end{array}
                                \bigg)  
                                \wedge 
                                 \bigg(  
                               \begin{array}{ccc}
                                0\\
                                Y_I\\
                                Z_I
                                \end{array}
                                \bigg)  \\


\v{M}_B(\v{T}_{3 \rightarrow 1}) = &  
                                \bigg(  
                               \begin{array}{ccc}
                                -R_I\cdot Z_I\\
                                (L_B-L_I)\cdot Z_I\\
                                -(L_B-L_I)\cdot Y_I
                                \end{array}
                                \bigg)  
                                
\end{array}
$$



\subsection{Calcul au point A}
$$ 
\begin{array}{cc}
\v{M}_B(\v{T}_{(2 \rightarrow 1)_A}) = &  \v{M}_A + \v{BA} \wedge \v{R}_{(2 \rightarrow 1)}_A \\

\v{M}_B(\v{T}_{(2 \rightarrow 1)_A}) = &  \v{0} + 
                                \bigg(  
                               \begin{array}{ccc}
                                -(L_B-L_A)\\
                                0\\
                                0
                                \end{array}
                                \bigg)  
                                \wedge 
                                 \bigg(  
                               \begin{array}{ccc}
                                0\\
                                Y_A\\
                                Z_A
                                \end{array}
                                \bigg)  \\


\v{M}_B(\v{T}_{(2 \rightarrow 1)_A}) = &  
                                \bigg(  
                               \begin{array}{ccc}
                               0\\
                               (L_B-L_A) \cdot Z_A\\
                                -(L_B-L_A) \cdot Y_A
                                \end{array}
                                \bigg)  
                                
\end{array}
$$


\subsection{Calcul au point B}
$$ 
\begin{array}{cc}
\v{M}_B(\v{T}_{(2 \rightarrow 1)_B}) = &  \v{M}_B + \v{BB} \wedge \v{R}_{(2 \rightarrow 1)}_B = \v{0}
\end{array}

$$



\subsection{Calcul au point P}
$$ 
\begin{array}{cc}
\v{M}_B(\v{T}_{(4 \rightarrow 1)}) = &  \v{M}_P + \v{BP} \wedge \v{R}_{(4 \rightarrow 1)} \\

\v{M}_B(\v{T}_{(4 \rightarrow 1)}) = &  \v{0} + 
                                \bigg(  
                               \begin{array}{ccc}
                                -(L_B-L_A)\\
                                0\\
                                R_d
                                \end{array}
                                \bigg)  
                                \wedge 
                                 \bigg(  
                               \begin{array}{ccc}
                                0\\
                                -m\cdot g\\
                                0
                                \end{array}
                                \bigg)  \\


\v{M}_B(\v{T}_{(4 \rightarrow 1)}) = &  
                                \bigg(  
                               \begin{array}{ccc}
                               R_d \cdot m \cdot g\\
                               0\\
                                (L_B-L_A)\cdot m \cdot g
                                \end{array}
                                \bigg)  
                                
\end{array}
$$

\newpage
\subsection{Calculs des Résultantes}

\textit{Avec les résultantes, nous avons les 3 équations suivantes } 

$$ \left\{
\begin{array}{ccc}
     0 + X_B + 0 & = 0\\
     Y_A + Y_B + Y_I -m\cdot g & = 0 \\
     Z_A + Z_B + Z_I  & = 0
\end{array}
\right.
$$

\textit{Avec les calculs des moments, nous en déduisons les 3 équations suivantes }

$$ \left\{
\begin{array}{ccc}
0 - R_I\cdot Z_I + R_d\cdot m\cdot g & = 0 \\
(L_B-L_A)\cdot Z_A + (L_B-L_I)\dcot Z_I & = 0 \\
-(L_B-L_A)\cdot Y_A -(L_B-L_I)\cdot Y_I +(L_B-L_A)\cdot m \cdot g & = 0 
\end{array}
\right.
$$

Donc 

$$ \left\{
\begin{array}{ccc}
Z_I = &  \frac{R_d\cdot m \cdot g}{R_I} \\
Z_A =  & \frac{-(L_B-L_I)\cdot Z_I}{(L_B-L_A)} \\
Z_B =  & \frac{-R_d\cdot m \cdot g}{R_I} + \frac{(L_B-L_I)\cdot Z_I}{(L_B-L_A)} \\
Y_A =  & (m\cdot g) - \frac{(L_B-L_I)\cdot Y_I}{(L_B-L_A)}  \\
Y_B =  & \frac{(L_B-L_I)\cdot Y_I}{(L_B-L_A)} - Y_I \\
\end{array}
\right.
$$

\iImage{img/pfs_excel.png}{Tableau de calculs}{0.6}

\chapter{Dimensionnement des roulements}

Dans un premier temps on regarde le rapport   $\frac{F_{axiale}}{F_{radiale}}$ \n

Or ce rapport est égal a 0 pour tous les roulements car la force axiale est nul dans chacun des cas. \node

Or comme ce rapport est inférieur à $e_{min} = 0.19 $\n
On obtient : \n 

$C = P \cdot (\frac{L_h \cdot 60 \cdot n}{10^6})^{\frac{1}{k}}$ \n

avec $ P = F_R $ la charge dynamique équivalente [N] \n

On calcul ensuite avec \n 
 n : la vitesse de rotation en tr/min du tambour \n
 $n = N_{mot} \cdot r = 3000 \cdot 0.185 = 558 tr/min $ \node
 
$k = 3$ car il s’agit d’un roulement avec des billes \n


$F_{R_A} = \sqrt{ {Y_A}^2 + {Z_A}^2 } = \sqrt{ {51}^2 + {22.8}^2 } = 55.9 N $ \n

$F_{R_B} = \sqrt{ {Y_B}^2 + {Z_B}^2 } = \sqrt{ {-147.2}^2 + {0}^2 } = 147.2 N $ \n

D'ou \n
(Les roulements possèdent la même vitesse de rotation) \n 

$Lh_A = 10000000  $: la durée de vie de la liaison en heures au roulement A \n\n

$C_{roulement\ A} = P_A \cdot (\frac{L_h \cdot 60 \cdot n}{10^6})^{\frac{1}{k}} = 55.9 \cdot (\frac{10000000 \cdot 60 \cdot 558}{10^6})^{\frac{1}{3}} = 3881 N \n

$Lh_B = 1000000  $: la durée de vie de la liaison en heures au roulement B \n\n 

$C_{roulement\ A} = P_A \cdot (\frac{L_h \cdot 60 \cdot n}{10^6})^{\frac{1}{k}} = 147.2 \cdot (\frac{1000000 \cdot 60 \cdot 558}{10^6})^{\frac{1}{3}} = 4744 N \n

On obtient donc : \n

Pour le roulement en A, C = 3881 N \n
Pour le roulement en B, C = 4744 N \n

Nous allons prendre des roulements identiques avec une charge dynamique supérieure à 4744 N. \n

Le roulement choisi sera un roulement avec un $C_{dynamique}$ de 4750N. \n 
La référence est la suivante : 6000-2-RSL - Diamètre intérieur :26 mm \n 
Fournisseur : SKF\n (\url{https://www.skf.com/binary/57-121486/0901d196803382dc-Roulements---10000_2.pdf}) \n

Achat :\n \url{https://www.123roulement.com/roulement-6000-2RS-SKF.php?gclid=CjwKCAjw8J32BRBCEiwApQEKgd79bcpisw-5aedWfZZesB3fOOI8trrD2N1c2Y_O8lwlZChrIWU0ARoCaKYQAvD_BwE&gclsrc=aw.ds}



\chapter{Dimensionnement de l'arbre}

Nous isolons la partie de droite de l'arbre pour effectuer les calculs de résistance des matériaux sur l'arbre.

\iImage{\imRoot Arbre_EDM}{Schéma de l'arbre utilisé pour les calculs}{0.4}
\newpage
\section{Bilan des actions mécaniques}

\begin{enumerate}
    \item Action de la partie gauche sur la partie de droite : 
    $\{\Gamma_{Coh_P}\} =
    \left\{ \begin{array}{ccc}
    N  &  M_t \\
    T_y & M_{fy}\\
    T_Z & M_{fz} \\
    \end{array} \right\}_{\v{x},\v{y},\v{z}}
    $

    \item Action du roulement en A : 
    $\{\Gamma_{A}\} =
    \left\{ \begin{array}{ccc}
    0  &  0 \\
    Y_A & 0 \\
    Z_A & 0 \\
    \end{array} \right\}_{\v{x},\v{y},\v{z}}
    $
    
    \item Action du roulement en B : 
    $\{\Gamma_{B}\} =
    \left\{ \begin{array}{ccc}
    X_B  &  0 \\
    Y_B & 0 \\
    Z_B & 0 \\
    \end{array} \right\}_{\v{x},\v{y},\v{z}}
    $
\end{enumerate}

\n 
La partie de droite est en équilibre par rapport au repère galiléen $R_0$, on peut donc appliquer le PFS :\n
$\Gamma_{5/D} = \Gamma_{coh} + \Gamma_A + \Gamma_B$

\subsection{Transposition des torseurs au point \bold{P}}

\subsubsection{Transposition du torseur $\Gamma_A$}

$
\v{M}(P,\Gamma_A) = \v{M}(A,\Gamma_A)+\v{PA}\wedge \evec{0}{Y_A}{Z_A} = \evec{-x + \frac{Leng}{2}}{0}{0} \wedge \evec{0}{Y_A}{Z_A} \newline \v{M}(P,\Gamma_A) = \evec{0}{-(-x+\frac{Leng}{2})Z_A}{(-x+\frac{Leng}{2})Y_A} $

\subsubsection{Transposition du torseur $\Gamma_B$}

$
\v{M}(P,\Gamma_B) = \v{M}(B,\Gamma_B) + \v{PB} \wedge \evec{X_B}{Y_B}{Z_B} = \evec{Leng + 64 - x}{0}{0} \wedge \evec{X_B}{Y_B}{Z_B} \n
\v{M}(P,\Gamma_B) = \evec{0}{(Leng + 64 - x)(-Z_B)}{(Leng + 64 - x)Y_B}
$\n\n 


Le morceau de droite est en équiliber par rapport au repere galiléen $R_0$, on peut donc appliquer le PFS




$\Gamma_{\overline{D}/D} = \Gamma_{coh} + \Gamma_A +\Gamma_B $
\section{Calcul du Torseur de cohésion}

$$ \v{R}_{(\overline{D}/D)} \left\{
\begin{array}{ccc}
\v{R}_{(\overline{D}/D)} . \v{x} =  & N + X_B &=  0 \\
\v{R}_{(\overline{D}/D)} . \v{y} =  & T_y + Y_A + Y_B & = 0 \\
\v{R}_{(\overline{D}/D)} . \v{z} =  & T_z + Z_A + Z_B & = 0\\
\end{array}
\right.
$$
\newline
$$ \v{M}_{(S,\overline{D}/D)}  \left\{
\begin{array}{ccc}
\v{M}_{(S,\overline{D}/D)} . \v{x} =  & M_T &=  0 \\
\v{M}_{(S,\overline{D}/D)} . \v{y} =  & M_{fy} - (-x+\frac{Leng}{2})  Z_A -(Leng + 64 - x) Z_B & = 0 \\
\v{M}_{(S,\overline{D}/D)} . \v{z} =  & M_{fz} + (-x+\frac{Leng}{2}) Y_A +(Leng + 64 - x)Y_B  &= 0\\
\end{array}
\right.
$$
\newline
$$  \left\{
\begin{array}{ccc}
 N   = - X_B \\
 T_y = - Y_A - Y_B  \\
 T_z = - Z_A - Z_B \\
\end{array}
\right.
$$


$$ \left\{
\begin{array}{ccc}
 M_T =& 0 \\
 M_{fy} =& (-Z_A-Z_B)x -\frac{Leng}{2}Z_A + (Leng +64) Z_B  \\
 M_{fz} =& (Y_B+Y_A)x  -\frac{Leng}{2}Y_A - (Leng + 64)Y_B \\
\end{array}
\right.
$$


\n
Vérification :\n

$$\frac{dM_{fy}}{dx} = -Z_A - Z_B = T_Z $$
$$ \frac{dM_{fz}}{dx} = Y_B+ Y_A = -T_Y$$

\iImage{img/torseur_cohesion.png}{Point d'application de la charge}{0.7}

\newpage
%%%%%%%%%%%%%%%%%%%%%%%%%%%%%%%%%%%%%%%%%%%%%%%%%%%%%%%%%%%%%%%%%%%%%%%%%%%%%%%%%%%%%%%%%%
%%%%%%%%%%%%%%%%%%%%%%%%%%%%%%%%%%%%%%%%%%%%%%%%%%%%%%%%%%%%%%%%%%%%%%%%%%%%%%%%%%%%%%%%%%
%%%%%%%%%%%%%%%%%%%%%%%%%%%%%%%%%%%%%%%%%%%%%%%%%%%%%%%%%%%%%%%%%%%%%%%%%%%%%%%%%%%%%%%%%%
\section{Matrice de contrainte}

\subsection{Traction}

\begin{bmatrix}
\sigma_t(M)
\end{bmatrix}
=\emat{\sigma_t & 0 & 0}{0 & 0 & 0}{0 & 0 & 0}
\n

$$\sigma_t= \frac{N}{S}=\frac{4 N }{\pi D^2}$$

\subsection{Flexion}

Avec $M_f=M_{fz}$ \n

\begin{bmatrix}
\sigma_{f}(M)
\end{bmatrix}
=\emat{\sigma_{M_f} & 0 & 0}{0 & 0 & 0}{0 & 0 & 0}

\n

$$\sigma_{Mf}= -\frac{M_f}{\frac{\pi D^4}{64}}\cdot \frac{D}{2}$$
$$\sigma_{Mf}= -\frac{32M_{f}}{\pi D^3}$$

On est au niveau d'un épaulement donc : \sskip
$ \frac{D}{d} = \frac{20}{17} = 1.18 $ \sskip
$\frac{r}{d} = \frac{2}{17} = 0.12$ \n

$\Rightarrow Ktf=1.5$
\subsection{Cisaillement}

Avec $M_f=M_{fz}$ \n

\begin{bmatrix}
\sigma_{ci}(M)
\end{bmatrix}
=\emat{0 & r_{ci} & 0}{r_{ci} & 0 & 0}{0 & 0 & 0}

$ r_{ci} = - \frac{F}{S} = -\frac{4N}{\pi D^2}$ \n

Torsion : $Mt=0$
\subsection {Principe de superposition}

\begin{bmatrix}
\sigma_{(M)}
\end{bmatrix}
=\emat{\sigma_t+\sigma_{Mf} & r_{ci} & 0}{r_{ci} & 0 & 0}{0 & 0 & 0}
\n

\subsection{Condition de résistance}

$r_{torsion\ idéale} \leq R_{pg }= \frac{R_{eg}}{K_s}$ \n

avec $R_{eg}=\frac{R_e}{2}$ 

$$r_{ti}=  \frac{16 \cdot \sqrt{{M_f}^2 + {M_t}^2}}{\pi D^3} \leq R_{pg} = \frac{R_e}{2K_s} $$


 $$\Rightarrow \frac{32 K_s M_{fz} K_{tf}}{\pi R_e} \leq D^3$$

\n 
$K_{tf} $ lié à l'épaulement 

$ D_{min} = (\frac{32\cdot K_s \cdot K_{tf} \cdot M_{fz}}{\pi R_e})^{\frac{1}{3}} = {(\frac{96 \cdot M_{fz}}{\pi \cdot 300\cdot 10^{6}})}^{\frac{1}{3}} $\n

Résultat obtenu avec EXCEL :  
\iImage{img/RDM.png}{Point d'application de la charge}{0.7}

\chapter{Conclusion}
A travers ce projet nous avons pu réinvestir les connaissances acquises au cours des précédents
semestres, notamment le Principe Fondamentale de la Statique mais également revoir la démarche de dimensionnement d'une pièce avec la détermination des efforts et l'application des contraintes avec les critères de Tresca.\n
    Nous avons également utilisé des   
connaissances générale sur les principe électrique 
et mécanique que sont les rapports entre 
puissance,énergie et force, ceci afin de choisir le 
matériel adéquat aux contraintes imposées.
    Nous avons également eu une réflexion sur les 
choix
technologiques (arbre double, système roue vis sans, 
frein à manque de courant, etc...)  ceci dans une 
démarche de réduction des coûts et de l'utilisation des matières 
premières. \n

Nous avons aussi abordé la mise en plan, à travers plusieurs schémas de principe que ce soit pour la modélisation des liaisons mais aussi, la répartition des efforts, ou encore une ébauche de système technologique. \n

Ceci permettant de nous questionner sur les choix pratique de montage et d'assemblage du réducteur.\n



\chapter{Annexes}

\subsubsection{Croquis pour PFS}
\iImage{\imRoot schema_treuil.png}{Croquis}{0.15}

\subsubsection{Image de principe}
\iImage{\imRoot shema_principe.jpg}{Schéma de principe}{0.55}
% arg 1 : chemin image
% arg 2 : description
% arg 3 : taille



\end{document}