%################################################################
%### Fichier Header contenant de nombreux outils latex        ###
%################################################################
%### Classe du document : report | article | slide | book     ###
\documentclass[12pt]{report}                                  %##
%################################################################
%### Modules à importer                                       
\usepackage[utf8]{inputenc}     %UTF-8
\usepackage{hyperref}           %Url
\usepackage[frenchb]{babel}     %Langue française
\usepackage{graphicx}           %Images
\usepackage{caption}            %légende
\usepackage{textcomp}
\usepackage{color}              %Couleurs
\usepackage{fancyhdr}           %headers & footers
\usepackage{lastpage}           %Nombre de page
\usepackage{listings}           %code source
\usepackage{xcolor}             %Couleurs personnalisées
\usepackage{fullpage}
\usepackage{float}              %Mise en page des images
\usepackage{xcolor}             %Couleurs personnalisées
\usepackage{amsmath}
\usepackage{tikz,pgfplots,pgf}
\usepackage{fullpage}
\usepackage{pgfplots}
\usepackage{tikz}
\usepackage{titlesec}
\usepackage{ENIB}
\usetikzlibrary{patterns,calc,positioning}
\hypersetup{colorlinks=true,urlcolor=blue,linkcolor=black}

%################################################################
%### Définition du style de page 'classic'
%### Toute image PNG contenu dans le dossier Images prendra place dans l'en-tête de chaque page 
\fancypagestyle{classic}{
    \rhead{\leftmark}   %Ajout du titre du chapitre courant
    \lhead{\small \includegraphics[scale=0.2]{Images/logo_enib.png}} %Ajout du logo 
    \renewcommand{\footrulewidth}{0.4pt} %Ligne du pied de page
    \renewcommand{\headrulewidth}{0.4pt} %Ligne de l'en-tête
    \rfoot{\thepage/\pageref{LastPage}}  %Page courante / Nombre de page
    \cfoot{}
    %Chaque auteur est séparé par un  '\\' en cas de manque de place
    \lfoot{\small Antoine SCAVINER - Théo MAINGUENÉ \\ Mathieu CHARLES - Nicolas LE GUERROUÉ}
}
\setlength{\headsep}{0.2in}
%################################################################
%### Déclaration du style de page global
\makeatletter
\renewcommand\chapter{\if@openright\cleardoublepage\else\clearpage\fi
                      \thispagestyle{classic} %Thème 'classic'
                      \global\@topnum\z@
                      \@afterindentfalse
                      \secdef\@chapter\@schapter}
\makeatother
\pagestyle{classic} %Thème 'classic'
%################################################################
%### Déclaration des titres de l'en-tête
\makeatletter
\renewcommand{\@chapapp}{Section}   %Le mot 'Chapitre' est remplacé par 'Section'
%################################################################


%### Commande de mise en forme
\renewcommand{\bold}[1]{{\bfseries #1}}
%\bold{texte en gras}

\newcommand{\italic}[1]{\textit{#1}}
%\italic{texte en italique}

\newcommand{\ib}[1]{{\bfseries\itshape #1}}
%\ib{texte en gras et italique}

\newcommand{\iImage}[3]{\begin{figure}[H]\centering\includegraphics[scale=#3]{#1}\caption{#2}\end{figure}}
%\iImage{chemin_src}{description}{echelle}

\newcommand{\n}{\newline}
%\n         retour à la ligne

\newcommand{\sskip}{\vskip 0.5cm}
%\sskip     Petit saut de ligne

\renewcommand*{\v}[1]{\vbox{\halign{##\cr      %vecteur \v{}
 \tiny\rightarrowfill\cr\noalign{\nointerlineskip\vskip1pt} 
  $#1\mskip2mu$\cr}}}

%################################################################


%### Définition de la page de garde classique                 ###
\renewcommand{\header[3]}{
\title{#1}
\author{#2}
\date{#3}
\maketitle
}
%###############################################################
%### Définition de la page de garde avec une image en plus
%### \headerimage{image_src}{echelle_img}{titre}{sous-titre}{Auteur}{Date}
\renewcommand{\headerimage[6]}{

\begin{titlepage}
  \begin{sffamily}
  \begin{center}
    \includegraphics[scale=#2]{#1}~\\[1.5cm]
    \hfill
\HRule \\[0.4cm]
{ \Huge \bfseries #3\\[0.4cm] }
\textsc{#4}\\[1.5cm]

\HRule \\[0.4cm]
{ \big \bfseries #5\\[0.4cm] }

    \vfill
    {\large #6}
  \end{center}
  \end{sffamily}
\end{titlepage}
}
%###############################################################
%### Couleurs programmation
\definecolor{green}{rgb}{0,0.6,0}
\definecolor{gray}{rgb}{0.5,0.5,0.5}
\definecolor{purple}{rgb}{0.58,0,0.82}
\definecolor{bg}{rgb}{0.98,0.98,0.98}
%###############################################################
%### définition du style 'Python' [begin{Python}]
\lstdefinestyle{Python}{
    backgroundcolor=\color{bg},   
    commentstyle=\color{green},
    keywordstyle=\color{magenta},
    numberstyle=\tiny\color{gray},
    stringstyle=\color{purple},
    basicstyle=\ttfamily\footnotesize,
    breakatwhitespace=false,         
    breaklines=true,                 
    captionpos=b,                    
    keepspaces=true,                 
    numbers=left,   
    numbersep=5pt,                  
    showspaces=false,                
    showstringspaces=false,
    showtabs=false,                  
    tabsize=2
}
\lstset{style=Python}
\lstset{literate=
  {á}{{\'a}}1 {é}{{\'e}}1 {í}{{\'i}}1 {ó}{{\'o}}1 {ú}{{\'u}}1
  {Á}{{\'A}}1 {É}{{\'E}}1 {Í}{{\'I}}1 {Ó}{{\'O}}1 {Ú}{{\'U}}1
  {à}{{\`a}}1 {è}{{\`e}}1 {ì}{{\`i}}1 {ò}{{\`o}}1 {ù}{{\`u}}1
  {À}{{\`A}}1 {È}{{\'E}}1 {Ì}{{\`I}}1 {Ò}{{\`O}}1 {Ù}{{\`U}}1
  {ä}{{\"a}}1 {ë}{{\"e}}1 {ï}{{\"i}}1 {ö}{{\"o}}1 {ü}{{\"u}}1
  {Ä}{{\"A}}1 {Ë}{{\"E}}1 {Ï}{{\"I}}1 {Ö}{{\"O}}1 {Ü}{{\"U}}1
  {â}{{\^a}}1 {ê}{{\^e}}1 {î}{{\^i}}1 {ô}{{\^o}}1 {û}{{\^u}}1
  {Â}{{\^A}}1 {Ê}{{\^E}}1 {Î}{{\^I}}1 {Ô}{{\^O}}1 {Û}{{\^U}}1
  {Ã}{{\~A}}1 {ã}{{\~a}}1 {Õ}{{\~O}}1 {õ}{{\~o}}1 {’}{{'}}1
  {œ}{{\oe}}1 {Œ}{{\OE}}1 {æ}{{\ae}}1 {Æ}{{\AE}}1 {ß}{{\ss}}1
  {ű}{{\H{u}}}1 {Ű}{{\H{U}}}1 {ő}{{\H{o}}}1 {Ő}{{\H{O}}}1
  {ç}{{\c c}}1 {Ç}{{\c C}}1 {ø}{{\o}}1 {å}{{\r a}}1 {Å}{{\r A}}1
  {€}{{\euro}}1 {£}{{\pounds}}1 {«}{{\guillemotleft}}1
  {»}{{\guillemotright}}1 {ñ}{{\~n}}1 {Ñ}{{\~N}}1 {¿}{{?`}}1
}
\lstnewenvironment{Python}{\lstset{language=Python}}{}
%###############################################################
%### définition du style 'Cpp' [begin{Cpp}]
\lstdefinestyle{Cpp}{
    basicstyle=\ttfamily,
    columns=fullflexible,
    keepspaces=true,
    upquote=true,
    showstringspaces=false,
    commentstyle=\color{olive},
    keywordstyle=\color{blue},
    identifierstyle=\color{violet},
    stringstyle=\color{purple},
    language=c,
    directivestyle=\color{teal},
}
\lstnewenvironment{Cpp}{\lstset{style=Cpp}}{}

%###############################################################
%### définition du style 'Bash' [begin{Cpp}]
\lstdefinestyle{Bash}{
    basicstyle=\ttfamily,
    columns=fullflexible,
    keepspaces=true,
    upquote=true,
    showstringspaces=false,
    commentstyle=\color{olive},
    keywordstyle=\color{blue},
    identifierstyle=\color{black},
    stringstyle=\color{gray},
    language=bash,
    directivestyle=\color{gray},
}
\lstnewenvironment{Bash}{\lstset{style=Bash}}{}
