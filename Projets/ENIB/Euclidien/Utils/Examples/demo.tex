\documentclass[12pt]{report}
\usepackage[frenchb]{babel}     %french language

\usepackage{../Adding}
\usepackage{../Graphics}
\usepackage{../Layout}
\usepackage{../Maths}
\usepackage{../Programming}


\nomenclature[E]{$r$}{Rapport cyclique d'un signal périodique}
\nomenclature[E]{$r$}{Rapport cyclique d'un signal périodique}

\begin{document}

\title{Exemples Utils}
\author{Nicolas LE GUERROUE}
\maketitle
\tableofcontents
\newpage


\part{test}

\section{Exemples Graphics}


\begin{exemple}
\begin{graphics}{0.8}{0.6}{0}{2.1}{-1.1}{1.1}{t(ms)}{vs}{Oscilloscope}
\addPointsFromCSV{red}{comma}{input_1.txt}
\addPointsFromCSV{blue}{comma}{input_2.txt}
\addLegend{voie A, voie B}
\end{graphics}
\end{exemple}


\begin{exemple}
\begin{graphics}{0.8}{0.4}{0}{40}{-1}{6}{t(s)}{vs}{Courbe avec données brutes}
\addPoints{blue}{(0,0)(10,0)(10,5)(15,5)(15,0)(20,0)(20,5)(25,5)(25,0)(30,0)(30,5)(35,5)(35,0)(100,0)}
\addLegend{Ve}
\end{graphics}
\end{exemple}



\begin{figure}
\begin{subfigure}{.5\textwidth}
  \centering
  % include first image
   \centering
 \begin{graphics}{0.9}{0.4}{-0.05}{20}{-1}{6}{t(ms)}{vs}{KEY\_DOWN}
\addPointsFromCSV{red}{comma}{out.csv}
%\addLegend{Signal "KEY\_POWER"}
\end{graphics}
\end{subfigure}
\begin{subfigure}{.5\textwidth}
  \centering
   \centering
 \begin{graphics}{0.9}{0.4}{-0.05}{80}{-1}{6}{t(ms)}{vs}{KEY\_UP}
\addPointsFromCSV{red}{comma}{out.csv}
%\addLegend{Signal "KEY\_POWER"}
\end{graphics}
\end{subfigure}

\begin{subfigure}{.5\textwidth}
  \centering
 \begin{graphics}{0.9}{0.4}{-0.05}{60}{-1}{6}{t(ms)}{vs}{KEY\_RIGHT}
\addPointsFromCSV{red}{comma}{out.csv}
%\addLegend{Signal "KEY\_POWER"}
\end{graphics}
\end{subfigure}
\begin{subfigure}{.5\textwidth}
  \centering
  % include fourth image
   \centering
 \begin{graphics}{0.9}{0.4}{-0.05}{40}{-1}{6}{t(ms)}{vs}{KEY\_LEFT}
\addPointsFromCSV{red}{comma}{out.csv}
%\addLegend{Signal "KEY\_POWER"}
\end{graphics}
\end{subfigure}
\caption{Put your caption here}
\label{fig:fig}
\end{figure}



\begin{exemple}
\begin{graphics}{0.9}{0.4}{-0.05}{125}{-1}{6}{t(ms)}{vs}{Données capteur IR}
\addPointsFromCSV{cyan}{comma}{out.csv}
\addLegend{Signal "KEY\_POWER"}
\end{graphics}
\end{exemple}


\begin{exemple}
\begin{graphics}{0.4}{0.4}{-10}{10}{-1}{100}{x}{y}{Courbe avec équation}
\addTrace{green}{-10}{10}{x*x}
\addLegend{f(x)}
\end{graphics}
\end{exemple}


\begin{exemple}
\begin{graphics}{0.8}{0.4}{0}{20}{-1}{10}{x}{y}{Courbes de provenances diverses}
\addPointsFromCSV{red}{comma}{input_2.txt}
\addTrace{green}{-10}{10}{x}
\addPoints{blue}{(0,0)(10,0)(10,5)(15,5)(15,0)(20,0)(20,5)(25,5)(25,0)(30,0)(30,5)(35,5)(35,0)(100,0)}
\addLegend{s1,s2,s3}
\end{graphics}
\end{exemple}

\begin{exemple}
\plot{Titre}{x*y*y}
\end{exemple}


\section{Exemples Maths}

$\evec{\v x}{\v y}{\v z}$ \\
$\emat{A}{B}{C} \cdot \emat{A'}{B'}{C'} $


\section{Exemples Programming}

\begin{Python}{Titre code python}
for a in liste:
    lst.remove(a)
if(condition):
    run()
else:
    exit()
\end{Python}

\begin{Cpp}{Code C++}
int a = 0;
float b = 0.0;
a = b*b;
std::cout << a << b <<std::endl;
\end{Cpp}

\begin{Bash}{Exemple Bash}
sudo apt-get update
sudo apt-get -y upgrade
source ~/.bashrc
\end{Bash}


\section{Exemples Layout}


Une onomatopée\index{onomatopée} est une catégorie d'interjections\index{interjection}
émise pour simuler un bruit particulier associé à un être,
un animal ou un objet, par l'imitation des sons\index{son} que ceux-ci produisent



\printnomenclature
\printindex 

\end{document}