
\chapter{La bataille de Waterloo}

\ball{red}{2}
\ball{green}{2}
\ball{red}{2}
\ball{red}{2}
\ball{green}{2}
\ball{green}{2}
\ball{red}{2}
\ball{red}{2}
\ball{red}{2}
\ball{red}{2}

\ball{red}{2}
\ball{red}{2}
\ball{red}{2}
\ball{red}{2}
\ball{red}{2}
\ball{green}{2}
\ball{red}{2}
\ball{green}{2}
\ball{red}{2}
\ball{red}{2}

\lorange{LIB}{OUT}
\lred{LIB}{OUT}
\lgreen{LIB}{OUT}
\lmagenta{LIB}{OUT}
\lpurple{LIB}{OUT}
\lcyan{LIB}{OUT}
\lblue{LIB}{OUT}
\lbrown{LIB}{OUT}
\lyellow{LIB}{OUT}
\lblack{LIB}{OUT}


\section{Introduction}

\badge{white}{black}{Electronique}
\badge{white}{blue}{Mécanique}
\badge{white}{green}{Informatique}

\img{\rootImages/tux.png}{Légende de l'image}{0.5}

% \img{\rootImages/tux.png}{Légende de l'image}{0.7}
% \imgr{\rootImages/tux.png}{Légende de l'image}{0.7}{30}
% \img{\rootImages/tu.png}{Légende de l'image}{0.9}

\foreach \i in {a,b,c,d,e,f,g,h,i,j}{
  \noindent\i\dotfill\\
}

À première vue, La bataille de Waterloo voit la confrontation finale entre deux puissances. Or, la troisième puissance partipante est vite oubliée : il s'agit de la Prusse, nation ayant émergée sous le règne du roi Frédéric.

Dans l'histoire collective, le vainquu a éclipsé le vainqueur. Combien de personnes conaissent le nom du vainqueur de Waterloo ? Sans doute un anglais, mais qui ?
La légende napoléonnienne a fait le reste, seul le dernier carré de la garde et le mot de Cambronne sont restés.



\begin{items}{orange}{\Triangle}
    \item A
    \item B
    \item C
\end{items}

\subsubsection{Introduction}

eu, accumsan eleifend, sagittis quis, diam. Duis eget orci sit amet orci dignissim rutrum.
Nam dui ligula, fringilla a, euismod sodales, sollicitudin vel, wisi. Morbi auctor lorem
non justo. Nam lacus libero, pretium at, lobortis vitae, ultricies et, tellus. Donec aliquet,
tortor sed accumsan bibendum, erat ligula aliquet magna, vitae ornare odio metus a mi.
Morbi ac orci et nisl hendrerit mollis. Suspendisse ut massa. Cras nec ante. Pellentesque
a nulla. Cum sociis natoque penatibus et magnis dis parturient montes, nascetur ridicu-
lus mus. Aliquam tincidunt urna. Nulla ullamcorper vestibulum turpis. Pellentesque cur-
sus



\begin{figure}[!t]
  \centering
  \begin{subfigure}{.22\linewidth}
    \centering
   \begin{tikzpicture}
  \begin{axis}[height=1\linewidth]
    \addplot+[thick,mark=none, color=blue] coordinates{ (0,1)(1,1)(1.01,0)(3,0) };
  \end{axis}
  \end{tikzpicture}
    \caption{Passe-bas}
    \label{fig2:sub1}
  \end{subfigure}%
  \hfill
  \begin{subfigure}{.22\linewidth}
    \centering
     \begin{tikzpicture}
  \begin{axis}[height=1\linewidth]
    \addplot+[thick,mark=none, color=blue] coordinates{(0,0) (0.99,0) (1,1) (3,1)} ;
  \end{axis}
  \end{tikzpicture}
    \caption{Passe-haut}
    \label{fig2:sub2}
  \end{subfigure}\hfill
  \begin{subfigure}{.22\linewidth}
    \centering
     \begin{tikzpicture}
  \begin{axis}[height=1\linewidth]
    \addplot+[thick,mark=none, color=blue] coordinates{(0,0) (0.99,0) (1,1) (2,1) (2.01,0) (3,0)};
  \end{axis}
  \end{tikzpicture}
    \caption{Passe-bande}
    \label{fig2:sub3}
  \end{subfigure}\hfill
  \begin{subfigure}{.22\linewidth}
    \centering
     \begin{tikzpicture}
  \begin{axis}[height=1\linewidth]
    \addplot+[thick,mark=none, color=blue] coordinates{(0,1) (0.99,1) (1,0) (2,0) (2.01,1)(3,1)};
  \end{axis}
  \end{tikzpicture}
    \caption{Coupe bande}
    \label{fig2:sub4}
  \end{subfigure}
  \caption{Filtres en "mur de briques"}
  \label{fig:murbrique}
  \end{figure}

%\chapter{La Bilbiothèque Layout}
hello

  \begin{figure}[h]  
  \centering 
    \begin{subfigure}[b]{0.4\linewidth}
      \begin{graphic}{0.8}{1}{0}{1.2}{-1}{5}{t(s)}{Tension V}{h}
        \addPointsFromCSV{red}{comma}{src_examples/sinus.txt}
        \addLegend{sin(t)}
        \end{graphic}%NO END  LINE HERE
      \caption{No interaction} 
    \end{subfigure}
  \begin{subfigure}[b]{0.4\linewidth}
    \begin{graphic}{0.8}{1}{0}{4}{-0.3}{0.3}{t(s)}{Tension V}{g}
      \addPointsFromCSV{blue}{comma}{src_examples/jack01.txt}
      \addLegend{g(t)}
      \end{graphic}%NO END  LINE HERE
  \caption{Interaction}
  \end{subfigure}
  \caption{Les tensions de service}
  \end{figure}  

Ce document a pour but de présenter les fonctionnalités de la bibliothèque Utils. \n
Chaque bibliothèque doit être indépendante afin de fonctionner correctement.


\chapter{Bibliothèque Layout}
\addQuote{En temps de guerre, la vérité est si précieuse qu'elle doit etre escortée par une garde de mensonge}{Sir Winston Spencer Churchill,\\ Duc de Malborought}
\addQuote{Si Dieu nous fait la grâce de perdre encore une pareille bataille, Votre Majesté peut compter que ses ennemis sont détruits}{Maréchal de Villars, \\ 1709}


\section{Les commandes}

\paragraph{la bibliothèque 'Layout' permet de gérer la disposition et la mise en forme de certains élements du document. \\}

\subsection{Mise en forme de la page de garde avec image}

\italic{Se référer au code source de la page.\footnote{Commande setHeaderImage}}
\index{setHeaderImage}

\subsection{Mise en forme de la page de garde sans image}

\bi{Se référer à la commande setHeader}
\index{setHeader}

\subsection{Ajout d'un trait entre l'en-tête et le corps de la page}

\bi{Se référer à la commande setHeaderLine}
\index{setHeaderLine}

\subsection{Ajout d'un trait entre le corps de la page et le bas de page}

\bi{Se référer à la commande setFooterLine}
\index{setFooterLine}

\subsection{Définition de la présentation globale des pages}

\bi{Se référer à la commande addPresentation}
\index{setFooterLine}


\subsection{Redéfinir les titres des chapitres}

\bi{Se référer à la commande setAliasChapter}
\index{setAliasChapter}

\subsection{Mettre le document en pleine page}

\bi{Se référer à la commande setFullPage}
\index{setFullpage}

\subsection{Mise en gras}

\bold{Texte en gras}
\index{bold}

\subsection{Mise en italique}

\italic{texte en italic}
\index{italic}

\subsection{Mise en gras et italique}

\ib{Texte en gras et italique}
\index{ib}

\subsection{Ajout d'une image non-flottante}

\img{\rootImages/tux.png}{Légende de l'image}{0.5}
\index{img}

\subsection{Ajout d'une image flottante}

Lorem ipsum dolor sit amet, consectetuer adipiscing elit. Ut purus elit, vestibulum
ut, placerat ac, adipiscing vitae, felis. Curabitur dictum gravida mauris. Nam arcu li-
bero, nonummy eget, consectetuer id, vulputate a, magna. Donec vehicula augue eu
neque. Pellentesque habitant morbi tristique senectus et netus et malesuada fames ac
turpis egestas. Mauris ut leo. Cras viverra metus rhoncus sem. Nulla et lectus vestibu-
lum urna fringilla ultrices. Phasellus eu tellus sit amet tortor gravida placerat. Integer
sapien est, iaculis in, pretium quis, viverra ac, nunc. Praesent eget sem vel leo ultrices
bibendum. Aenean faucibus. Morbi dolor nulla, malesuada eu, pulvinar at, mollis ac,
nulla. Curabitur auctor semper nulla. Donec varius orci eget risus. Duis nibh mi, congue

% \imgf{\rootImages/tux.png}{Tux}{0.6}{0.5}
% \index{imgf}

% eu, accumsan eleifend, sagittis quis, diam. Duis eget orci sit amet orci dignissim rutrum.
% Nam dui ligula, fringilla a, euismod sodales, sollicitudin vel, wisi. Morbi auctor lorem
% non justo. Nam lacus libero, pretium at, lobortis vitae, ultricies et, tellus. Donec aliquet,
% tortor sed accumsan bibendum, erat ligula aliquet magna, vitae ornare odio metus a mi.
% Morbi ac orci et nisl hendrerit mollis. Suspendisse ut massa. Cras nec ante. Pellentesque
% a nulla. Cum sociis natoque penatibus et magnis dis parturient montes, nascetur ridicu-
% lus mus. Aliquam tincidunt urna. Nulla ullamcorper vestibulum turpis. Pellentesque cur-
% sus
% luctus
% mauris.

\subsection{Ajout d'un espace vertical}

Lorem ipsum dolor sit amet, consectetuer adipiscing elit. Ut purus elit, vestibulum
ut, placerat ac, adipiscing vitae, felis. Curabitur dictum gravida mauris. Nam arcu li- \sn
\index{sn}
bero, nonummy eget, consectetuer id, vulputate a, magna. Donec vehicula augue eu
neque. Pellentesque habitant morbi tristique senectus et netus et malesuada fames ac
turpis egestas. Mauris ut leo. Cras viverra metus rhoncus sem. Nulla et lectus vestibu-

\subsection{Ajout d'une boite de dialogue}

\messageBox{Message}{orange}{white}{Voici un message}{black}
\index{messageBox}
\messageBox{Message}{red}{white}{Et un autre message}{black}
\messageBox{Message}{green}{white}{bref...}{white}

\section{Les environnements de la bibliothèque}

\subsection{Environnement 'question' }

\begin{question}
Quelle heure est-il ?
\end{question}
\index{question}

\subsection{Environnement 'reponse' }

\begin{reponse}
il est 18 h.
\end{reponse}
\index{reponse}

\subsection{Environnement 'propriete' }

\begin{propriete}
Un produit scalaire est commutatif.
\end{propriete}
\index{propriete}

\subsection{Environnement 'proposition' }

\begin{proposition}
Les chats sont des mammifères.
\end{proposition}
\index{proposition}

\subsection{Environnement 'remarque' }

\begin{remarque}
Les chats sont des mammifères.
\end{remarque}
\index{remarque}

\subsection{Environnement 'exemple' }

\begin{exemple}
Ceci est un exemple d'exemple
\end{exemple}
\index{exemple}

\subsection{Environnement 'definition' }

\begin{definition}
Une phrase est un ensemble de mots.
\end{definition}
\index{definition}


% ###########################################################################
% LINKS
% ###########################################################################

\chapter{Bilbiothèque 'Links'}


\section{Les commandes}


\subsection{définition des liens dans le document}

\bi{Se référer à la commande setParameters}
\index{setParameters}


% ###########################################################################
% MATHS
% ###########################################################################

\chapter{Bilbiothèque 'maths'}


\section{Les commandes}


\subsection{Création d'une matrice 3*3}


$\emat{a & b & c}{d & e & f}{g & h & i}  $
\index{emat}

\subsection{Création d'un vecteur à trois dimensions}
$\evec{a}{b}{c}  $
\index{evec}




% ###########################################################################
% PROGRAMMING
% ###########################################################################

\chapter{Bilbiothèque 'Programming'}


\section{Les environnements}


\subsection{Affichage d'un code C/C++ avec titre}


\begin{Cpp}{Titre}
#include <iostream>

#define CONST 1

int var = 1;
float 

int main() {
  
  call();
  return 0;

}//End main

\end{Cpp}
\index{cpp}
\subsection{Affichage d'un code C/C++ sans titre}


\begin{Cpp}
#include <iostream>

#define CONST 1 #const var

int var = 1;
float g = 2.5;
...

\end{Cpp}


\subsection{Affichage d'un code Python avec titre}


\begin{Python}{Titre du code}
def call(input):

  """docstring"""
  a = input
  for elem in a:
    print(elem) #show
\end{Python}
\index{python}
\subsection{Affichage d'un code Python sans titre}


\begin{Python}
def call(input):

  """docstring"""
  ...
\end{Python}


\subsection{Affichage d'un code Bash avec titre}


\begin{Bash}{Titre du code}
sudo apt-get -y update
sudo apt-get -y upgrade
echo -e "content"
\end{Bash}
\index{bash}
\subsection{Affichage d'un code bash sans titre}

\bold{Ce type d'affichage n'est pas encore supporté par la bibliothèque.}




%###########################################################################
%PROGRAMMING
%###########################################################################

\chapter{Bilbiothèque 'Labels'}



%###########################################################################
%GRAPHIC
%###########################################################################

\chapter{Bilbiothèque 'Graphic'}

\section{Graphiques 2D}
\subsection{Affichage d'un graphique 2D avec insertion des données depuis un fichier txt (csv)}


\begin{graphic}{0.8}{0.6}{0}{2.1}{-1.1}{1.1}{t(ms)}{vs}{Oscilloscope}
\addPointsFromCSV{red}{comma}{src_examples/input_1.txt}
\addPointsFromCSV{blue}{comma}{src_examples/input_2.txt}
\addLegend{voie A, voie B}
\end{graphic}
\index{graphic}
\index{addPointsFromCSV}

\subsection{Affichage d'un graphique 2D avec insertion des données depuis une liste de points}


\begin{graphic}{0.8}{0.4}{0}{40}{-1}{6}{t(s)}{Tension (V)}{Signal numérique}
\addPoints{blue}{(0,0)(10,0)(10,5)(15,5)(15,0)(20,0)(20,5)(25,5)(25,0)(30,0)(30,5)(35,5)(35,0)(100,0)}
\addLegend{Tension (V)}
\end{graphic}


\subsection{Affichage de plusieurs graphiques avec insertion des données depuis un fichier externe}

% \begin{figure}
%   \centering
%   \subfloat[fig 1]{
%     \begin{graphic}{0.9}{0.4}{0.0}{5}{-2}{2}{t(ms)}{vs}{$sin(t)$}
%     \addPointsFromCSV{red}{comma}{src_examples/sinxx.txt}
%     \addPointsFromCSV{blue}{comma}{src_examples/gauss.txt}
%     \addLegend{Sinus}
%     \end{graphic}} 
%   \subfloat[fig 2]{\includegraphics[width = 1in]{\rootImages/tux}}\\
%   \subfloat[fig 3]{\includegraphics[width = 1in]{\rootImages/tux}}
%   \subfloat[fig 4]{\includegraphics[width = 1in]{\rootImages/tux}} 
%   \caption{Add your own figures before compiling}
%   \label{some example}
%   \end{figure}


% \begin{figure}

%     \begin{subfigure}{.30\textwidth}
%      \begin{graphic}{0.9}{0.4}{0.0}{5}{-2}{2}{t(ms)}{vs}{$sin(t)$}
%     \addPointsFromCSV{red}{comma}{src_examples/sinxx.txt}
%     \addPointsFromCSV{blue}{comma}{src_examples/gauss.txt}
%     \addLegend{Sinus}
%     \end{graphic}

%     \begin{graphic}{0.9}{0.4}{0.0}{5}{-2}{2}{t(ms)}{vs}{$sin(t)$}
%       \addPointsFromCSV{red}{comma}{src_examples/gauss.txt}
%       \addLegend{Sinus}
%       \end{graphic}

%     \end{subfigure}
    

%     \begin{subfigure}{.30\textwidth}
%      \begin{graphic}{0.9}{0.4}{0.0}{5}{-2}{2}{t(ms)}{vs}{$sin(t)$}
%     \addPointsFromCSV{red}{comma}{src_examples/gauss.txt}
%     \addLegend{Sinus}
%     \end{graphic}

%     \end{subfigure}

%   \caption{plots of....}
%   \label{fig:fig}
%   \end{figure}

\newpage

% \subsection{Affichage d'un graphique 2D avec insertion des données depuis une équation}



% \begin{graphicFigure}{0.4}{0.4}{0}{3}{-1}{5}{t(s)}{Tension V}{Signal analogique}
% \addPointsFromCSV{red}{comma}{src_examples/input_2.txt}
% \addLegend{AA}
% \end{graphicFigure}


  
\begin{figure}[!h]
  \centering
\begin{tabular}{|c|c|c|c|}
  \hline
  $U_A$ (V) & $U_B$ (V) & Sens du courant & $U_A-U_B$\\
  \hline
  10 & 5 & \colors{blue}{De A vers B} & 5\\
  \hline
  5 & 10 & \colors{blue}{de B vers A} & -5\\
  \hline
  5 & 5 & \colors{blue}{Aucun courant ne circule} & 0\\
  \hline
\end{tabular}
\caption{Réponse sur le sens du courant en fonction des tensions $U_A$ et $U_B$}
\end{figure}



\begin{numeric}{exemple numérique}
  D1 &  20{C}   \\
  D2 &  [green] 1H1L1L1L1H1L1L1H1L1H1L1H1L1H1L1H  \\
  D3 &  [brown] 20{C}   \\
  D4 &  [red] 1H1L1L1L1H1L1L1H1L1H1L1H1L1H1L1H  \\
  D5 &  [blue] 20{C}   \\
  D6 &  [black] 1H1L1L1L1H1L1L1H1L1H1L1H1L1H1L1H  \\
  D7 &  [black] 1H1L1L1L1H1L1L1H1L1H1L1H1L1H1L1H  \\
  D8 & 8D5U7U5D \\
  D9 & LLL 2{0.1H 0.1L} 0.6H HH \\
  D10 & ZZ G ZZ G XX G X \\
  D11 & [d] 4{5D{Text}} 0.2D \\
  D12 & [L][timing/slope=1.0] HL HL HL HL HL \\
\end{numeric}

Notre signal d'horloge (\texttiming{[blue]CCCCCC}) provient d'un oscillateur à quartz.
\texttiming[timing/draw grid]{LHLHLHLHLHLHLHL}


\begin{numeric}{Chronogramme du compteur 4 bits}
  INPUT &  CC [blue]16{CC} CCC   \\
  D0 &  HL 8{LHHL} LHL   \\
  D1 &  H  4{LLLLHHHH} LLLL \\
  D2 &  H 2{LLLLLLLLHHHHHHHH} LLLL   \\
  D3 &  H{LLLLLLLLLLLLLLLLHHHHHHHHHHHHHHHH} LLLL  \\
  END &  LL [green]14{LL} LHHLLL  \\
  VALUE & L 2D{0} 2D{1} 2D{2} 2D{3} 2D{4} 2D{5} 2D{6} 2D{7} 2D{8} 2D{9} 2D{10} 2D{11} 2D{12} 2D{13} 2D{14} 2D{15} 2D{0} 2D{1}  \\
\end{numeric}%

\begin{schema} {Convertisseur Analogique-Numérique}

  \addPower{6,5}{power1}{$+5V$}
  \addGround{4,0}{gnd1}{}

  \setDeviceBackgroundColor{white}
  \setRotate{0}
  \addLogicGate{0,0}{mynor}{nor}{}{A}{B}{G1}

  \setDeviceBackgroundColor{green}
  \addLogicGate{0,2}{mynand}{nand}{}{C}{D}{G2}
  \addLogicGate{2,1}{myor}{or}{}{}{}{G3}
  \resetColors
          
  \addTransistor{6,1}{npnA}{nmos}{B}{C}{E}
  \addTransistor{6,3}{pnpA}{pmos}{b}{e}{c}

  \resetColors
  \addTransistor{10,2}{npnR}{nmos}{b}{e}{c}

  \addWire{mynor.out}{myor.in 2}{\orthogonalWireA}
  \addWire{mynand.out}{myor.in 1}{\orthogonalWireA}

  %\addWire{myor.out}{npnA.B}{\orthogonalWireA}
  \addWire{mynand.out}{pnpA.B}{\orthogonalWireA}
  \addWire{pnpA.C}{npnA.C}{\orthogonalWireA}

  \addWire{pnpA.E}{power1}{\orthogonalWireA}

  \addWire{npnA.E}{gnd1}{\orthogonalWireA}

  \addNode{$(pnpA.C)+(1,0)$}{node1}{}
  \addWire{pnpA.C}{node1}{\orthogonalWireA}

  \setDeviceBackgroundColor{red}
  \addLed{myor.out}{\Right}{npnA.B}{\orthogonalWireA}{L1}
  \addResistor{node1}{\Right}{npnR.B}{\orthogonalWireA}
          
\end{schema}


\begin{schema}{Pont de Graëtz}

  \addNode{0,0}{a}{A}
  \addNode{2.5,2.5}{c}{C}
  \addNode{5,5}{b}{B}
  \addLogicGate{0,2}{mynand}{nand}{}{C}{D}

  \addTransistor{4,1}{npnA}{nmos}{B}{C}{E}

  \addResistor{mynand.out}{1.5,0}{b}{\orthogonalWireA}

  \addPower{1,4}{power1}{$+5V$}
  \addResistor{power1}{\Down}{npnA.B}{\orthogonalWireB}

\end{schema}


\begin{graphic}{0.8}{0.4}{0}{40}{-1}{6}{t(s)}{Tension (V)}{Signal numérique}
\addPoints{blue}{(0,0)(10,0)(10,5)(15,5)(15,0)(20,0)(20,5)(25,5)(25,0)(30,0)(30,5)(35,5)(35,0)(100,0)}
\addLegend{Tension (V)}
\end{graphic}

\begin{tabular}{|c|c||c|}
          \hline
          $A$& $B$&$S$\\
          \hline
          0 & 0 & 0\\
          \hline
          0 & 1 & 1\\
          \hline
          1 & 0 & 1\\
          \hline
          1 & 1 & 1\\
          \hline
\end{tabular}

% \begin{figure}
% \begin{subfigure}{.5\textwidth}
% \centering
% % include first image
% \centering
% \begin{schema}{Pont de Graëtz}
% \setBipolesLength{0.5}
% \addNode{0,0}{a}{A}
% \addNode{2.5,2.5}{c}{C}
% \addNode{5,5}{b}{B}
% \addLogicGate{0,2}{mynand}{nand}{}{C}{D}

% \addTransistor{4,1}{npnA}{nmos}{B}{C}{E}

% \addResistor{mynand.out}{1.5,0}{b}{\orthogonalWireA}

% \addPower{1,4}{power1}{$+5V$}
% \addResistor{power1}{\Down}{npnA.B}{\orthogonalWireB}

% \end{schema}
% \end{subfigure}
% \begin{subfigure}{.5\textwidth}
% \centering
% % \centering
% % \begin{graphics}{0.9}{0.4}{-0.05}{80}{-1}{6}{t(ms)}{vs}{}
% % \addPoints{blue}{(0,0)(10,0)(10,5)(15,5)(15,0)(20,0)(20,5)(25,5)(25,0)(30,0)(30,5)(35,5)(35,0)(100,0)}
% % %\addLegend{Signal "KEY\_POWER"}
% % \end{graphics}
% \end{subfigure}

% \begin{subfigure}{1\textwidth}
% \centering
% \begin{numeric}{Chronogramme du compteur 4 bits}
% INPUT &  CC [blue]16{CC} CCC   \\
% D0 &  HL 8{LHHL} LHL   \\
% D1 &  H  4{LLLLHHHH} LLLL \\
% D2 &  H 2{LLLLLLLLHHHHHHHH} LLLL   \\
% D3 &  H{LLLLLLLLLLLLLLLLHHHHHHHHHHHHHHHH} LLLL  \\
% END &  LL [green]14{LL} LHHLLL  \\
% VALUE & L 2D{0} 2D{1} 2D{2} 2D{3} 2D{4} 2D{5} 2D{6} 2D{7} 2D{8} 2D{9} 2D{10} 2D{11} 2D{12} 2D{13} 2D{14} 2D{15} 2D{0} 2D{1}  \\
% \end{numeric}%

% \end{subfigure}
% \caption{Titre}
% \label{fig:fig}
% \end{figure}


\subsection{Affichage d'un graphique 2D avec insertion des données depuis plusieurs sources}


\begin{graphic}{0.8}{0.4}{0}{20}{-1}{10}{x}{y}{Courbes de provenances diverses}
\addPointsFromCSV{red}{comma}{src_examples/input_2.txt}
\addTrace{green}{-10}{10}{x}
\addPoints{blue}{(0,0)(10,0)(10,5)(15,5)(15,0)(20,0)(20,5)(25,5)(25,0)(30,0)(30,5)(35,5)(35,0)(100,0)}
\addLegend{s1,s2,s3}
\end{graphic}
\index{addTrace}


\section{Graphiques 3D}

\subsection{Affichage d'un graphique 3D avec insertion des données depuis une équation}

\plot{Titre 3D}{x*y*y}
\index{plot}

\chapter{Bilbiothèque 'Adding'}

\section{Nomenclature}

Un exemple de nomenclature est présenté à la page suivante
\section{Exemples Layout}


Une onomatopée\index{onomatopée} est une catégorie d'interjections\index{interjection}
émise pour simuler un bruit particulier associé à un être,
un animal ou un objet, par l'imitation des sons\index{son} que ceux-ci produisent

\nomenclature[E]{$r$}{Rapport cyclique d'un signal périodique}
\nomenclature[A]{$A_d$}{Coefficient d'amplification, gain différentiel }
\nomenclature[A]{$\varepsilon$}{Tension différentielle $(\varepsilon = E_+ - E_-)$\addUnit{V}}
\nomenclature[A]{$E_+$}{Tension entrée non inverseuse \addUnit{V}}
\nomenclature[A]{$E_-$}{Tension entrée inverseuse \addUnit{V}}
\nomenclature[E]{$\eta$}{Rendement d'un mécanisme \addUnit{\%}}
\nomenclature[E]{$\varphi$}{Déphasage entre deux signaux \addUnit{rad}}

\printnomenclature
\index{printnomenclature}
\index{nomenclature}
\index{addUnit}

\printindex 
\index{printindex}

\end{document}